\abschnitt{passing data between fibers}

Data can be transferred between two fibers via global pointer, a calling
wrapper (like \cpp{std::bind}) or lambda capture.
\cppfl{passing_lambda}

The \resume call at line 8 enters the lambda and passes 1 into the
new fiber. The value is incremented by one, as shown at line 4. The expression
\cpp{caller.resume()} at line 5 resumes the original context (represented
within the lambda by \cpp{caller}).\\
The call to \cpp{lambda.resume()} at line 10 resumes the lambda, returning from
the \cpp{caller.resume()} call at line 5. The \fiber instance \cpp{caller}
emptied by the \resume call at line 5 is replaced with the new instance
returned by that same \resume call.\\
Finally the lambda returns (the updated) \cpp{caller} at line 6, terminating its
context.\\

Since the updated \cpp{caller} represents the fiber suspended by the call at
line 10, control returns to \main.\\

However, since context \cpp{lambda} has now terminated, the updated \cpp{lambda}
is empty. Its \opbool returns \cpp{false}.\\

\zs{Using lambda capture is the preferred way to transfer data between two
fibers; global pointers or a calling wrapper (such as \cpp{std::bind}) are
alternatives.}
