\abschnitt{passing data between fibers}

Data can be transferred between two fibers via global pointer, calling
wrappers (like \cpp{std::bind}) or lambda captures.
\cppfl{passing_lambda}

The \resume call at line 9 enters the lambda and passes 1 into the
new fiber. The value is incremented by one, as shown at line 5. The expression
\cpp{caller.resume()} at line 6 resumes the original context (represented
within the lambda by \cpp{caller}).\\
The call to \cpp{lambda.resume()} at line 11 resumes the lambda, returning from
the \cpp{caller.resume()} call at line 6. The \fiber instance \cpp{caller}
invalidated by the \resume call at line 6 is in-place assigned by the
synthesized fiber.\\
Finally the lambda returns (the updated) \cpp{caller} at line 7, terminating its
context.\\

Since the updated \cpp{caller} represents the fiber suspended by the call at
line 11, control returns to \main.\\

However, since context \cpp{lambda} has now terminated, the updated \cpp{lambda}
is invalid. Its \opbool returns \cpp{false}; its \cpp{operator\!()} returns
\cpp{true}.\\

\zs{Using lambda capture is the preferred way to transfer data between two
fibers; global pointer and calling wrappers (like \cpp{std::bind}) are
alternatives.}
