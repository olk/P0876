\newpage
\abschnitt{Appendix A: support code for examples}\label{launch}\label{appendixa}

Destroying a non-empty \fiber instance invokes Undefined Behaviour -- you must
first call \reqstop (see \nameref{termination}). To simplify code examples in
this paper, we introduce an \cpp{autocancel} wrapper class that tracks the
sequence of \fiber instances representing a particular fiber. When
an \cpp{autocancel} instance is destroyed, it calls \reqstop on the
stored \fiber and loops until the fiber voluntarily terminates.

\cppf{autocancel}
