%//////////////////////////////////////////////////////////////////////////////

\documentclass[fontsize=10pt,paper=A4,pagesize,DIV=15]{scrartcl}

\usepackage[T1]{fontenc}
\usepackage[utf8]{inputenc}
\usepackage[american]{babel}        % required for ISO dates
\usepackage[iso,american]{isodate}  % ISO format of dates
\usepackage[final]{listings}        % code listings
\usepackage{booktabs}               % fancy tables
\usepackage[color]{changebar}       % changebars for large inserted passages
\usepackage{longtable}              % auto breaking tables
\usepackage{ltcaption}              % fix captions for long tables
\usepackage{relsize}                % provide relative font size changes
%\usepackage{underscore}             % remove special status of '_' in ordinary text
%\usepackage{verbatim}               % improved verbatim environment
\usepackage{parskip}                % handle non-indented paragraphs "properly"
\usepackage{array}                  % new column definitions for tables
\usepackage[normalem]{ulem}         % underline commands
\usepackage{xcolor}                 % driver-independent color extensions
\usepackage{amsmath}                % mathematical symbols
\usepackage{mathrsfs}               % mathscr font
\usepackage{xspace}                 % inserts a space to replace one "eaten" by TeX
\usepackage[final]{microtype}       % micro-typographic extensions introduced by pdfTeX
\usepackage{xstring}                % manipulating strings
\usepackage{fixme}                  % collaborative annotations
\usepackage{multicol}               % intermix single and multiple columns
\usepackage{perpage}                % counter reset at every page boundary
\usepackage{palatino}               % Adobe Palatino font
\usepackage{overcite}               % citations
\usepackage{boxedminipage}          % framed mini-pages
\usepackage{fancyhdr}               % control of page headers and footers
\usepackage{soul}                   % hyphenatable spacingout), underlining, striking out, et.
\usepackage{svg}                    % SVG pictures
\usepackage{tikz}                   % creating PS and PDF graphics
\usetikzlibrary{arrows,automata}

\cbcolor{green}

\usepackage[pdftex,
            pdftitle    = {thread\_local means fiber-specific},
            pdfsubject  = {},
            pdfauthor   = {Nat Goodspeed},
            pdfkeywords = {C++,fiber_context,thread_local},
            bookmarks=true,
            bookmarksnumbered=true,
            pdfpagelabels=true,
            pdfpagemode=UseOutlines,
            pdfstartview=FitH,
            linktocpage=true,
            colorlinks=true,
            linkcolor=blue,
            plainpages=false
           ]{hyperref}

%//////////////////////////////////////////////////////////////////////////////

\setlength{\parindent}{0pt} 
\renewcommand\sfdefault{phv}

\makeatletter
    \renewcommand*\l@subsection{\@dottedtocline{2}{0em}{2.3em}}
    \renewcommand*\l@subsection{\@dottedtocline{3}{0em}{3.2em}}
    \renewcommand{\tableofcontents}{\@starttoc{toc}}
\makeatother

\MakePerPage{footnote}
\renewcommand*{\thefootnote}{\fnsymbol{footnote}}

\newcommand{\pdfimg}[1]{\pdfximage{pics/#1}\pdfrefximage\pdflastximage}
\newcommand{\img}[1]{\mbox{\pdfimg{#1}}}
\newcommand{\imgc}[1]{\begin{center}\img{#1}\end{center}}
\newcommand{\graph}[1]{\input{graphs/#1}}
\newcommand{\graphc}[1]{\begin{center}\graph{#1}\end{center}}
\newcommand{\bfs}[1]{{\bfseries #1}}
\newcommand{\zs}[1]{\begin{boxedminipage}[t]{16.8cm}\bfs{#1}\end{boxedminipage}}

\newcommand{\cpp}[1]{{\lstinline[
		basicstyle=\ttfamily\small\color{black},
        breakatwhitespace=true,
        breaklines=true,
        captionpos=b,
        commentstyle=\ttfamily\color{red},
        keywordstyle=\ttfamily\color{blue},
        language={C++},
        morekeywords={},
        showspaces=false,
        showstringspaces=false,
        showtabs=false,
        stringstyle=\ttfamily\color{magenta}
] !#1!}\xspace}
\newcommand{\cppf}[1]{\lstinputlisting[
		basicstyle=\ttfamily\small\color{black},
        breakatwhitespace=true,
        breaklines=true,
        captionpos=b,
        commentstyle=\ttfamily\color{red},
        keywordstyle=\ttfamily\color{blue},
        language={C++},
        morekeywords={},
        showspaces=false,
        showstringspaces=false,
        showtabs=false,
        stringstyle=\ttfamily\color{magenta}
] {code/#1.cpp}}
\newcommand{\cppfl}[1]{\lstinputlisting[
		basicstyle=\ttfamily\small\color{black},
        breakatwhitespace=true,
        breaklines=true,
        captionpos=b,
        commentstyle=\ttfamily\color{red},
        keywordstyle=\ttfamily\color{blue},
        language={C++},
        morekeywords={},
        numbers=left,
        showspaces=false,
        showstringspaces=false,
        showtabs=false,
        stringstyle=\ttfamily\color{magenta}
] {code/#1.cpp}}

\newcommand{\dtor}{\cpp{\~fiber\_context()}}
\newcommand{\main}{\cpp{main()}}
\newcommand{\fiber}{\cpp{std::fiber\_context}}
\newcommand{\op}{\cpp{operator()()}}
\newcommand{\opbool}{\cpp{operator bool()}}
\newcommand{\resume}{\cpp{resume()}}
\newcommand{\resumewith}{\cpp{resume\_with()}}
\newcommand{\xtresume}{\cpp{resume\_other\_thread()}}
\newcommand{\xtresumewith}{\cpp{resume\_other\_thread\_with()}}
\newcommand{\usessysstack}{\cpp{uses\_system\_stack()}}
\newcommand{\prevtid}{\cpp{previous\_thread()}}
\newcommand{\thread}{\cpp{std::thread}}
\newcommand{\unwindex}{\cpp{std::unwind\_exception}}
\newcommand{\unwindfib}{\cpp{std::unwind\_fiber()}}

\newcommand{\sym}{\emph{symmetric}\xspace}
\newcommand{\asym}{\emph{asymmetric}\xspace}
\newcommand{\entryfn}{\emph{entry-function}}

\newcommand{\abschnitt}[1]{\addcontentsline{toc}{subsection}{#1}\subsection*{#1}}
\newcommand{\uabschnitt}[1]{\paragraph*{#1}}

\newcommand{\bcontext}{
        \href{http://www.boost.org/doc/libs/release/libs/context/doc/html/index.html}
        {\emph{Boost.Context}}}
\newcommand{\bcoroutine}{
        \href{http://www.boost.org/doc/libs/release/libs/coroutine2/doc/html/index.html}
        {\emph{Boost.Coroutine2}}}
\newcommand{\bfiber}{
        \href{http://www.boost.org/doc/libs/release/libs/fiber/doc/html/index.html}
        {\emph{Boost.Fiber}}}
\newcommand{\fbmcrouter}{
        \href{https://code.facebook.com/posts/296442737213493/introducing-mcrouter-a-memcached-protocol-router-for-scaling-memcached-deployments}
        {\emph{mcrouter}}}
\newcommand{\fbfibers}{
        \href{https://github.com/facebook/folly/tree/master/folly/fibers}
        {\emph{folly::fibers}}}
\newcommand{\fbthrift}{
        \href{https://github.com/facebook/fbthrift}
        {\emph{Thrift}}}
\newcommand{\synca}{
        \href{https://github.com/gridem/Synca}
        {\emph{Synca}}}


%//////////////////////////////////////////////////////////////////////////////

\begin{document}
\small
\begin{tabbing}
    Document number: \= P3346R0\\
    Date:            \> 2024-06-28\\
    Author:          \> Nat Goodspeed (nat@lindenlab.com)\\
    Audience:        \> LEWG, EWG\\
\end{tabbing}

\section*{thread\_local means fiber-specific}

%//////////////////////////////////////////////////////////////////////////////

\tableofcontents

%//////////////////////////////////////////////////////////////////////////////

\abschnitt{abstract}\label{abstract}

P0876R17\cite{P0876R17} does not specify any changes to the semantics of
\stdterm{thread storage duration}{basic.stc.thread}. The implication is that
if two fibers are running on the same thread, they will both share the same
value of a given \tlocal variable. Each fiber will see modifications made by
the other fiber. They will not \stdterm{race}{intro.races}, since at every
moment a specific thread is running exactly one fiber.

Nonetheless, this could be problematic for an existing library that relies on
\tlocal variables if multiple fibers on the same thread take turns making
calls into that library. From the library's point of view, the value of its
\tlocal variables might change whimsically.

This is analogous to the problem faced by libraries relying on \cpp{static}
variables when threads were first introduced, albeit without the UB resulting
from data races.

It is suggested that every \tlocal variable should have a distinct value for
each fiber that accesses it. This paper details changes to P0876R17 and to
the Standard\cite{Standard} to express that functionality.

\abschnitt{Revision History}\label{history}
Initial revision.

%%This document supersedes P3346R0.
%%
%%\uabschnitt{Changes since P3346R0}
%%
%%\begin{itemize}
%%    \item Update to reference N4981.
%%\end{itemize}

%%\newpage

\abschnitt{Wording}\label{api}

This wording is relative to N4981\cite{Standard} and P0876R17.\cite{P0876R17}

\zs{Modify \stdsection{6.7.5.3}{basic.stc.thread} as follows:}

1 All variables declared with the \cpp{thread\_local} keyword have
\emph{thread storage duration}.
The storage for these entities lasts for the duration of
the \replace{thread}{fiber} in which they are created. There is a distinct object or reference
per \replace{thread}{fiber}, and use of the declared name refers to the entity associated with
the current \replace{thread}{fiber}.

2 \tsnoten{1}{A variable with thread storage duration is initialized as specified
in \stdclause{basic.start.static}, \stdclause{basic.start.dynamic}, and \stdclause{stmt.dcl}
and, if constructed, is destroyed on \replace{thread}{fiber} exit \xref{basic.start.term}.}

\zs{Modify \stdsection{6.9.3.4}{basic.start.term} paragraph 2 as follows:}

2 Constructed objects with thread storage duration within a given \replace{thread}{fiber}
are destroyed as a result of returning from the initial function of that \replace{thread}{fiber} and as a
result of that \replace{thread}{fiber} calling \cpp{std::exit}.
The destruction of all constructed objects with thread storage
duration within that \replace{thread}{fiber} strongly happens before destroying
any object with static storage duration.

\zs{Modify \stdsection{11.4.9.3}{class.static.data} paragraph 1 as follows:}

1 A static data member is not part of the subobjects of a class. If a
static data member is declared \cpp{thread\_local} there is one copy of
the member per \replace{thread}{fiber}. If a static data member is not declared
\cpp{thread\_local} there is one copy of the data member that is shared by all
the objects of the class.

\zs{Modify \stdsection{17.5}{support.start.term} paragraph 9.1 as follows:}

(9.1) --- First, objects with thread storage duration and associated with the current
\replace{thread}{thread's default fiber}
are destroyed. Next, objects with static storage duration are destroyed
and functions registered by calling
\cpp{atexit}
are called.\textsuperscript{191}

See \stdclause{basic.start.term} for the order of destructions and calls.
(Objects with automatic storage duration are not destroyed as a result of calling
\cpp{exit()}.)\textsuperscript{192}

If a registered function invoked by \cpp{exit} exits via an exception,
the function \cpp{std::terminate} is invoked\xref{except.terminate}.

\zs{Modify \stdsection{33.10.10.2}{futures.task.members} paragraph 23 as follows:}

23 \effects
As if by \emph{INVOKE}\cpp{<R>(f, t}$_1$\cpp{, t}$_2$, $\dotsc$\cpp{, t}$_N$\cpp{)}\xref{func.require},
where \cpp{f} is the stored task and
\cpp{t}$_1$\cpp{, t}$_2$, $\dotsc$\cpp{, t}$_N$ are the values in \cpp{args...}. If the task returns normally,
the return value is stored as the asynchronous result in the shared state of
\cpp{*this}, otherwise the exception thrown by the task is stored. In either
case, this is done without making that state ready\xref{futures.state} immediately. Schedules
the shared state to be made ready when the current thread exits,
after all objects of thread storage duration associated with the current \replace{thread}
{thread's default fiber} have been destroyed.

\zs{Append to ???? in P0876R17:}

\cbstart

\rSec2[thread.ptr]{thread\_specific\_ptr}

\rSec3[thread.ptr.overview]{Overview}

1 Objects with thread storage duration now have a distinct instance for each
fiber within a thread, and are destroyed when the fiber terminates. It is
sometimes desirable to access storage that is shared between all fibers on a
thread, but distinct for each referencing thread.

2 The \tptr class manages pointers, one per referencing thread. This can be
used to access a distinct object with dynamic storage duration for each
thread, that is nonetheless shared between all fibers on that thread.

%--------------------------------- synopsis ----------------------------------
\rSec3[thread.ptr.synopsis]{Header <thread\_specific\_ptr> synopsis}

\cppf{synopsis.tls}

%--------------------------------- class def ---------------------------------

thread_specific_ptr construct/copy/destruct
thread_specific_ptr();
Requires:	The expression delete get() is well formed.
Effects:	A thread-specific data key is allocated and visible to all threads in the process. Upon creation, the value NULL will be associated with the new key in all active threads. A cleanup method is registered with the key that will call delete on the value associated with the key for a thread when it exits. When a thread exits, if a key has a registered cleanup method and the thread has a non-NULL value associated with that key, the value of the key is set to NULL and then the cleanup method is called with the previously associated value as its sole argument. The order in which registered cleanup methods are called when a thread exits is undefined. If after all the cleanup methods have been called for all non-NULL values, there are still some non-NULL values with associated cleanup handlers the result is undefined behavior.
Throws:	boost::thread_resource_error if the necessary resources can not be obtained.
Notes:	There may be an implementation specific limit to the number of thread specific storage objects that can be created, and this limit may be small.
Rationale:	The most common need for cleanup will be to call delete on the associated value. If other forms of cleanup are required the overloaded constructor should be called instead.
thread_specific_ptr(void (*cleanup)(void*) cleanup);
Effects:	A thread-specific data key is allocated and visible to all threads in the process. Upon creation, the value NULL will be associated with the new key in all active threads. The cleanup method is registered with the key and will be called for a thread with the value associated with the key for that thread when it exits. When a thread exits, if a key has a registered cleanup method and the thread has a non-NULL value associated with that key, the value of the key is set to NULL and then the cleanup method is called with the previously associated value as its sole argument. The order in which registered cleanup methods are called when a thread exits is undefined. If after all the cleanup methods have been called for all non-NULL values, there are still some non-NULL values with associated cleanup handlers the result is undefined behavior.
Throws:	boost::thread_resource_error if the necessary resources can not be obtained.
Notes:	There may be an implementation specific limit to the number of thread specific storage objects that can be created, and this limit may be small.
Rationale:	There is the occasional need to register specialized cleanup methods, or to register no cleanup method at all (done by passing NULL to this constructor.
~thread_specific_ptr();
Effects:	Deletes the thread-specific data key allocated by the constructor. The thread-specific data values associated with the key need not be NULL. It is the responsibility of the application to perform any cleanup actions for data associated with the key.
Notes:	Does not destroy any data that may be stored in any thread's thread specific storage. For this reason you should not destroy a thread_specific_ptr object until you are certain there are no threads running that have made use of its thread specific storage.
Rationale:	Associated data is not cleaned up because registered cleanup methods need to be run in the thread that allocated the associated data to be guarranteed to work correctly. There's no safe way to inject the call into another thread's execution path, making it impossible to call the cleanup methods safely.
thread_specific_ptr modifier functions
T* release();
Postconditions:	*this holds the null pointer for the current thread.
Returns:	this->get() prior to the call.
Rationale:	This method provides a mechanism for the user to relinquish control of the data associated with the thread-specific key.
void reset(T* p = 0);
Effects:	If this->get() != p && this->get() != NULL then call the associated cleanup function.
Postconditions:	*this holds the pointer p for the current thread.
thread_specific_ptr observer functions
T* get() const;
Returns:	The object stored in thread specific storage for the current thread for *this.
Notes:	Each thread initially returns 0.
T* operator->() const;
Returns:	this->get().
T& operator*()() const;
Requires:	this->get() != 0
Returns:	this->get().

\cbend

%%\abschnitt{\fiber and the larger C++ ecosystem}

\uabschnitt{higher-level libraries}

\nameref{low_level} enumerates a number of higher-level abstraction libraries
built upon the \bcontext\xspace implementation of the API proposed in this paper.
This is not an exhaustive list, but it suffices to illustrate that there is
widespread interest in this functionality.

The most significant point about this proposal is that, given \fiber, all
those libraries can be written in standard C++. They need not themselves be
integrated into the Standard.

Because it creates and switches between different function call stacks,
though, the \fiber facility cannot be written in portable C++. There is real
value to integrating this library into the Standard.

\emph{Boost.Context} is maintained by one individual to support the specific
set of processors and operating systems to which he has access. The \fiber
facility will ensure support in every implementation of the C++ runtime,
extending into the future.

Given the lively ecosystem of open-source libraries, it's possible that
standardizing \fiber could suffice. It is not essential that
WG21 must standardize additional higher-level libraries before the facility
would become useful. The uptake of \emph{Boost.Context} illustrates that the
community can make good use of \fiber.

However, the evolution of this proposal and the WG21 discussions thereof have
surfaced a number of interesting adjacencies.

\uabschnitt{cancellation}

Given C++ support for concurrency, in various forms, within a program,
cancellation of an asynchronous task remains a topic of widespread interest.
It has been much discussed, e.g. in P1677R2\cite{P1677R2},
P1820R0\cite{P1820R0} and P2175R0\cite{P2175R0}.

Previous revisions of this paper have proposed canceling a suspended fiber by
injecting an exception, e.g. using \fiber\cpp{::}\resumewith. A comparable
approach was rejected for \cpp{std::jthread}, although it's worth noting that
cooperative fibers differ in a very significant respect: every fiber suspends
at a well-defined point, namely a call to \resumewith.\footnote{Although
exception-based cancellation is not implicitly supported, a consumer of \fiber
may still explicitly pass to \resumewith an invocable that raises an exception
in the suspended fiber.}

Evolution of the exception mechanism itself\cite{P0709R4} may affect the
viability of using exceptions for cancellation.

This paper simply notes that an invoker can use lambda binding to pass (e.g.)
a \cpp{std::stop\_token} from the Standard\cite{Standard}, section 33.3, to a
fiber at launch time.

\uabschnitt{modules and optimizations}

Before modules, the only information the compiler could know about a function
in an external translation unit was what a human coder stated in the relevant
header file. But since the information in a module is prepared by the compiler
itself, a subsequent compile of a translation unit that imports that module
can know as much about each module function as it would if the function's
source code was found within the current translation unit.

This permits the compiler to infer and propagate attributes. If a function
neither contains a throw statement nor calls other functions, the compiler can
conclude that it doesn't throw. It can encode this information in the module
produced for that translation unit, so that subsequent compiles can make use
of the knowledge. If another function contains no throw statement and calls
only functions known not to throw, it too can be implicitly marked nothrow.

Similarly, when compiling a function that can never return, the compiler can
so indicate in the output module. Any caller whose code path leads
unconditionally to any such function can also be known never to return.

In much the same way, the module describing the
library's \fiber\cpp{::}\resumewith method can mark it as \emph{can-suspend}.
Then any caller of \resumewith will also be marked \emph{can-suspend}, and so
forth. The compiler can use this to improve its optimization tactics around
any call to a \emph{can-suspend} function.

(The \emph{can-suspend} characteristic of a \cpp{co\_await} coroutine function
is just as pervasive, but in that case the coder must manually propagate it.)

\uabschnitt{synchronization primitives}

The Standard\cite{Standard} provides an assortment of primitives for
synchronizing work between threads, e.g. sections 33.6, 33.7, 33.8, 33.9,
33.10. An essential behaviour of many such synchronization primitives is to
pause, or suspend, execution of the current thread until some external
condition is satisfied.

Such suspension is very different from fiber suspension as proposed in this
paper. This proposal neither requires nor implies a scheduler. A fiber
suspends by explicitly designating the next fiber to resume, either by passing
its \fiber to \resumewith or by returning that \fiber from its \entryfn.

C++ threads, in contrast, assume a thread scheduler, usually provided by the
operating system. Suspending a thread means passing control to the scheduler,
which reallocates CPU resources to other pending threads. At some future time,
the scheduler is responsible for directing some CPU core to resume the suspended
thread.

Fiber suspension as implemented by \fiber is independent of thread suspension.
Suspending the running fiber simply means directing the thread to run a
different fiber; the thread continues running. Conversely, suspending the host
thread (e.g. by invoking a synchronization primitive) means that \emph{no}
fiber is running on that thread.

A higher-level fiber-based library that emulates the \cpp{std::thread} API,
such as \bfiber\cite{bfiber}, necessarily implements a fiber scheduler,
permitting implicit fiber suspension. Standardizing such a library would raise
the interesting question of how to present fiber-aware synchronization
primitives.

A straightforward approach is to present a suite of fiber-aware
synchronization primitives distinct from, but analogous to, the thread-based
synchronization primitives.\footnote{This is the approach taken
by \emph{Boost.Fiber}.} A program running multiple fibers within a thread
would use fiber-aware synchronization primitives rather than thread-based
synchronization primitives. Evaluating a thread-based synchronization
primitive would suspend the entire thread, as usual, halting all fibers within
that thread.

It is tempting to contemplate modifying the semantics of the present suite of
synchronization primitives to make them fiber-aware. Naturally this is a
matter of some concern.

For purposes of this \fiber proposal, though, it is entirely moot.

\uabschnitt{Execution Agent Local Storage}

A similar question arises concerning variable storage duration. Should the
Standard introduce a fiber-specific storage duration, e.g. \cpp{fiber\_local},
analogous to \cpp{thread\_local}\cite{Standard}? (section 6.7.5.3 \bfs{Thread
storage duration})

The Standard defines the general term \emph{execution agent} (section
33.2.5.1) to allow for multiple kinds of parallelism. It seems reasonable to
assume that over time, new types of execution agents will be defined. Will we
want the Standard to present a new \cpp{xyz\_local} storage duration for each
new ``xyz'' execution agent type?

P0772R1\cite{P0772R1} notes that library code should not have to care what
kind of execution agent is running it. Already it's important to ensure that
library code avoids \cpp{static} variables because any such variable prohibits
calling that library from more than one thread. P0772R1 suggests a generalized
variable storage duration dynamically local to the innermost current execution
agent.

(The same consideration about library code impacts the above question about
presenting fiber-aware synchronization primitives.)

It's true that if:

\begin{itemize}
    \item on fiber X, function F relies on a \cpp{thread\_local} variable V
    \item function F calls function G that resumes fiber Y
    \item fiber Y calls function F, or another function that modifies variable V
    \item fiber Y resumes fiber X
    \item on fiber X, function G returns to function F
\end{itemize}

then function F on fiber X will observe fiber Y's value for variable V.

This is analogous to use of a \cpp{static} variable by multiple threads in the
same program -- though not as bad, since it doesn't produce race-related
Undefined Behaviour on top of correctness problems.

\cpp{std::thread} was introduced despite this problem because it's \emph{useful.}

Multiple C++ implementations cache a pointer to thread-local storage in the
stack frame of a function referencing TLS. If a suspended fiber were resumed
by a thread other than the one on which it previously ran, such cached TLS
pointers would point to TLS for the wrong thread. This is why such
cross-thread resumption is forbidden.

(This is the only optimization that has yet been surfaced by implementers as a
potentially problematic interaction with fibers.)

P3346R0\cite{P3346R0} proposed to modify \tlocal to mean fiber-specific. This
was rejected by SG1 in Wrocław in 2024\cite{wroclawp3346}.

That said, in an environment in which \tlocal referenced fiber-specific
storage, TLS pointers cached in function stack frames would remain valid even
if the original fiber were later resumed on some other thread, thus removing
the restriction against cross-thread resumption.

\uabschnitt{tooling} One particularly valuable consequence of adding \fiber to
the Standard will be to add fiber awareness to debuggers, performance
analyzers and other tools that inspect a running C++ program.

Such tools need only be aware of \fiber. They would \emph{not} need to be
further adapted to support higher-level libraries built on
the \fiber facility.

\newpage

%%\abschnitt{control transfer mechanism}

According to literature\cite{Moura2009} the control-transfer operations can be
distinguished into the concepts of \sym and \asym operations.

\uabschnitt{symmetric fiber} A symmetric fiber provides a single
control-transfer operation. This single operation requires that the control is
passed explicitly between the fibers.\\
\graphc{symm}

\cppfl{symmetric_op}

In the pseudo-code example above, a chain of fibers is created.\\
The execution control is transferred to fiber \cpp{f1} at line 15 and the lambda
passed to constructor of \cpp{f1} is entered. The control is transferred from
fiber \cpp{f1} to \cpp{f2} at line 12 and from \cpp{f2} to \cpp{f3} (line 9) and
so on. Fiber \cpp{f4} itself transfers the execution control directly back to
fiber \cpp{f1} at line 3.

\uabschnitt{asymmetric fiber} Two control-transfer operations are part of
asymmetric fiber's interface. One operation for resuming (\resume) and one for
suspending (\cpp{suspend()}) the fiber. The suspending operation returns the
control back to the fiber caller.\\
\graphc{asymm}

\cppfl{asymmetric_op}
In the pseudo code above execution control is transferred to fiber \cpp{f1} at
line 16. Fiber \cpp{f1} resumes fiber \cpp{f2} at line 13 and so on. At line 2
fiber \cpp{f4} calls its suspend operation \cpp{self::suspend()}. Fiber \cpp{f4}
is suspended and \cpp{f3} resumed. Inside the lambda, \cpp{f3} returns from
\cpp{f4.resume()} and calls \cpp{self::suspend()} (line 6). Fiber \cpp{f3} gets
suspended while \cpp{f2} will be resumed and so on ...\\
The asymmetric version needs \bfs{N-1 more} fiber switches than the variant
using symmetric fibers.\\

\zs{While asymmetric fibers establish a caller-callee relationship (strong
coupled), symmetric fibers operate at the same level (loose coupled).}\\
\newline

Symmetric fibers represent independent units of execution, making symmetric
fibers a suitable mechanism for concurrent programming. Additionally,
constructs that produce sequences of values (\emph{generators}) are easily
constructed out of two symmetric fibers (one represents the caller, the other
the callee).\\

Asymmetric fibers incorporate additional fiber switches as shown in the pseudo
code above. It is obvious that asymmetric fibers are less efficient than their
symmetric counterparts in order to achieve the same functionality.\\
Additionally, the calling fiber must be kept alive until the called fiber
terminates. Otherwise the call of \cpp{suspend()} will be undefined behaviour
(where to transfer execution control to?).\\

\zs{A symmetric fibers are more efficient, have less restrictions (no
caller-callee relationship) and can be used to create a wider set of
applications (generators, cooperative multitasking, backtracking ...).}

%%\abschnitt{fiber as a first-class object}

Because the symmetric control-transfer operation requires explicitly passing
control between fibers, fibers must be expressed as
\emph{first-class objects}.\\

Fibers exposed as first-class objects can be passed to and returned from
functions, assigned to variables or stored into containers. With fibers as
first-class objects, a program can \bfs{explicitly control the flow of
execution} by suspending and resuming fibers, enabling control to pass into a
function at exactly the point where it previously suspended.\\

\zs{Symmetric control-transfer operations require fibers to be first-class
objects. First-class objects can be returned from functions, assigned to
variables or stored into containers.}

%%\abschnitt{encapsulating the stack}\label{stackmgmt}

Each fiber is associated with a stack and is responsible for managing the lifespan
of its stack (allocation at construction, deallocation when fiber terminates). The
RAII-pattern\footnote{resource acquisition is initialisation} should apply.\\

Copying a fiber must not be permitted!\\
If a fiber were copyable, then its stack with all the objects allocated on it
must be copied too. That presents two implementation choices.
\begin{itemize}
    \item One approach would be to capture sufficient metadata to permit
          object-by-object copying of stack contents. That would require
          dramatically more runtime information than is presently available --
          and would take considerably more overhead than a coder might expect.
          Naturally, any one move-only object on the stack would prohibit
          copying the entire stack.
    \item The other approach would be a bytewise copy of the memory occupied
          by the stack. That would force undefined behaviour if any stack
          objects were RAII-classes (managing a resource via RAII pattern). When the first
          of the fiber copies terminates (unwinds its stack), the RAII class destructors
          will release their managed resources. When the second copy terminates, the same
          destructors will try to doubly-release the same resources, leading to undefined
          behavior.
\end{itemize}

\zs{
A fiber API must:
\begin{itemize}
    \item encapsulate the stack (hiding for user)
    \item manage lifespan of an explicitly-allocated stack: the stack gets
          deallocated when fiber goes out of scope
    \item prevent accidentally copying the stack
\end{itemize}
Class \fiber must be \emph{moveable-only}.\\
}

%%\abschnitt{invalidation at resumption}\label{invalidation}

The framework must prevent the resumption of an already running or terminated
(computation has finished) fiber.

Resuming an already running fiber will cause overwriting and corrupting the stack
frames (note, the stack is not copyable).  Resuming a terminated fiber will
cause undefined behaviour because the stack might already be unwound (objects
allocated on the stack were destroyed or the memory used as stack was already
deallocated).

As a consequence each call of \resume will empty the \fiber object.

Whether or not a \fiber is empty can be tested with member function \opbool.

To make this more explicit, functions \allresume are rvalue-reference qualified.
%If a fiber calls \cpp{f.resume()} then the  fiber is suspended and \cpp{f} is
%invalidated. When the fiber is resumed later, it returns from \cpp{f.resume()}
%and object \cpp{f} references the calling fiber (the fiber that has resumed
%the current fiber).

The essential points:
\begin{itemize}
    \item regardless of the number of \fiber declarations, exactly one \fiber
          object represents each suspended fiber
    \item no \fiber object represents the currently-running fiber
\end{itemize}

Section \nameref{solution_gpub} describes how an object of type\\
\fiber is synthesized from the active fiber that suspends.

\zs{
A fiber API must:
\begin{itemize}
    \item prevent accidentally resuming a running fiber
    \item prevent accidentally resuming a terminated fiber
    \item \allresume are rvalue-reference qualified
\end{itemize}
}

%%\abschnitt{problem: avoiding non-const global variables and
undefined behaviour}\label{problem_gpub}

According to \emph{C++ core guidelines}\cite{coreguidlines}, non-const global
variables should be avoided because of hiding dependencies and making the
dependencies subject to unpredictable changes.\\
Global variables can be changed by assigning them indirectly using a pointer or
by a function call. As a consequence, the compiler can't cache the value of a
global variable in a register, degrading performance (unnecessary
loads and stores to global memory especially in performance critical loops).\\
Accessing a register is one to three orders of magnitude faster than accessing
memory (dependence to whether the cache line is in cache and not invalidated by
another core; and depending on whether the page is in the TLB).\\
The order of initialisation (and thus destruction) of static global
variables is not defined introducing additional problems for depended static
global variables.\\

\zs{A library designed to be used as building block by other higher-level
frameworks should prevent introducing global variables. If this API would be
specified in terms of internal global variables, no higher level layer can undo
that: it would be stuck with the global variables.}

\uabschnitt{switch back to \emph{main()} by return}
Switching back to \main by returning from the fiber function has two drawbacks:
it requires a internal global variable pointing to the suspended \main and
restricts the set of usages.
\cppf{return_to_main}
For instance the generator pattern is impossible because the only way
for a fiber to transfer execution control back to \main is to terminate. But
this means that no way exists to transfer data (sequence of values) back and
forth between a fiber an \main.\\

\zs{Switching to \main only by returning is impractical because it limits the
applicability of fibers and requires internal global variable pointing to
\main.}

\uabschnitt{static member function returns active fiber}
P0099R0\cite{P0099R0} has introduced a static member function\\
(\cpp{execution_context::current()}) that returned an instance of the active
fiber. This allows to pass the fiber \cpp{m} (represents \main) into the fiber
\cpp{l} via lambda capture. This mechanism enables to switch back and forth
between fiber and \main enabling a rich set of applications (for instance
generators).
\cppf{static_current}

But this solution requires a internal global variable pointing to the active
fiber and some kind of reference counting. Reference counting is needed because
\cpp{fiber::current()} allows to construct multiple instance of the active
fiber. If one instance goes out of scope the other instances must be still
valid. But the last reference must destruct the fiber and deallocate the stack.
\cppf{multi_current}

Additionally a static member function returning a instance of the active fiber
would violate the protection requirements of sections \nameref{stackmgmt} and
\nameref{invalidation}. For instance the active fiber could accidentally be
resumed by invoking \resume.
\cppf{resume_current}

\zs{A static member function returning the active fiber requires a reference
counted global variable and does not prevent for accidentally resuming the
active fiber.}

%%\abschnitt{solution: avoiding non-const global variables and
undefined behaviour}\label{solution_gpub}

\zs{The \emph{avoid con-const global variables} guideline has an important
impact on the design of the fiber API!}

\uabschnitt{synthesizing the suspended fiber}\label{synthesizing}
The problem of global variables or the need for static member functions
returning the active fiber can be avoided by \bfs{synthesizing} the
\bfs{suspended fiber} and passing it into the resumed fiber (as parameter if the
fiber is started the first time or returned from \resume by the instance that
has suspended the fiber previously).
\cppfl{synthesized_foo}

In the pseudo-code above the fiber \cpp{f} is started by invoking its member
function \resume at line 7. This operation suspends \cpp{foo}, invalidates
instance \cpp{f} and synthesizes a new fiber \cpp{m} that is passed as parameter
to the lambda of \cpp{f} (line 2).\\
Invoking \cpp{m.resume()} (line 3) suspends the lambda, invalidates \cpp{m} and
synthesizes a fiber that is returned by \cpp{f.resume()} at line 7. The
synthesized fiber is assigned to \cpp{f}. Instance \cpp{f} represents now the
suspended fiber running the lambda (that is suspended at line 3). The control
flow has been transferred from line 3 (lambda) to line 7 (\cpp{foo()}).\\
Call \cpp{f.resume()} at line 8 invalidates \cpp{f} and suspends \cpp{foo()}
again. A fiber, representing the suspended \cpp{foo()} is synthesized, returned
from \cpp{m.resume()} and assigned to \cpp{m} at line 3. The execution control
is transferred back to the lambda and instance \cpp{m} represents the suspended
\cpp{foo()}.\\
Function \cpp{foo()} is resumed at line 4 by executing \cpp{m.resume()} so that
execution control returns in line 8 and so on ...\\

Class \cpp{symmetric_coroutine<>::yield_type} from  N3985\cite{N3985} is
\bfs{not} equivalent to the synthesized fiber.\\
\cpp{symmetric_coroutine<>::yield_type} does not represent the suspended context,
instead it is a special representation of the same coroutine. Thus \main or
thread's \entryfn\xspace can \bfs{not} be represented by \cpp{yield_type}
(see next section \nameref{representation}).\\
Because \cpp{symmetric_coroutine<>::yield_type()} yields back to the starting
point, e.g. invocation of\\
\cpp{symmetric_coroutine<>::call_type::operator()()},
both instances (\cpp{call_type} as well ass \cpp{yield_type}) must be preserved
(otherwise UB occurs at resumption).\\

\zs{This API is specified in terms of passing the suspended fiber, a higher
level layer can hide that by using global variables.}

\uabschnitt{distributing \emph{main()} and thread's \entryfn\xspace as fiber}\label{representation}
As shown in the previous section a synthesized fiber is created and passed as
instance of \fiber into the resumed fiber.\\
\cppf{synthesized_main}

This is a nice feature because it allows \main and thread's \entryfn\xspace to
be represented as a fiber. A fiber representing \main or thread's
\entryfn\xspace can be handled like explicitly created fibers, it passed to
and returned from functions or stored in containers.\\
In the code snippet above the suspended \main is represented by instance
\cpp{m} and can be stored in containers, scheduled together with \cpp{f}
according to a scheduling algorithm.\\

\zs{The proposed fiber API allows to represent and handle \main and thread's
\entryfn\xspace by a instance of \fiber in the same way as explicitly created fibers.}

\uabschnitt{fiber returns (terminates)} When a fiber returns (terminates), what
should happen next? Which fiber should resumed next? The only way to avoid
internal global variables that point to \main is to explicitly return a fiber
instance that will be resumed after the active fiber terminates.
\cppfl{terminating_fiber}

In line 5 the fiber is started by invoking \resume on instance \cpp{f}. \main
is suspended and a instance of type \cpp{fiber} is synthesized and passed as
parameter \cpp{m} to the lambda at line 2. The fiber terminates by returning
\cpp{m}. The control is transferred to \main (return from \cpp{f.resume()} at
line 5 while the fiber \cpp{f} is destructed.\\

In a more advanced example another fiber is used as return value instead of the
synthesized fiber that is passed in.
\cppfl{terminating_fiber_complex}

At line 13 fiber \cpp{f2} is resumed and the lambda is entered at line 8. The
synthesized fiber \cpp{f} (representing suspended \main) is passed as a
parameter \cpp{f} and stored in \cpp{m} (captured by the lambda) at line 10.
This is necessary in order to prevent destructing \cpp{f} when the lambda
returns. Fiber \cpp{f2} uses \cpp{f1}, that was also captured by the lambda as
return value. Fiber \cpp{f2} terminates while fiber \cpp{f1} is resumed (entered
the first time). The synthesized fiber \cpp{f} passed into the lambda at line 3
represents the terminated fiber \cpp{f2} (e.g. the calling fiber). Thus instance
\cpp{f} is invalid as the assert statement checks at line 5. Fiber \cpp{f1} uses
the captured fiber \cpp{m} as return value (line 6). The control is returned to
\main returning from \cpp{f2.resume()} at line 13.\\

\zs{\fiber as return value from a function used to construct a fiber avoids
global variables -- the function passed to fiber's constructor must be of
signature `\cpp{fiber(fiber&&)}`.}

\uabschnitt{returning synthesized fiber instance from \cpp{resume()}}\label{fiberreturn}
A instance of \fiber remains invalid after returning of \resume or \resumewith -
the synthesized fiber is returned instead of implicitly assigning it.\\
If the underlying fiber object would be implicitly be replaced, the fiber would 
change its identity because each fiber is associated with a stack. Each stack
contains a chain of function calls (call stack). If this association would be
implicitly modified, unexpected behaviour happens.\\
The example below demonstrates the problem:
\cppfl{return_from_resume_inplace}

In the pseudo-code above the underlying fiber object is implicitly replaced.\\
The example creates a circle of fibers: each fiber prints out its name and
resumes the next fiber (f1 -> f2 -> f3 -> f1 -> ...).\\
Fiber \cpp{f1} is started at line 26. The synthesized fiber \cpp{main} passed 
to the resumed fiber is not used. In the for-loop the name \emph{f1} is printed
out and fiber \cpp{f2} is resumed. Inside the for-loop \cpp{f2} prints its name
and resumes \cpp{f3}. Fiber \cpp{f3} resumes fiber \cpp{f1} at line 7. Inside
\cpp{f1} control returns from \cpp{f2.resume()} loops, prints out the name and
invokes \cpp{f2.resume()}. But this time fiber \cpp{f3} instead of \cpp{f2} is
resumed. This is caused by the fact the instance \cpp{f2} gets the synthesized
fiber of \cpp{f3} implicitly assigned. Remember that at line 7 fiber \cpp{f3}
gets suspended while \cpp{f1} is resumed through \cpp{f1.resume()}.
\cppf{return_from_resume_invalid}

This problem can be solved by returning the synthesized fiber from  \resume or
\resumewith. In the example above the synthesized fiber returned by \resume is
assigned to the correct fiber instance (that has resumed the current fiber).\\

If the overall control flow isn't know, member function \resumewith (see section
\nameref{resumewith}) can be used to assign the synthesized to the correct fiber
instance.
\cppf{assign_resumewith}

Member function \cpp{resume_next()} resumes the next \cpp{filament} passed as
parameter. The active fiber invokes \resumewith on the fiber aggregated by
\cpp{fila}. The lambda captures \cpp{this}, the current fiber is suspended, a
new \fiber is synthesized and passed as parameter \cpp{f} to the function
injected to the resumed fiber of \cpp{fila}.\\
Inside the lambda \cpp{f} is moved into the instance \cpp{f_} aggregated by the
\cpp{filament} that was suspended.
\footnote{\bfiber\cite{bfiber} uses this pattern for resuming user-land threads
to assign the synthesized fiber and unlock a lock if one was provided or
re-schedule the suspended user-land thread.}

\zs{The synthesized fiber must be returned from \resume or \resumewith in order
to prevent changing the identity of the fiber.}

%%\abschnitt{inject function into suspended fiber}\label{resumewith}
Sometimes it is useful to inject a new function (for instance, to throw an
exception or assigning the synthesized fiber to the caller as described in
\nameref{fiberreturn}) into a suspended fiber. For this purpose
\cpp{resume\_with(Fn&& fn)} may be called, passing the function \cpp{fn()} to
execute.\\
If the fiber represented by the \fiber instance \cpp{f} has called a function
\cpp{suspender()}, which has called \resume and is now suspended, function
\cpp{fn()} is injected into fiber \cpp{f} as if \cpp{suspender()}'s \resume call
had directly called \cpp{fn()}.\\

Like an \entryfn\xspace passed to \fiber, \cpp{fn()} must accept
\cpp{std::fiber&&} and return \fiber. The \fiber instance returned by \cpp{fn()}
will, in turn, be returned to \cpp{suspender()} by \resume.\\

Suppose that code running on the program's main fiber calls \resume, thereby
entering the first lambda shown below. This is the point at which \cpp{m} is
synthesized and passed into the lambda at (2).\\
Suppose further that after doing some work (4), the lambda calls
\cpp{m.resume()}, thereby switching back to the main fiber. The lambda remains
suspended in the call to \cpp{m.resume()} at (8).\\
At (18) the main fiber calls \cpp{f.resume\_with()} where the passed lambda
accepts \cpp{fiber &&}. That new lambda is entered in the fiber of the suspended
lambda. It is as if the \cpp{m.resume()} call at (8) directly called the second
lambda.\\

The function passed to \resumewith has almost the same range of possibilities as
any function called on the fiber represented by \cpp{f}. Its special invocation
matters when control leaves it in either of two ways:

\begin{enumerate}
  \item If it throws an exception, that exception unwinds all previous stack
        entries in that fiber (such as the first lambda's) as well, back to
        a matching \cpp{catch} clause.\footnote{As stated
        in \nameref{exceptions}, if there is no matching \cpp{catch}
        clause in that fiber, \cpp{std::terminate()} is called.}
  \item If the function returns, the returned \fiber instance is returned by
        the suspended \cpp{m.resume()} (or \entryfn, or \resumewith) call.
\end{enumerate}

\cppfl{ontop}

The \cpp{f.resume\_with(<lambda>)} call at (18) passes control to the second
lambda on the fiber of the first lambda.\\

As usual, \resumewith synthesizes a \fiber instance representing the calling
fiber, passed into the lambda as \cpp{m}. This particular lambda returns \cpp{m}
unchanged at (21); thus that \cpp{m} instance is returned by the \resume call
and assigned at (8).\\

Finally, the first lambda returns at (10) the \cpp{m} variable updated at (8),
switching back to the main fiber.\\

\zs{Member function \resumewith allows to inject a function into suspended
fiber.}

%%\abschnitt{passing data between fibers}

Data can be transferred between two fibers via global pointer, a calling
wrapper (like \cpp{std::bind}) or lambda capture.
\cppfl{passing_lambda}

The \resume call at line 8 enters the lambda and passes 1 into the
new fiber. The value is incremented by one, as shown at line 4. The expression
\cpp{caller.resume()} at line 5 resumes the original context (represented
within the lambda by \cpp{caller}).\\
The call to \cpp{lambda.resume()} at line 10 resumes the lambda, returning from
the \cpp{caller.resume()} call at line 5. The \fiber instance \cpp{caller}
emptied by the \resume call at line 5 is replaced with the new instance
returned by that same \resume call.\\
Finally the lambda returns (the updated) \cpp{caller} at line 6, terminating its
context.\\

Since the updated \cpp{caller} represents the fiber suspended by the call at
line 10, control returns to \main.\\

However, since context \cpp{lambda} has now terminated, the updated \cpp{lambda}
is empty. Its \opbool returns \cpp{false}.\\

\zs{Using lambda capture is the preferred way to transfer data between two
fibers; global pointers or a calling wrapper (such as \cpp{std::bind}) are
alternatives.}

%%\abschnitt{termination}\label{termination}

%% There are a few different ways to terminate a given fiber without
%% terminating the whole process, or engaging undefined behaviour.
%% 
%% When a \fiber instance is constructed with an \entryfn, its new stack is
%% initialized with the frame of an implicit top-level function that marks the
%% end of the stack. \unwindfib unwinds the stack back to
%% that top-level function, which resumes the \fiber passed to \unwindfib.
%% 
%% Therefore, any of the following will gracefully terminate a fiber:
%% 
%% \begin{itemize}
%%     \item Cause its \entryfn to return a non-empty \fiber.
%%     \item From within the fiber you wish to terminate, call \unwindfib with a
%%           non-empty \fiber. That fiber will be resumed
%%           when the active fiber terminates.
%%     \item Call \cpp{fiber\_context::resume\_with(unwind\_fiber)}. This is what \dtor
%%           does. Since\\\unwindfib accepts a \fiber, and since \resumewith
%%           synthesizes a\\\fiber representing its caller and passes it to the
%%           subject function, this terminates the fiber referenced by the
%%           original \fiber instance and then resumes the caller.
%%     \item Engage \dtor: switch to some other fiber, which will
%%           receive a \fiber instance representing the current fiber. Make that
%%           other fiber destroy the received \fiber instance.
%% \end{itemize}
%% 
%% The above are all equivalent: stack variables are properly destroyed, since
%% the stack is unwound. (See \nameref{unwinding}.)
%% 
%% In an environment that forbids exceptions, 
Every \fiber you launch must
terminate gracefully by returning from its \entryfn.
%% You may not
%% call \unwindfib. You may not call \dtor, explicitly or implicitly, on a
%% non-empty \fiber instance. With these restrictions, it is possible to use
%% the \fiber facility without exception support.

When an explicitly-launched fiber's \entryfn returns a non-empty \fiber
instance, the running fiber is terminated. Control switches to the fiber
indicated by the returned \fiber instance. The \entryfn may return (switch to)
any reachable non-empty \fiber instance -- it need not be the instance originally
passed in, or an instance returned from the \resume family of methods.

\emph{Calling} \resume means: ``Please switch to the indicated fiber; I
am suspending; please resume me later.''

\emph{Returning} a particular \fiber means: ``Please switch to the indicated
fiber; and by the way, I am done.''

Cancellation of another fiber is not explicitly supported
by \cpp{fiber\_context}. If it is important for consuming code to communicate
to a suspended fiber the desire that it should terminate, lambda binding may
be used to pass some relevant object, e.g. a \cpp{stop\_token}.

It is up to the code running on the fiber in question to observe and respond
to any such termination request. The fiber must be resumed \emph{after} the
request before it could possibly observe the change. Even then, the \entryfn
might not immediately return.

One tactic would be to request termination, then loop over \anyresume calls until
the returned \fiber is \cpp{empty()}. However, that information is ambiguous.

Suppose we have a \fiber instance \cpp{f1} representing suspended fiber F,
with an application-specific termination request mechanism. The running fiber
M requests F to terminate, then calls \cpp{f1.resume()}, which in due course
returns another \fiber instance \cpp{f2}.

\cpp{f2} has various possible values.

\begin{itemize}
    \item \cpp{f2} might be empty. This might mean that fiber F did in fact
          terminate.
    \item Alternatively, it might mean that fiber F, instead of terminating,
          resumed fiber G, which terminated by resuming fiber M.
    \item Or fiber F might have terminated by resuming fiber G, which might
          have terminated by resuming fiber M.
    \item In other words, if \cpp{f2} is empty, fiber M cannot know the
          present state of fiber F.
    \item \cpp{f2} might not be empty. That might mean that fiber F did not
          terminate before resuming fiber M. \cpp{f2} would represent fiber F.
    \item Or it might mean that fiber F terminated by resuming fiber G, which
          might have resumed fiber M. \cpp{f2} would represent fiber G.
    \item Or it might mean that fiber F, instead of terminating, resumed fiber
          G, which resumed fiber M. \cpp{f2} would (again) represent fiber G.
    \item In other words, if \cpp{f2} is not empty, fiber M cannot know the
          present state of fiber F.
\end{itemize}

The \cpp{autocancel} class introduced in \nameref{launch} illustrates a
possible cancellation implementation, subject to the limitations described
above.

%% The \emph{last} fiber on a particular thread has no non-empty \fiber to
%% return. For this reason, returning an empty \fiber instance (\opbool
%% returns \cpp{false}) terminates the calling thread. This is true whether or
%% not the thread's default fiber (see \nameref{fiber-context.implicit}) has
%% terminated.

%% \uabschnitt{stack unwinding}\label{unwinding}
%% 
%% Stack unwinding caused by an exception, thread termination or fiber
%% destruction exits functions on that stack without executing a \cpp{return} statement. Local variables
%% that go out of scope may have destructors that must be called.
%% The implementation must walk the stack and call the destructor for each object
%% in every such stack frame.
%% 
%% The C++ standard does not define how exception handling is implemented. Stack unwinding differs
%% among different systems. The process of stack unwinding is described in the
%% system ABI, for instance:
%% \begin{itemize}
%%     \item \emph{.eh\_frame}/\emph{personality routine} on SYS V AMD64 ABI\cite{SYSVAMD64} (de facto standard among Unix-like operating systems)
%%     \item \emph{RUNTIME\_FUNCTION}/\emph{UNWIND\_INFO} on x64 Windows\cite{WinX64}
%%     \item \emph{.pdata}/\emph{.xdata} on ARM64 Windows\cite{WinARM64}
%% \end{itemize}
%% 
%% \paragraph{SYS V AMD64 unwind library}
%% is based on DWARF CFI (call frame information) that are stored in the \emph{.eh\_frame} section.
%% Unwinding happens under following circumstances:
%% \begin{itemize}
%%     \item A C++ exception has been thrown
%%     \item unwinding is forced by an external agent (longjmp for instance)
%% \end{itemize}
%% \uwforced takes a \foreignex (non-C++ exception; for instance Java or GO) and walks the stack frame by frame
%% inspecting the \emph{unwind tables} for cleanup functions (for instance destructors of
%% local variables) and catch blocks.
%% 
%% \uwforced calls a \emph{personality routine} (\cpp{__gxx_personality_v0()} for GCC).\footnote{The
%% personality routine passed by a specific runtime serves as interface between system unwinding library
%% and language specific exception handling (not only C++; GO and Java are also supported). It is always invoked via pointer (saved
%% as a function pointer in \ehframe\xspace for each stack frame).}
%% \uwforced takes a stop function that controls the termination of the unwinding
%% (reaching end of stack for fibers).
%% The stop function intercepts calls to the personality routine, letting the external
%% agent override the defaults of the stack frame's personality routine.\footnote{As
%% a consequence the C++ personality routine deals only with C++ exceptions;
%% it does not need to know anything specific about unwinding done by an external
%% agent such as fiber or pthreads cancellation.}
%% When the destination frame (last frame on fiber
%% stack) is reached, control jumps back to the caller without further popping
%% the stack.
%% 
%% The code snippet below is a proof of concept available at \href{https://github.com/boostorg/context/tree/p0876r6}{Boost.Context branch p0876r6}.
%% \cppf{unwind}
%% \cpp{fiber_unwind()} is called by \unwindfib or \dtor and starts the stack unwinding.
%% The foreign exception \cpp{foreign_unwind_ex}\footnote{setting member variable makes \cpp{foreign_unwind_ex} a foreign exception}
%% is allocated and passed as parameter to the unwinding library. Function \cpp{fiber_unwind_stop()} transfers execution control
%% to the calling fiber once the last stack frame has been unwound.
%% 
%% \subparagraph{non-catchable \foreignex}
%% \unwindfib uses a non-C++ \foreignex to force stack unwinding.
%% As stated in the \emph{SYS V AMD64 ABI}\cite{SYSVAMD64} standard:
%% "A runtime is not allowed to catch an exception if the \cpp{_UA_FORCE_UNWIND} flag was passed to the personality routine."
%% and "... since it is not possible to determine if a given catch clause will re-throw or not without executing it ...", the
%% \foreignex must not be catchable by C++ \cpp{try-catch} blocks.
%% 
%% As a consequence, \curex can not return a \cpp{std::exception\_ptr} pointing
%% to a \foreignex.
%% 
%% In order to detect if stack unwinding is currently in progress \uncex returns \cpp{true} and\\
%% \uncexs counts the \foreignex.
%% 
%% The rationale for moving to an uncatchable exception is further explained in
%% the \nameref{history}.
%% 
%% The specific characteristics of a \foreignex:
%% 
%% \begin{itemize}
%%     \item Throwing the \foreignex can only be effected by the \fiber
%%     facility. The proposed \unwindfib function is the only way to cause that
%%     explicitly.
%%     \item The ultimate "catch" -- the point at which stack unwinding stops --
%%     is likewise determined by the \fiber facility. There is no explicit syntax
%%     for this.
%%     \item Along the way, as with a normal C++ exception, every object in every
%%     stack frame is destroyed.
%%  \item \catchall clauses along the way are executed, but:
%%  \begin{itemize}
%%      \item \cpp{throw;} resumes stack unwinding, as usual
%%      \item a \catchall clause that does not execute a \cpp{throw;}
%%      statement behaves as if it ends with a \cpp{throw;} statement
%%      \item a \catchall clause that attempts to throw a normal C++
%%      exception engages Undefined Behaviour
%%  \end{itemize}
%%  \item \cpp{catch (}\emph{anything else}\cpp{)}
%%     \item \cpp{catch} clauses along the way are ignored.
%% \end{itemize}
%% 
%% \zs{The system's exception handling, i.e. its unwinding framework, is used to clean up the stack
%% of a fiber by using a foreign exception that is not catchable by C++ \cpp{try-catch} blocks.}
%% 
%% Since unwinding a fiber's stack requires destroying objects declared in stack
%% frames,
%% %% and may involve executing \catchall clauses,
%% it is worth pointing out that destroying a non-empty \fiber on a thread other
%% than the thread on which it was last resumed will run those object destructors
%% %% and \catchall clauses
%% on the thread destroying the \fiber instance.
%% 
%% As a consequence, destroying a \fiber instance representing a thread's default
%% fiber (see \nameref{fiber-context.implicit})
%% from any other thread engages Undefined Behaviour.\footnote{One unobvious case
%% would be if a fiber running on non-\main thread \cpp{T} stores a \fiber
%% representing \cpp{T}'s default fiber in a static variable, whether
%% module-scope or function-scope. That variable will be destroyed at program
%% termination, probably on a thread other than \cpp{T}.}
%% 
%% \subparagraph{marker frame}
%% 
%% The \fiber facility behaves as if there is an implicit top-level function above
%% each explicit fiber's \entryfn. (See \nameref{fiber-context.implicit}) This
%% top-level function serves to delimit stack unwinding. Once the stack has been
%% unwound to that point, it is as if control returns to the implicit top-level function. The
%% implicit top-level function is conceptually responsible for freeing the explicit fiber's
%% stack memory and for resuming the \fiber designated as the next fiber.
%% 
%% \subparagraph{destroying a \fiber representing a thread's default fiber}
%% 
%% Similarly, the C++ runtime behaves as if there is a stack marker at or above \main (and
%% each explicitly-launched thread's \entryfn) that serves to delimit stack
%% unwinding due to the \foreignex. Unlike an explicit fiber's top-level
%% function, though, the conceptual top-level function on a thread's default fiber
%% does \emph{not} deallocate that fiber's stack: the OS, which provided the
%% stack in the first place, will do that.
%% 
%% Unwinding the stack belonging to a thread's default fiber leaves the stack
%% allocated but unreachable. That thread may continue to execute explict fibers
%% as long as desired.
%% 
%% Ultimately, however, it must be possible to exit a fiber in such as way as to
%% terminate the calling thread. Returning an empty \fiber instance from
%% a fiber's \entryfn terminates the running thread. Consequently, passing an
%% empty \fiber instance to \unwindfib also terminates the calling thread.
%% 
%% The \fiber facility does not defend against the case in which a thread's
%% default fiber suspends (rather than terminating), but the explicit fiber it
%% resumes ultimately causes thread termination in either of the ways described
%% above. A higher-level library built on \fiber can provide a scheduler.
%% The \fiber facility intentionally does not.
%% 
%% The conceptual top-level function above \main, given an empty \fiber instance to resume,
%% terminates the whole process instead of that one thread.

%%\abschnitt{exceptions}\label{exceptions}

In general, if an uncaught exception escapes from the \entryfn,
\cpp{std::terminate} is called. There is one exception: \unwindex. The \fiber
facility internally uses \unwindex to clean up the stack of a suspended context
being destroyed. This exception must be allowed to propagate out of an \entryfn.\\

A correct \entryfn\ \cpp{try/catch} block looks like this:
\cppf{rethrow_unwind}

Of course, no \cpp{try/catch} block is needed if neither \entryfn\xspace nor
anything it calls throws exceptions.

%%%%\abschnitt{stack destruction}\label{destruction}

On construction of a fiber a stack is allocated. If the \entryfn returns, the
stack will be destroyed. If the function has not yet returned and the
\nameref{destructor} of the \fiber instance representing that context is called,
the stack will be unwound and destroyed.\\

For this purpose member-function \resumewith is called with \unwindfib as
argument. The execution context will be temporarily resumed and \unwindfib is
invoked. Function \unwindfib throws exception \unwindex.
\footnote{\unwindex binds an instance of \fiber that represents the fiber that
called \resumewith.}
The exception is caught by the first frame on the stack: the one created by
constructor. Control is switched back to the fiber that called \dtor and the
stack gets deallocated.\\

The StackAllocator's deallocate operation is called on the fiber that invoked
\dtor.\\

The stack on which \cpp{main()} is executed, as well as the stack implicitly
created by \cpp{std::thread}'s constructor, is allocated by the operating
system. Such stacks are recognized by \fiber, and are not deallocated by its
destructor.


\abschnitt{stack allocators}\label{stackalloc}

Stack allocators are used to create stacks and might implement arbitrary stack
strategies. For instance, a stack allocator might append a guard page at the end
of the stack, or cache stacks for reuse, or create stacks that grow on demand.\\

Because stack allocators are provided by the implementation, and are only used
as parameters of the constructor, the StackAllocator concept is an
implementation detail, used only by the internal mechanisms of the
implementation. Different implementations might use different StackAllocator
concepts.\\

However, when an implementation provides a stack allocator matching one of
the descriptions below, it should use the specified name.\\

Possible types of stack allocators:
\begin{itemize}
    \item \cpp{protected\_fixedsize}: The constructor accepts a \cpp{size\_t}
          parameter. This stack allocator constructs a contiguous stack of
          specified size, appending a guard page at the end to protect against
          overflow. If the guard page is accessed (read or write operation), a
          segmentation fault/access violation is generated by the operating
          system.
    \item \cpp{fixedsize}: The constructor accepts a \cpp{size\_t} parameter.
          This stack allocator constructs a contiguous stack of specified size.
          In contrast to \cpp{protected\_fixedsize}, it does not append a guard
          page. The memory is simply managed by \cpp{std::malloc()}
          and \cpp{std::free()}, avoiding kernel involvement.
    \item \cpp{segmented}: The constructor accepts a \cpp{size\_t} parameter.
          This stack allocator creates a segmented stack\cite{gccsplit} with the
          specified initial size, which \bfs{grows on demand}.\footnote{An
          implementation of the \cpp{segmented} StackAllocator necessarily
          interacts with the C++ runtime. For instance, with gcc, the
          Boost.Context\cite{bcontext} library invokes
          the \cpp{\_\_splitstack\_makecontext()}
          and \cpp{\_\_splitstack\_releasecontext()} intrinsic
          functions.\cite{splitalloc}}
\end{itemize}

It is expected that the StackAllocator's allocation operation will run in the
context of the constructor (before control is passed to the new context), and
that the StackAllocator's deallocation operation will run in the context of
the \dtor call (after control returns from the destroyed
context). No special constraints need apply to either operation.

%%\abschnitt{\fiber as building block for higher-level frameworks}

A low-level API enables a rich set of higher-level frameworks that provide
specific syntaxes/semantics suitable for specific domains. As an example, the
following four frameworks are based on the low-level fiber switching API of
\bcontext\cite{bcontext} (implements the API of this proposal).

\uabschnitt{\bcoroutine}\cite{bcoroutine2} implements \bfs{asymmetric coroutines}
\cpp{coroutine<>::push_type} and\\
\cpp{coroutine<>::pull_type}, providing a
unidirectional transfer of data. These stackful coroutines are only used in
pairs. When \cpp{coroutine<>::push_type} is explicitly
instantiated, \cpp{coroutine<>::pull_type} is synthesized and passed as
parameter into the coroutine function. In the
example below, \cpp{coroutine<>::push_type} (variable \cpp{writer}) provides the
resume operation, while \cpp{coroutine<>::pull_type} (variable \cpp{in})
represents the suspend operation. Inside the lambda,\cpp{in.get()}
pulls strings provided by \cpp{coroutine<>::push_type}'s output iterator support.
\cppf{bcoroutine_ex}

\uabschnitt{\synca}\cite{synca} (by Grigory Demchenko) is a small, efficient
library to perform asynchronous operations in synchronous manner. The main
features are a \bfs{GO-like} syntax, support for transferring execution context
explicitly between different thread pools or schedulers (portals/teleports) and
asynchronous network support.
\cppf{synca_ex}

The code itself looks like synchronous invocations while internally it uses
asynchronous scheduling.

\uabschnitt{\bfiber}\cite{bfiber} implements \bfs{user-land threads} and combines
fibers with schedulers (scheduler-algorithms are customization points). The API
is modelled after the \thread-API and contains objects like
\cpp{future}, \cpp{mutex},\\
\cpp{condition_variable} ...
\cppf{bfiber_ex}

\uabschnitt{Facebook's \fbfibers}\cite{fbfiber} is an asynchronous C++ framework
using \bfs{user-land threads} for parallelism. In contrast to \bfiber,
\fbfibers\xspace exposes the scheduler and permits integration with various
event dispatching libraries.
\cppf{fbfiber_ex}

\fbfibers\xspace is used in many critical applications at Facebook for instance
in \fbmcrouter\cite{fbmcrouter} and some other Facebook services/libraries like
ServiceRouter (routing framework for \fbthrift\cite{fbthrift}), Node API (graph
ORM API for graph databases) ...\\

\uabschnitt{Bloomber's \bbquantum}\cite{bbquantum} is a full-featured and
powerful C++ framework that allows users to dispatch units of work (a.k.a.
tasks) as coroutines and execute them concurrently using the 'reactor' pattern.
Main features are support for streaming futures which allows faster processing
of large data sets, task prioritization, fast pre-allocated memory pools and
parallel forEach and mapReduce functions.
\cppf{bbquantum}

\bbquantum\xspace is used in large projects at Bloomberg.

\uabschnitt{Habanero Extreme Scale Software Research Project\cite{habanero}}
provides a task-based parallel programming model via its \hclib\cite{hclib}.
The runtime provides work-stealing, finish-async, parallel-for and
future-promise parallel programming patterns. The library is not an exascale
programming system itself, but it manages intra-node resources and schedules
components within an exascale programming system.

\zs{As shown in this section a low-level API can act as building block for a
rich set of high-level frameworks designed for specific application domains
that require different aspects of design, semantics and syntax.}

%%\abschnitt{interaction with STL algorithms}

In the following example STL algorithm \cpp{std::generator} and fiber \cpp{g}
generate a sequence of Fibonacci numbers and store them into \cpp{std::vector}
\cpp{v}.
\cppf{generator}

\zs{The proposed fiber API does not require modifications of the STL and can be
used together with  STL algorithms.}

%%\abschnitt{possible implementation strategies}\label{implementations}

\zs{This proposal does \so{NOT} want to standardize a special implementation or
calling convention!}
\xspace\\

Modern \bfs{micro-processors} are \bfs{register machines}; the content of
processor registers represent the execution context of the program at a given
point in time.\\
\bfs{Operating systems} maintain for each process all relevant data (execution
context, other hardware registers etc.) in the process table.\\
Operating system's \bfs{CPU scheduler} periodically suspends and resumes
processes in order to share CPU time between multiple processes. When a process
is suspended, its execution context (processor registers, instruction pointer,
stack pointer, ...) is stored in the associated process table entry. On
resumption, the CPU scheduler loads the execution context into the CPU and the
process continues to be executed.\\
The CPU scheduler does a \bfs{full context switch}, beside of preserving
the execution context (complete CPU state), the cache has to be invalidated and
the memory map modified.\\
A kernel-level context switch is several orders of magnitude slower than a
context switch at user-level\cite{Tanenbaum2009}.

\uabschnitt{fiber preserves the complete CPU state} This strategy tries to
preserve the complete CPU state, e.g. all CPU registers. This requires, that the
code identifies the concrete micro-processor type and supported processor
features. For instance the x86-architecture has several flavours of extensions
like MMX, SSE1-4, AVX1-2, AVX-512.\\
Depending on the detected processor features, implementations of certain
functionality are switched on or off. The CPU scheduler in the operating system
uses those informations for the context switch  between processes.\\
A fiber implementation using this strategy needs such a detection mechanism
too (equivalent to swapper/\cpp{system_32()} in the Linux kernel).\\
Beside of the complexity of such detection mechanisms, preserving the complete
CPU state for each fiber switch is expensive.

\zs{A context switch facility that preserves the complete CPU state like a
operating system is possible but impractical for the user-land.}

\uabschnitt{fiber switch using the calling convention}\label{callingconvention}
For \fiber, not all registers are required to be preserved because the fiber
switch is effected by a visible function call. It need not be undetectable like
an operating-system context switch; it only needs to be as transparent as a call
to any other function. The calling convention -- the part of the ABI that
specifies how a function's arguments and return values are passed -- determines
which subset of micro-processor registers must be preserved by the called
subroutine.\\

The \bfs{calling convention}\cite{SYSVABI} of \bfs{SYSV ABI} for \bfs{x86\_64}
architecture determines that general purpose registers R12, R13, R14, R15, RBX
and RBP must be preserved by the sub-routine - the first arguments are passed
to functions via RDI, RSI, RDX, RCX, R8 and R9 and return values are stored in
RAX, RDX.\\
In fact, for \resume the \bfs{general purpose registers} (R12-R15, RBX and RBP)
specified by the calling convention are preserved. In addition, the \bfs{stack
pointer} and \bfs{instruction pointer} are preserved and exchanged too -- thus,
from the point of view of calling code, \resume behaves like an ordinary
function call.\\
In other words, \resume acts on the level of a simple function invocation --
with the same performance characteristics (in terms of CPU cycles).\\

This technique is used in \bcontext\cite{bcontext} which acts as building block
for \fbfibers. The \fbfibers\xspace framework itself is the basis of many
critical applications at Facebook. For instance in \fbmcrouter\cite{fbmcrouter}
and some other Facebook services/libraries like ServiceRouter (routing framework
for \fbthrift\cite{fbthrift}), Node API (graph ORM API for graph databases) ...

\uabschnitt{in-place substitution at compile time} During the code generation
the compiler could inject the assembler code responsible for the fiber switch
directly in the function that calls \resume. This saves an extra indirection
(JMP + PUSH/MOV of certain registers used to invoke \resume).

\uabschnitt{CPU state at the stack} Because each fiber has to preserve CPU
registers at suspension and to load those registers at resumption, some storage
is required.\\
Instead of allocating extra memory for each fiber, a implementation could use
the stack by simply advancing the stack pointer at suspension and pushing the
CPU registers (CPU state) onto the stack (owned by the suspending fiber). When
the fiber is resumed, the fiber pops the values from its stack and loads them
into the appropriate registers.\\

This strategy works because only a resumed fiber creates new stack frames
(advancing the stack pointer). If a fiber is suspended it is save to keep the
CPU state on its stack.\\

Using the stack as storage for the CPU state has the advantage that \fiber needs
only to aggregate a pointer to the stack location (memory footprint is equal to
a pointer).\\
Section \nameref{synthesizing} describes how global variables are prevented
by synthesizing a fiber from the active fiber (execution context) and passing
this synthesized fiber (representing the now suspended fiber) into the resumed
fiber. Using the stack as storage makes this mechanism very easy to implement
\footnote{The implementation of \bcontext\cite{bcontext} utilizes this
technique.}.
Inside \resume the code pushes the relevant CPU registers onto the stack and
creates from the resulting stack address a new \fiber. This instance is then
passed into the resume fiber as parameter if the resumed fiber is started for the
first time or returned by \resume/\resumewith otherwise (see example in
\nameref{synthesizing}).\\

\zs{Using fiber's stack as storage for the CPU state is efficient because no
additional allocations and deallocations are required.}

%%\abschnitt{fiber switch on architectures with register window}

The implementation of fiber switch is possible -- many libc implementations
still provide the ucontext-API (\cpp{swapcontext()} and related
functions)\footnote{ucontext was removed from POSIX standard by POSIX.1-2008}
for architectures using a register window (such as SPARC). The implementation of
\cpp{swapcontext()} could be used as blueprint for a fiber implementation.

%%\abschnitt{how fast is a fiber switch}

A fiber switch takes 19 CPU cycles on a \emph{x86\_64-Linux}
system\footnote{Intel XEON E5 2620v4 2.2GHz} using an implementation based on
the strategy described in \nameref{callingconvention} (implemented in
\bcontext\cite{bcontext}, branch \emph{fiber}).

%%\abschnitt{interaction with accelerators}

For many core devices several programming models, such as OpenACC, CUDA, OpenCL
etc., have been developed targeting \bfs{host-directed} execution using an
attached or integrated accelerator. The CPU executes the main program while
controlling the activity of the accelerator. Accelerator devices typically
provide capabilities for efficient vector processing\footnote{warp on CUDA
devices, wavefront on AMD GPUs, 512-bit SIMD on Intel Xeon Phi}. Usually the
host-directed execution uses \bfs{computation offloading} that permits
executing computationally intensive work on a separate device
(accelerator)\cite{OpenAcc}.\\
For instance CUDA devices use a \bfs{command buffer} to establish communication
between host and device. The host puts commands (op-codes) into the command
buffer and the device process them \bfs{asynchronously}\cite{CUDA}.\\
It is obvious that a fiber switch does \bfs{not} interact with
\bfs{host-directed device-offloading}. A fiber switch works like a function
call (see \nameref{callingconvention}).

%%\abschnitt{multi-threading environment}

Member function \canxtresume returns \cpp{false} if the stack represented by the
\fiber instance was created by the operating system (main application or thread
stack), and the caller is running on a different thread. When the stack
represented by the \fiber instance was created by \fiber's
constructor, \canxtresume returns \cpp{true}.\footnote{A possible
implementation could mark the first stack frame by creating a special marker
(for instance \emph{0xBADCAFFEE} etc.) at a specific offset in the first stack
frame or use a special function name for the first function and walk the stack
searching for these markers.} You must not attempt to resume an instance
representing a stack provided by the operating system on some other thread!

\canresume can be called to determine whether a given \fiber instance might
safely be resumed on a particular thread by calling \resume or \resumewith.

\fiber is TLS-agnostic - best practices related to TLS apply to fibers too
(see P0772R0.)

There could potentially be Undefined Behavior if:
\begin{itemize}
    \item code running on a fiber references \cpp{thread\_local} variables
    \item the compiler/runtime implementation caches a pointer
          to \cpp{thread\_local} storage on the stack
    \item that fiber is suspended, and
    \item the suspended fiber is resumed on a different thread.
\end{itemize}

The cached TLS pointer is now pointing to storage belonging to some other
thread. If the original thread terminates before the new thread, the cached
TLS pointer is now dangling.

For a runtime that caches TLS pointers in such fashion, an implementation
of \xtresume or\\
\xtresumewith could conceivably walk the suspended stack,
patching cached pointers.

%%\abschnitt{acknowledgment}

I'd like to thank Grigory Demchenko.

%%\uabschnitt{class fiber\_context}

\cppf{fiber}

\paragraph*{member functions}

\subparagraph*{(constructor)}
constructs new fiber for use solely within the current \thread\\

\begin{tabular}{ l l }
    \midrule

    \cpp{fiber\_context() noexcept} & (1)\\

    \midrule

    \cpp{template<typename Fn>}\\
    \cpp{explicit fiber\_context(Fn&& fn)} & (2)\\

    \midrule

    \cpp{template<typename StackAlloc, typename Fn>}\\
    \cpp{fiber\_context(std::allocator\_arg\_t, StackAlloc&& salloc, Fn&& fn)} & (3)\\

    \midrule

    \cpp{fiber\_context(fiber\_context&& other) noexcept} & (4)\\

    \midrule

    \cpp{fiber\_context(const fiber\_context& other)=delete} & (5)\\

    \midrule
\end{tabular}

\begin{description}
    \item[1)] this constructor instantiates an invalid \fiber. Its \opbool
              returns \cpp{false}; its \cpp{operator\!()} returns \cpp{true}.
    \item[2)] takes a callable (function, lambda, object with \op) as
              argument. The callable must have signature as described
              in \nameref{solution_gpub}. The stack is constructed using
              either \cpp{fixedsize} or \cpp{segmented} (see \nameref{stackalloc}).
              An implementation may infer which of these best suits the code
              in \cpp{fn}. If it cannot infer, \cpp{fixedsize} will be used.
    \item[3)] takes a callable as argument, requirements as for (2). The stack
              is constructed using \emph{salloc}
              (see \nameref{stackalloc}).\footnote{This constructor,
              along with the \nameref{stackalloc} section, is an
              optional part of the proposal. It might be that implementations
              can reliably infer the optimal stack representation.}
    \item[4)] moves underlying state to new \fiber
    \item[5)] copy constructor deleted
\end{description}

{\bfseries Notes}
\begin{description}
    \item The entry-function \cpp{fn} is \emph{not} immediately entered. The
          stack and any other necessary resources are created on construction,
          but \cpp{fn} is not entered until \resume or \resumewith is called.
    \item The entry-function \cpp{fn} passed to \fiber will be passed a synthesized \fiber
          instance representing the suspended caller of \resume.
    \item The function \cpp{fn} passed to \resumewith will be passed a
          synthesized \fiber instance representing the suspended caller of \resumewith.
\end{description}

\subparagraph*{(destructor)}\label{destructor}
destroys a fiber\\

\begin{tabular}{ l l }
    \midrule

    \dtor & (1)\\

    \midrule
\end{tabular}

\begin{description}
    \item[1)] destroys a \fiber instance. If this instance represents a fiber
              of execution (\opbool returns \cpp{true}), then the fiber of
              execution is destroyed too. Specifically, the stack is unwound
              by throwing \unwindex.\footnote{ In a program in which exceptions
              are thrown, it is prudent to code a fiber's \entryfn\xspace with a
              last-ditch \cpp{catch (...)} clause: in general, exceptions must
              \emph{not} leak out of the \entryfn. However, since stack
              unwinding is implemented by throwing an exception, a correct
              \entryfn\ \cpp{try} statement must also
              \cpp{catch (std::unwind\_exception const&)} and rethrow it.}
\end{description}


\subparagraph*{operator=}
moves the \fiber object\\

\begin{tabular}{ l l }
    \midrule

    \cpp{fiber\_context& operator=(fiber\_context&& other) noexcept} & (1)\\

    \midrule

    \cpp{fiber\_context& operator=(const fiber\_context& other)=delete} & (2)\\

    \midrule
\end{tabular}

\begin{description}
    \item[1)] assigns the state of \cpp{other} to \cpp{*this} using move semantics
    \item[2)] copy assignment operator deleted
\end{description}

{\bfseries Parameters}
\begin{description}
    \item[other]   another \fiber to assign to this object\\
\end{description}

{\bfseries Return value}
\begin{description}
    \item[*this]
\end{description}


\subparagraph*{resume()}
resumes a fiber\\

\begin{tabular}{ l l }
    \midrule

    \cpp{fiber\_context resume() &&} & (1)\\

    \midrule

    \cpp{template<typename Fn>}\\
    \cpp{fiber\_context resume\_with(Fn&& fn) &&} & (2)\\

    \midrule
\end{tabular}

\begin{description}
    \item[1)] suspends the active fiber, resumes fiber \cpp{*this}
    \item[2)] suspends the active fiber, resumes fiber \cpp{*this}
              but calls \cpp{fn()} in the resumed fiber (as if called by the
              suspended function)
\end{description}

{\bfseries Parameters}
\begin{description}
    \item[fn] function injected into resumed fiber\\
\end{description}

{\bfseries Return value}
\begin{description}
    \item[fiber\_context] the returned instance represents the fiber that has been
                 suspended in order to resume the current fiber
\end{description}

{\bfseries Exceptions}
\begin{description}
    \item[1)] \resume or \resumewith might throw \unwindex if, while suspended,
              the calling \fiber is destroyed
    \item[2)] \resume or \resumewith might throw \emph{any} exception if,
              while suspended:
              \begin{itemize}
                  \item some other fiber calls \resumewith to resume this
                        suspended fiber
                  \item the function \cpp{fn} passed to \resumewith -- or some
                        function called by \cpp{fn} -- throws an exception
              \end{itemize}
    \item[3)] Any exception thrown by the function \cpp{fn} passed to
              \resumewith, or any function called by \cpp{fn}, is thrown in the
              fiber referenced by \cpp{*this} rather than in the fiber of
              the caller of \resumewith.
\end{description}

{\bfseries Preconditions}
\begin{description}
    \item[1)] \cpp{*this} represents a valid fiber (\opbool returns \cpp{true})
    \item[2)] the current \thread is the same as the thread on which
              \cpp{*this} was originally launched. An implementation is not
              required to verify.
\end{description}

{\bfseries Postcondition}
\begin{description}
    \item[1)] \cpp{*this} is invalidated (\opbool returns \cpp{false})
\end{description}

{\bfseries Notes}
\newline
\resume preserves the execution context of the calling fiber. Those data are
restored if the calling fiber is resumed.\\
A suspended \cpp{fiber\_context} can be destroyed. Its resources will be cleaned
up at that time.\\
The returned \cpp{fiber\_context} indicates via \opbool whether the previous active
fiber has terminated (returned from \entryfn).\\
Because \resume invalidates the instance on which it is called, \emph{no valid
\fiber instance ever represents the currently-running fiber.} In order to
express the invalidation explicitly, \resume and \resumewith are
rvalue-reference qualified. This means that both functions can only be invoked on
rvalues.\\
When calling \resume, it is conventional to replace the newly-invalidated
instance -- the instance on which \resume was called -- with the new instance
returned by that \resume call. This helps to avoid inadvertent calls to \resume
on the old, invalidated instance.
\newline
An injected function \cpp{fn()} must accept \cpp{std::fiber\_context&&} and
return \fiber. The \fiber instance returned by \cpp{fn()} is, in turn, used as
the return value for the suspended function: \resume or \resumewith.

\subparagraph*{uses\_system\_stack()}
query whether this \fiber instance uses a system provided stack. A \fiber
instance using a system provided stack can not be resumed on a different thread
than the one on which it was originally launched.

\begin{tabular}{ l l }
    \midrule

    \cpp{bool uses\_system\_stack()} & (1)\\

    \midrule
\end{tabular}

\begin{description}
    \item[1)] \cpp{fiber\_context::uses\_system\_stack()} returns \cpp{false}
        if the stack used by the fiber was not provided by the operating system;
        otherwise \cpp{true}.
\end{description}

{\bfseries Notes}
\newline
\fiber is optimized for use only within a single \thread. Resuming
it on a thread other than the one on which it was launched results in
Undefined Behavior.

\subparagraph*{previous\_thread()}
returns the \cpp{std::thread::id} of the thread at which the \fiber instance was
suspended. A \fiber instance that was not resumed yet, returns a
default constructed \cpp{std::thread::id}.

\begin{tabular}{ l l }
    \midrule

    \cpp{std::thread::id previous\_thread()} & (1)\\

    \midrule
\end{tabular}

\begin{description}
    \item[1)] returns \cpp{std::thread::id} of the thread at which \cpp{*this}
        was suspended. If the \fiber was not started yet, an default constructed
        \cpp{std::thread::id} will be returned.
\end{description}

\subparagraph*{operator bool}
test whether \fiber is valid\\

\begin{tabular}{ l l }
    \midrule

    \cpp{explicit operator bool() const noexcept} & (1)\\

    \midrule
\end{tabular}

\begin{description}
    \item[1)] returns \cpp{true} if \cpp{*this} represents a fiber of
              execution, \cpp{false} otherwise.
\end{description}

{\bfseries Notes}
\newline
A \fiber instance might not represent a valid fiber for any of a number of reasons.
\begin{itemize}
    \item It might have been default-constructed.
    \item It might have been assigned to another instance, or passed into a
          function.\\
          \fiber instances are move-only.
    \item It might already have been resumed -- calling \resume invalidates the
          instance.
    \item The \entryfn\xspace might have voluntarily terminated the fiber by
          returning.
\end{itemize}
The essential points:
\begin{itemize}
    \item Regardless of the number of \fiber declarations, exactly one\\
          \fiber instance represents each suspended fiber.
    \item No \fiber instance represents the currently-running fiber.
\end{itemize}


\subparagraph*{operator!}
test whether \fiber is invalid\\

\begin{tabular}{ l l }
    \midrule

    \cpp{bool operator\!() const noexcept} & (1)\\

    \midrule
\end{tabular}

\begin{description}
    \item[1)] returns \cpp{false} if \cpp{*this} represents a valid fiber,
              \cpp{true} otherwise.
\end{description}

{\bfseries Notes}
\newline
See {\bfseries Notes} for \opbool.

\subparagraph*{(comparisons)}
establish an arbitrary total ordering for \fiber instances\\

\begin{tabular}{ l l }
    \midrule

    \cpp{bool operator<(const fiber\_context& other) const noexcept} & (1)\\

    \midrule
\end{tabular}

\begin{description}
    \item[1)] This comparison establishes an arbitrary total ordering of \fiber
              instances, for example to store in ordered containers. (However,
              key lookup is meaningless, since you cannot construct a search key
              that would compare equal to any entry.) There is no significance
              to the relative order of two instances.
\end{description}


\subparagraph*{swap}
swaps two \fiber instances\\

\begin{tabular}{ l l }
    \midrule

    \cpp{void swap(fiber\_context& other) noexcept} & (1)\\

    \midrule
\end{tabular}

\begin{description}
    \item[1)] Exchanges the state of \cpp{*this} with \cpp{other}.\\
\end{description}

%%\newpage
\abschnitt{Appendix A: support code for examples}\label{launch}\label{appendixa}

Destroying a non-empty \fiber instance invokes Undefined Behaviour -- you must
first call \reqstop (see \nameref{termination}). To simplify code examples in
this paper, we introduce an \cpp{autocancel} wrapper class that tracks the
sequence of \fiber instances representing a particular fiber. When
an \cpp{autocancel} instance is destroyed, it calls \reqstop on the
stored \fiber and loops until the fiber voluntarily terminates.

\cppf{autocancel}


\newpage
\addcontentsline{toc}{subsection}{references}
\begin{thebibliography}{99}

    \bibitem{P0876R17}
        \href{https://www.open-std.org/jtc1/sc22/wg21/docs/papers/2024/p0876r17.pdf}
        {P0876R17: fibers without scheduler}

    \bibitem{Standard}
        \href{https://www.open-std.org/jtc1/sc22/wg21/docs/papers/2024/n4981.pdf}
        {N4981: Working Draft, Programming Languages -- C++}

\end{thebibliography}

%//////////////////////////////////////////////////////////////////////////////

\end{document}
