%//////////////////////////////////////////////////////////////////////////////

\documentclass[fontsize=10pt,paper=A4,pagesize,DIV=15]{scrartcl}

\usepackage[T1]{fontenc}
\usepackage[utf8]{inputenc}
\usepackage[american]{babel}        % required for ISO dates
\usepackage[iso,american]{isodate}  % ISO format of dates
\usepackage[final]{listings}        % code listings
\usepackage{booktabs}               % fancy tables
\usepackage[color]{changebar}       % changebars for large inserted passages
\usepackage{longtable}              % auto breaking tables
\usepackage{ltcaption}              % fix captions for long tables
\usepackage{relsize}                % provide relative font size changes
%\usepackage{underscore}             % remove special status of '_' in ordinary text
%\usepackage{verbatim}               % improved verbatim environment
\usepackage{parskip}                % handle non-indented paragraphs "properly"
\usepackage{array}                  % new column definitions for tables
\usepackage[normalem]{ulem}         % underline commands
\usepackage{xcolor}                 % driver-independent color extensions
\usepackage{amsmath}                % mathematical symbols
\usepackage{mathrsfs}               % mathscr font
\usepackage{xspace}                 % inserts a space to replace one "eaten" by TeX
\usepackage[final]{microtype}       % micro-typographic extensions introduced by pdfTeX
\usepackage{xstring}                % manipulating strings
\usepackage{fixme}                  % collaborative annotations
\usepackage{multicol}               % intermix single and multiple columns
\usepackage{perpage}                % counter reset at every page boundary
\usepackage{palatino}               % Adobe Palatino font
\usepackage{overcite}               % citations
\usepackage{boxedminipage}          % framed mini-pages
\usepackage{fancyhdr}               % control of page headers and footers
\usepackage{soul}                   % hyphenatable spacingout), underlining, striking out, et.
\usepackage{svg}                    % SVG pictures
\usepackage{tikz}                   % creating PS and PDF graphics
\usetikzlibrary{arrows,automata}

\cbcolor{green}

\usepackage[pdftex,
            pdftitle    = {thread\_local means fiber-specific},
            pdfsubject  = {},
            pdfauthor   = {Nat Goodspeed},
            pdfkeywords = {C++,fiber_context,thread_local},
            bookmarks=true,
            bookmarksnumbered=true,
            pdfpagelabels=true,
            pdfpagemode=UseOutlines,
            pdfstartview=FitH,
            linktocpage=true,
            colorlinks=true,
            linkcolor=blue,
            plainpages=false
           ]{hyperref}

%//////////////////////////////////////////////////////////////////////////////

\setlength{\parindent}{0pt} 
\renewcommand\sfdefault{phv}

\makeatletter
    \renewcommand*\l@subsection{\@dottedtocline{2}{0em}{2.3em}}
    \renewcommand*\l@subsection{\@dottedtocline{3}{0em}{3.2em}}
    \renewcommand{\tableofcontents}{\@starttoc{toc}}
\makeatother

\MakePerPage{footnote}
\renewcommand*{\thefootnote}{\fnsymbol{footnote}}

\newcommand{\pdfimg}[1]{\pdfximage{pics/#1}\pdfrefximage\pdflastximage}
\newcommand{\img}[1]{\mbox{\pdfimg{#1}}}
\newcommand{\imgc}[1]{\begin{center}\img{#1}\end{center}}
\newcommand{\graph}[1]{\input{graphs/#1}}
\newcommand{\graphc}[1]{\begin{center}\graph{#1}\end{center}}
\newcommand{\bfs}[1]{{\bfseries #1}}
\newcommand{\zs}[1]{\begin{boxedminipage}[t]{16.8cm}\bfs{#1}\end{boxedminipage}}

\newcommand{\cpp}[1]{{\lstinline[
		basicstyle=\ttfamily\small\color{black},
        breakatwhitespace=true,
        breaklines=true,
        captionpos=b,
        commentstyle=\ttfamily\color{red},
        keywordstyle=\ttfamily\color{blue},
        language={C++},
        morekeywords={},
        showspaces=false,
        showstringspaces=false,
        showtabs=false,
        stringstyle=\ttfamily\color{magenta}
] !#1!}\xspace}
\newcommand{\cppf}[1]{\lstinputlisting[
		basicstyle=\ttfamily\small\color{black},
        breakatwhitespace=true,
        breaklines=true,
        captionpos=b,
        commentstyle=\ttfamily\color{red},
        keywordstyle=\ttfamily\color{blue},
        language={C++},
        morekeywords={},
        showspaces=false,
        showstringspaces=false,
        showtabs=false,
        stringstyle=\ttfamily\color{magenta}
] {code/#1.cpp}}
\newcommand{\cppfl}[1]{\lstinputlisting[
		basicstyle=\ttfamily\small\color{black},
        breakatwhitespace=true,
        breaklines=true,
        captionpos=b,
        commentstyle=\ttfamily\color{red},
        keywordstyle=\ttfamily\color{blue},
        language={C++},
        morekeywords={},
        numbers=left,
        showspaces=false,
        showstringspaces=false,
        showtabs=false,
        stringstyle=\ttfamily\color{magenta}
] {code/#1.cpp}}

\newcommand{\dtor}{\cpp{\~fiber\_context()}}
\newcommand{\main}{\cpp{main()}}
\newcommand{\fiber}{\cpp{std::fiber\_context}}
\newcommand{\op}{\cpp{operator()()}}
\newcommand{\opbool}{\cpp{operator bool()}}
\newcommand{\resume}{\cpp{resume()}}
\newcommand{\resumewith}{\cpp{resume\_with()}}
\newcommand{\xtresume}{\cpp{resume\_other\_thread()}}
\newcommand{\xtresumewith}{\cpp{resume\_other\_thread\_with()}}
\newcommand{\usessysstack}{\cpp{uses\_system\_stack()}}
\newcommand{\prevtid}{\cpp{previous\_thread()}}
\newcommand{\thread}{\cpp{std::thread}}
\newcommand{\unwindex}{\cpp{std::unwind\_exception}}
\newcommand{\unwindfib}{\cpp{std::unwind\_fiber()}}

\newcommand{\sym}{\emph{symmetric}\xspace}
\newcommand{\asym}{\emph{asymmetric}\xspace}
\newcommand{\entryfn}{\emph{entry-function}}

\newcommand{\abschnitt}[1]{\addcontentsline{toc}{subsection}{#1}\subsection*{#1}}
\newcommand{\uabschnitt}[1]{\paragraph*{#1}}

\newcommand{\bcontext}{
        \href{http://www.boost.org/doc/libs/release/libs/context/doc/html/index.html}
        {\emph{Boost.Context}}}
\newcommand{\bcoroutine}{
        \href{http://www.boost.org/doc/libs/release/libs/coroutine2/doc/html/index.html}
        {\emph{Boost.Coroutine2}}}
\newcommand{\bfiber}{
        \href{http://www.boost.org/doc/libs/release/libs/fiber/doc/html/index.html}
        {\emph{Boost.Fiber}}}
\newcommand{\fbmcrouter}{
        \href{https://code.facebook.com/posts/296442737213493/introducing-mcrouter-a-memcached-protocol-router-for-scaling-memcached-deployments}
        {\emph{mcrouter}}}
\newcommand{\fbfibers}{
        \href{https://github.com/facebook/folly/tree/master/folly/fibers}
        {\emph{folly::fibers}}}
\newcommand{\fbthrift}{
        \href{https://github.com/facebook/fbthrift}
        {\emph{Thrift}}}
\newcommand{\synca}{
        \href{https://github.com/gridem/Synca}
        {\emph{Synca}}}


%//////////////////////////////////////////////////////////////////////////////

\begin{document}
\small
\begin{tabbing}
    Document number: \= P3346R0\\
    Date:            \> 2024-10-16\\
    Author:          \> Nat Goodspeed (nat@lindenlab.com)\\
    Audience:        \> LEWG, EWG\\
\end{tabbing}

\section*{thread\_local means fiber-specific}

%//////////////////////////////////////////////////////////////////////////////

\tableofcontents

%//////////////////////////////////////////////////////////////////////////////

\abschnitt{Abstract}\label{abstract}

P0876R18\cite{P0876R18} does not specify any changes to the semantics of
\stdterm{thread storage duration}{basic.stc.thread}. The implication is that
if two fibers are running on the same thread, they will both share the same
value of a given \tlocal variable. Each fiber will see modifications made by
the other fiber. They will not \stdterm{race}{intro.races}, since at every
moment a specific thread is running exactly one fiber.

Nonetheless, this could be problematic for an existing library that relies on
\tlocal variables if multiple fibers on the same thread take turns making
calls into that library. From the library's point of view, the value of its
\tlocal variables might change unexpectedly.

It is suggested that every \tlocal variable should have a distinct value for
each fiber that accesses it. This paper details changes to the
Standard\cite{Standard} to express that functionality. 

When the Standard first introduced \tlocal, library authors were strongly
encouraged to migrate from \cpp{static} variables to \tlocal. The C++
community has already paid that price. Now, instead of introducing another new
keyword requiring another new migration, we should leverage the existing
keyword to retain the desired behavior of a variable whose value persists
between successive calls, independently of calls into the same library by
other execution agents.

To address the less common use case of a variable local to a \thread but
shared by all fibers within that thread, we also introduce \tptr, a design
pioneered in the Boost library suite.\cite{Boost}

This paper depends on P0876R18.

\abschnitt{Revision History}\label{history}
Initial revision.

%%This document supersedes P3346R0.
%%
%%\uabschnitt{Changes since P3346R0}
%%
%%\begin{itemize}
%%    \item Update to reference N4981.
%%\end{itemize}

%%\newpage
\abschnitt{Acknowledgments}

The author would like to thank ADAM Martin.

\abschnitt{Wording}\label{api}

This wording is relative to N4981\cite{Standard}.

\zs{Modify \stdsection{6.7.5.3}{basic.stc.thread} as follows:}

1 All variables declared with the \cpp{thread\_local} keyword have
\emph{thread storage duration}.
The storage for these entities lasts for the duration of
the \replace{thread}{fiber} in which they are created. There is a distinct object or reference
per \replace{thread}{fiber}, and use of the declared name refers to the entity associated with
the current \replace{thread}{fiber}.

2 \tsnoten{1}{A variable with thread storage duration is initialized as specified
in \stdclause{basic.start.static}, \stdclause{basic.start.dynamic}, and \stdclause{stmt.dcl}
and, if constructed, is destroyed on \replace{thread}{fiber} exit \replace{\xref{basic.start.term}}{\xref{fiber.context.overview}}.}

\zs{Modify \stdsection{6.9.3.4}{basic.start.term} paragraph 2 as follows:}

2 Constructed objects with thread storage duration within a given \replace{thread}{fiber}
are destroyed as a result of returning from the initial function of that \replace{thread}{fiber} and as a
result of that \replace{thread}{fiber} calling \cpp{std::exit}.
The destruction of all constructed objects with thread storage
duration within that \replace{thread}{fiber} strongly happens before destroying
any object with static storage duration.

\zs{Modify \stdsection{11.4.9.3}{class.static.data} paragraph 1 as follows:}

1 A static data member is not part of the subobjects of a class. If a
static data member is declared \cpp{thread\_local} there is one copy of
the member per \replace{thread}{fiber}. If a static data member is not declared
\cpp{thread\_local} there is one copy of the data member that is shared by all
the objects of the class.

\zs{Modify \stdsection{17.5}{support.start.term} paragraph 9.1 as follows:}

(9.1) --- First, objects with thread storage duration and associated with the current
\replace{thread}{thread's default fiber}
are destroyed. Next, objects with static storage duration are destroyed
and functions registered by calling
\cpp{atexit}
are called.\textsuperscript{191}

See \stdclause{basic.start.term} for the order of destructions and calls.
(Objects with automatic storage duration are not destroyed as a result of calling
\cpp{exit()}.)\textsuperscript{192}

If a registered function invoked by \cpp{exit} exits via an exception,
the function \cpp{std::terminate} is invoked\xref{except.terminate}.

\zs{Modify \stdsection{33.10.10.2}{futures.task.members} paragraph 23 as follows:}

23 \effects
As if by \emph{INVOKE}\cpp{<R>(f, t}$_1$\cpp{, t}$_2$, $\dotsc$\cpp{, t}$_N$\cpp{)}\xref{func.require},
where \cpp{f} is the stored task and
\cpp{t}$_1$\cpp{, t}$_2$, $\dotsc$\cpp{, t}$_N$ are the values in \cpp{args...}. If the task returns normally,
the return value is stored as the asynchronous result in the shared state of
\this, otherwise the exception thrown by the task is stored. In either
case, this is done without making that state ready\xref{futures.state} immediately. Schedules
the shared state to be made ready when the current thread exits,
after all objects of thread storage duration associated with the current \replace{thread}
{thread's default fiber} have been destroyed.

\zs{Insert new final subclause in clause 33 \stdclause{thread} as indicated:}

\setcounter{section}{33}
\setcounter{subsection}{12}
\setcounter{secnumdepth}{4}

\cbstart

\rSec2[thread.ptr]{thread\_specific\_ptr}

\rSec3[thread.ptr.overview]{Overview}

1 Objects with thread storage duration now have a distinct instance for each
fiber within a thread, and are destroyed when the fiber terminates. It is
sometimes desirable to access storage that is shared between all fibers on a
thread, but distinct for each referencing thread.

2 The \tptr class manages pointers, one per referencing thread. This can be
used to access a distinct object with dynamic storage duration for each
thread, that is nonetheless shared between all fibers on that thread.

%--------------------------------- synopsis ----------------------------------
\rSec3[thread.ptr.synopsis]{Header <thread\_specific\_ptr> synopsis}

\cppf{synopsis.tls}

%--------------------------------- class def ---------------------------------

\rSec3[thread.ptr.class]{Class thread\_specific\_ptr}

\cppf{tsp}

\rSec4[thread.ptr.cons]{Constructors, move and assignment}

%------------------------------- default ctor --------------------------------
\mbrhdr{thread\_specific\_ptr()}

1 \constraints
\cpp{is\_pointer\_v<Deleter> == false} and \cpp{is\_default\_constructible\_v<Deleter> == true}

2 \mandates
\cpp{is\_nothrow\_invocable\_v<Deleter, T*> == true}

3 \precond
\cpp{Deleter} meets the \emph{Cpp17DefaultConstructible} requirements and such
construction does not exit via an exception.

4 \effects
\begin{description}
    \item[---] Constructs a \cpp{thread\_specific\_ptr} object for storing a
               pointer to an object of type \cpp{T} specific to each thread.
               Value-initializes \cpp{del}.
\end{description}

5 \postcond
\cpp{bool(*this) == false}

6 \except
\begin{description}
    \item[---] \cpp{system\_error} if an error occurs.
\end{description}

7 \errors
\cpp{resource\_unavailable\_try\_again} -- the system lacked the necessary
resources to instantiate a \tptr.

%----------------------------- copy deleter ctor -----------------------------
\mbrhdr{explicit thread\_specific\_ptr(const Deleter\& deleter)}

1 \constraints
\cpp{is\_constructible\_v<Deleter, decltype(deleter)> == true}

2 \mandates
\cpp{is\_nothrow\_invocable\_v<Deleter, T*> == true}

3 \precond
\cpp{Deleter} meets the \emph{Cpp17CopyConstructible} requirements and such
construction does not exit via an exception.

4 \effects
\begin{description}
    \item[---] Constructs a \cpp{thread\_specific\_ptr} object for storing a
               pointer to an object of type \cpp{T} specific to each thread.
               Initializes \cpp{del} from
               \cpp{std::forward<decltype(deleter)>(deleter)}.
\end{description}

5 \postcond
\cpp{bool(*this) == false}

6 \except
\begin{description}
    \item[---] \cpp{system\_error} if an error occurs.
\end{description}

7 \errors
\cpp{resource\_unavailable\_try\_again} -- the system lacked the necessary
resources to instantiate a \tptr.

%----------------------------- move deleter ctor -----------------------------
\mbrhdr{explicit thread\_specific\_ptr(Deleter\&\& deleter)}

1 \constraints
\cpp{is\_move\_constructible\_v<Deleter> == true}

2 \mandates
\cpp{is\_nothrow\_invocable\_v<Deleter, T*> == true}

3 \precond
\cpp{Deleter} meets the \emph{Cpp17MoveConstructible} requirements and such
construction does not exit via an exception.

4 \effects
\begin{description}
    \item[---] Constructs a \cpp{thread\_specific\_ptr} object for storing a
               pointer to an object of type \cpp{T} specific to each thread.
               Initializes \cpp{del} from
               \cpp{std::forward<decltype(deleter)>(deleter)}.
\end{description}

5 \postcond
\cpp{bool(*this) == false}

6 \except
\begin{description}
    \item[---] \cpp{system\_error} if an error occurs.
\end{description}

7 \errors
\cpp{resource\_unavailable\_try\_again} -- the system lacked the necessary
resources to instantiate a \tptr.

%----------------------------------- dtor ------------------------------------
\mbrhdr{\cpp{\~thread\_specific\_ptr()}}

1 \precond
\begin{description}
    \item[---] All thread specific instances associated with this \tptr
               (except the one associated with the calling thread) must be
               \cpp{nullptr}.
\end{description}

2 \effects
\begin{description}
    \item[---] Calls \cpp{this->reset()} to clean up the associated value for the
               calling thread, and destroys \this.
\end{description}

\tsnote{The Precondition avoids the necessity for an implementation to track
all thread specific instances associated with each \tptr.}

\rSec4[thread.ptr.mem]{Members}

%----------------------------------- get() -----------------------------------
\mbrhdr{T* get() const noexcept}

1 \returns
\begin{description}
    \item[---] The pointer associated with the calling thread.
\end{description}

%------------------------------- operator->() --------------------------------
\mbrhdr{T* operator->() const noexcept}

1 \returns
\begin{description}
    \item[---] \cpp{this->get()}
\end{description}

%-------------------------------- operator*() --------------------------------
\mbrhdr{T\& operator*() const noexcept}

1 \precond
\begin{description}
    \item[---] \cpp{bool(*this) == true}
\end{description}

2 \returns
\begin{description}
    \item[---] \cpp{*(this->get())}
\end{description}

%--------------------------------- release() ---------------------------------
\mbrhdr{[[nodiscard]] T* release() noexcept}

1 \effects
\begin{description}
    \item[---] Let \cpp{ptr} be the current value of \cpp{this->get()}.
    \item[---] Stores \cpp{nullptr} as the pointer associated with the
               calling thread.
\end{description}

2 \returns
\begin{description}
    \item[---] \cpp{ptr}
\end{description}

3 \postcond
\begin{description}
    \item[---] \cpp{bool(*this) == false}
\end{description}

%------------------------------ nullary reset() ------------------------------
\mbrhdr{void reset() noexcept}

1 \effects
Equivalent to: \cpp{reset(nullptr);}

2 \postcond
\begin{description}
    \item[---] \cpp{bool(*this) == false}
\end{description}

%------------------------------- unary reset() -------------------------------
\mbrhdr{void reset(T* new\_value)}

1 \effects
\begin{description}
    \item[---] If \cpp{this->get() \!= new\_value \&\& bool(*this)},
               invokes \cpp{this->del(this->get())}.
    \item[---] Stores \cpp{new\_value} as the pointer associated with the
               calling thread.
\end{description}

2 \postcond
\begin{description}
    \item[---] \cpp{this->get() == new\_value}
\end{description}

3 \except
\begin{description}
    \item[---] \cpp{system\_error} if an error occurs.
\end{description}

4 \errors
\cpp{resource\_unavailable\_try\_again} -- the system lacked the necessary
resources to store a new thread specific pointer value.

%------------------------------ operator bool() ------------------------------
\mbrhdr{explicit operator bool() const noexcept}

1 \effects
Equivalent to: \cpp{return (this->get() \!= nullptr);}

\cbend

%---------------------------------- header -----------------------------------
\abschnitt{Header File}

\zs{Add a new header file to Table 24 in \stdsection{16.4.2.3}{headers}:}

\add{\cpp{<thread\_specific\_ptr>}}

%---------------------------- feature test macro -----------------------------
\abschnitt{Feature-test Macro}
\zs{Add a new feature-test macro to \stdsection{17.3.2}{version.syn} as indicated:}

\add{\cpp{#define \__cpp\_lib\_thread\_specific\_ptr 202XXXL // also in <thread\_specific\_ptr>}}

\newpage
\addcontentsline{toc}{subsection}{References}
\begin{thebibliography}{99}

    \bibitem{P0876R18}
        \href{https://www.open-std.org/jtc1/sc22/wg21/docs/papers/2024/p0876r18.pdf}
        {P0876R18: fibers without scheduler}

    \bibitem{Standard}
        \href{https://www.open-std.org/jtc1/sc22/wg21/docs/papers/2024/n4981.pdf}
        {N4981: Working Draft, Programming Languages -- C++}

    \bibitem{Boost}
        \href{https://www.boost.org/doc/libs/release/doc/html/thread/thread_local_storage.html}
        {Boost C++ Libraries: Thread Local Storage}

\end{thebibliography}

%//////////////////////////////////////////////////////////////////////////////

\end{document}
