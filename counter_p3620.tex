\abschnitt{P3620R0: Concerns with the proposed addition of fibers to C++26}

At a high level, P3620R0\cite{P3620R0} appears to argue that unless fibers are
appropriate for all use cases, they must not be available for any use case.
This ignores the industry experience cited in \nameref{low_level}.

Not every C++ feature is applicable to every environment. \cpp{breakpoint()}
is not generally found in production code. A library that writes to
\cpp{std::cerr} will cause problems for an application running in a windowed
environment that has no stderr file handle. A library that throws exceptions
is a poor choice for an application that forbids exceptions. A library that
creates \cpp{std::thread}s will cause trouble for an application that's not
expecting them.

\uabschnitt{Fibers are not lightweight threads}

P3620R0 states that operating system vendors have largely abandoned attempts
to support fibers as N:M threading, because operating system threads have more
state than it's feasible to manage with fibers.

\fiber does not claim to support lightweight threads. \fiber is a tool for
organizing the flow of control within an operating system thread. It does not
need to manage signals, signal masks or other facilities beyond the C++
abstract machine.

\uabschnitt{TLS}

P3620R0 notes that \tlocal storage is shared between all the fibers on a
thread. P3346R0\cite{P3346R0} proposed to modify \tlocal to mean
fiber-specific. This was rejected by SG1 in Wrocław.\cite{wroclawp3346}

This semantic can nonetheless be addressed by a higher-level library. For
instance, Boost.Fiber\cite{bfiber} provides
\href{https://www.boost.org/doc/libs/release/libs/fiber/doc/html/fiber/fls.html}
{\cpp{fiber\_specific\_ptr}}.

P3620R0 further claims that C++20 coroutines do not have this problem.
Actually, they do. If, on entry, a coroutine links an object into a linked
list anchored with a \cpp{static} or \tlocal pointer, then unlinks it on final
return, reaching that coroutine from different interleaved invocation
sequences will corrupt that linked list. This issue did not block adoption
of C++20 coroutines.

It may be worth noting that coroutines provide no entity analogous to a fiber.
It would not be straightforward to support chain-of-coroutines-local storage.

\uabschnitt{Deadlocks}

P3620R0 points out that switching fibers within a thread while holding a lock
may lead to accidental deadlock.

This semantic can be addressed by a higher-level library. For instance,
Boost.Fiber\cite{bfiber} provides fiber-aware synchronization primitives such as
\href{https://www.boost.org/doc/libs/release/libs/fiber/doc/html/fiber/synchronization/mutex_types.html}
{\cpp{boost::fibers::mutex}}.

C++20 coroutines have the same problem. This issue did not block adoption of
C++20 coroutines.

It would not be straightforward to support chain-of-coroutines-aware
synchronization primitives.
