%//////////////////////////////////////////////////////////////////////////////

\documentclass[fontsize=10pt,paper=A4,pagesize,DIV=15]{scrartcl}

\usepackage[T1]{fontenc}
\usepackage[utf8]{inputenc}
\usepackage[american]{babel}        % required for ISO dates
\usepackage[iso,american]{isodate}  % ISO format of dates
\usepackage[final]{listings}        % code listings
\usepackage{booktabs}               % fancy tables
\usepackage[color]{changebar}       % changebars for large inserted passages
\usepackage{longtable}              % auto breaking tables
\usepackage{ltcaption}              % fix captions for long tables
\usepackage{relsize}                % provide relative font size changes
%\usepackage{underscore}             % remove special status of '_' in ordinary text
%\usepackage{verbatim}               % improved verbatim environment
\usepackage{parskip}                % handle non-indented paragraphs "properly"
\usepackage{array}                  % new column definitions for tables
\usepackage[normalem]{ulem}         % underline commands
\usepackage{xcolor}                 % driver-independent color extensions
\usepackage{amsmath}                % mathematical symbols
\usepackage{mathrsfs}               % mathscr font
\usepackage{xspace}                 % inserts a space to replace one "eaten" by TeX
\usepackage[final]{microtype}       % micro-typographic extensions introduced by pdfTeX
\usepackage{xstring}                % manipulating strings
\usepackage{fixme}                  % collaborative annotations
\usepackage{multicol}               % intermix single and multiple columns
\usepackage{perpage}                % counter reset at every page boundary
\usepackage{palatino}               % Adobe Palatino font
\usepackage{overcite}               % citations
\usepackage{boxedminipage}          % framed mini-pages
\usepackage{fancyhdr}               % control of page headers and footers
\usepackage{soul}                   % hyphenatable spacingout), underlining, striking out, et.
\usepackage{svg}                    % SVG pictures
\usepackage{tikz}                   % creating PS and PDF graphics
\usetikzlibrary{arrows,automata}

\cbcolor{green}

\usepackage[pdftex,
            pdftitle    = {fibers without scheduler},
            pdfsubject  = {},
            pdfauthor   = {Oliver Kowalke},
            pdfkeywords = {C++,callcc,call/cc,context,continuation,coroutine,execution,fiber,fiber_context,switch,P0099,P0534,P0876},
            bookmarks=true,
            bookmarksnumbered=true,
            pdfpagelabels=true,
            pdfpagemode=UseOutlines,
            pdfstartview=FitH,
            linktocpage=true,
            colorlinks=true,
            linkcolor=blue,
            plainpages=false
           ]{hyperref}

%//////////////////////////////////////////////////////////////////////////////

\input{commands}

%//////////////////////////////////////////////////////////////////////////////

\begin{document}
\small
\begin{tabbing}
    Document number: \= P0876R21\\
    Date:            \> 2025-07-13\\
    Author:          \> Oliver Kowalke (oliver.kowalke@gmail.com)\\
                     \> Nat Goodspeed (nat.cognitoy@gmail.com)\\
    Audience:        \> LWG, CWG\\
\end{tabbing}

\section*{\emph{fiber\_context} - fibers without scheduler}

%//////////////////////////////////////////////////////////////////////////////

\tableofcontents

%//////////////////////////////////////////////////////////////////////////////

\input{abstract}
\abschnitt{Recent WG21 History}\label{wg21_history}

In St. Louis in June 2024,
\href{https://docs.google.com/document/d/1ebdFai3h2Y4g5NayNf_pG6Gl2qidYdF6G-smMWNw-Go/edit?tab=t.0#heading=h.9aqgmrf0hvh1}{LWG tentatively approved}
P0876 Library wording.

In Tokyo in March 2024, CWG finished initial P0876 Core wording review, with
\href{https://wiki.edg.com/bin/view/Wg21tokyo2024/CoreWorkingGroup#D0876R16_fiber_context}{one requested change}:
that P0876 mandate per-fiber exception state. That required EWG approval.

In St. Louis in June 2024,
\href{https://wiki.edg.com/bin/view/Wg21stlouis2024/NotesEWGP0876}{EWG approved} the change:

\begin{table}[ht]
\begin{tabular}{|r|r|r|r|r|} % right-just columns (5 columns)
\hline %inserts horizontal line
SF & F & N & A & SA \\ [0.5ex] % inserts table heading
\hline % inserts single horizontal line
6 & 8 & 3 & 0 & 0 \\ % [1ex] % [1ex] adds vertical space
\hline %inserts single line
\end{tabular}
\end{table}

However, EWG did not forward P0876 back to CWG, requesting implementation
experience with the proposed change.

In Wrocław in November 2024, Nat Goodspeed presented implementation experience
with libstdc++.
\href{https://wiki.edg.com/bin/view/Wg21wroclaw2024/NotesEWGP0876}{Microsoft requested}
time to consult the backend team. EWG agreed to defer to Hagenberg.

In Hagenberg in February 2025, late in the week,
\href{https://lists.isocpp.org/ext/2025/02/25138.php}{Microsoft conceded} that
per-fiber exception state is implementable with the MSVC runtime (while voicing
performance concerns). Unfortunately this response arrived so late that EWG
ran out of time without considering P0876.

In Sofia in June 2025,
\href{https://docs.google.com/document/d/1wItX212LurEjkJK53XxnxumciqTRzxTSTJ48GwFihpE/edit?tab=t.0#heading=h.zc6lxwkrnobj}{EWG forwarded P0876}
back to CWG and LWG for inclusion in C++26:

\begin{table}[ht]
\begin{tabular}{|r|r|r|r|r|} % right-just columns (5 columns)
\hline %inserts horizontal line
SF & F & N & A & SA \\ [0.5ex] % inserts table heading
\hline % inserts single horizontal line
10 & 14 & 4 & 5 & 1 \\ % [1ex] % [1ex] adds vertical space
\hline %inserts single line
\end{tabular}
\end{table}

But both CWG and LWG ran out of time in Sofia without considering P0876,
thereby postponing it to C++29.

Concerning a <feature> that fails to make the deadline for C++<NN>,
\href{https://www.open-std.org/jtc1/sc22/wg21/docs/papers/2024/p1000r6.pdf}{P1000R6}
says:
\begin{quote}
Just wait a couple more meetings and C++<NN+3> will be open for business and <feature> can be
the first thing voted into the C++<NN+3> working draft.
\end{quote}

This is the promise of the train model. It matters to all of us that the train
model works as promised.

\abschnitt{Revision History}\label{history}
This document supersedes P0876R20.

\uabschnitt{Changes since P0876R20}

\begin{itemize}
    \item Apply P3472R1: Make \fiber\cpp{::}\canresume \cpp{const}.
    \item Remove ``Instantiating'' from proposed wording. Remove remaining
          instances of ``instance'' in front matter.
    \item Clarify that bad behaviour in Appendices A and B is observed only in
          implementations predating proposed changes to \stdclause{except}.
    \item Add ``Recent WG21 History'' section.
\end{itemize}

\uabschnitt{Changes since P0876R19}

\begin{itemize}
    \item Add information about implementability of per-fiber exception state.
    \item Add links to St. Louis 2024 EWG notes, Wrocław 2024 EWG notes and
          Microsoft implementability email.
    \item Add discussion of P3620R0.
    \item Mention P3367R3 constexpr coroutines.
\end{itemize}

\uabschnitt{Changes since P0876R18}

\begin{itemize}
    \item Move exception state test programs to Appendices.
    \item Link Boost.Context patch that produces correct fiber-specific
          exception behavior on Windows and Linux using libstdc++.
    \item Add references to six additional production libraries built on fiber
          technology.
\end{itemize}

\uabschnitt{Changes since P0876R17}

\begin{itemize}
    \item Distinguish between a \emph{prepared} and a \emph{suspended} fiber.
    \item Distinguish the two context switches implied by entry to, and return
          from, \resumewith.
    \item Remove \cpp{current\_exception\_within\_fiber()}, which became moot
          in P0876R17.
\end{itemize}

\uabschnitt{Changes since P0876R16}

\begin{itemize}
    \item Update to reference N4981.
    \item Add \cpp{<fiber\_context>} header file to headers table.
    \item Remove \resumewith ``Case A'' and ``Case B'' in favor of
          nested bullet lists. Fix a bug in definition of internal-resume.
    \item Revert \resumewith\xspace\returns and \except clauses to R15
          structure, eliminating ``Case C'' and ``Case D''.
    \item Use scoped exposition-only terms \emph{calling fiber}, \emph{target
          fiber} and \emph{previous fiber} instead of quoting the phrases.
          Give previous fiber definition its own bullet.
    \item Eliminate \emph{internal-resume} parameter \cpp{after}, also
          definitions of \cpp{after\_entry\_copy}, \cpp{after\_stack\_copy}
          and \cpp{after\_deleter\_copy}. Describe \emph{internal-resume} in
          terms of the currently running fiber.
    \item Explicitly state that \emph{internal-resume} is exposition only, and
          italicize references.
    \item Move predicate for first \emph{internal-resume} definition to start
          of bullet text. Don't state the inverse predicate for the second.
    \item Remove one level of bullet list nesting from the
          second \emph{internal-resume} definition. Sequence the bullet list
          by appending ``, then'' after each item.
    \item Per EWG in St. Louis, remove implementation defined meaning
          of \emph{currently handled exception} and \cpp{uncaught\_exceptions}.
          Now both are fiber-specific.
    \item Clarify explicit constructor \except clause.
\end{itemize}

\uabschnitt{Changes since P0876R15}

\begin{itemize}
    \item \null [fiber.context.overview] is now ``Overview'' instead of ``Preamble''.
    \item Make default \fiber constructor \cpp{= default}.
          Remove its section from member descriptions.
    \item In unary constructor, move ``Mandates'' before ``Constraints''. The
          ``Preconditions'' entry is actually a Constraint: move it and remove
          ``Preconditions''.
    \item In span constructor, stated ``Preconditions'' are actually ``Constraints''.
    \item In constructor descriptions, use less precise language about copying
          \cpp{entry}, \cpp{stack} and \cpp{deleter}. Add Note about them not
          being \fiber members. Move mention of stack to a Note.
    \item Rephrase \resumewith Note about emptying its \fiber object to avoid
          the appearance of a normative statement.
    \item Remove mention of ``legacy behaviour'' from
          \cpp{current\_exception\_within\_fiber()}.
    \item Remove mention of ``thread of execution'' from ``abstract,''
          ``control transfer mechanism'' and the section on \exfns.
    \item Simplify definitions of implicit fiber vs. explicit fiber.
    \item Add [intro.fibers] statement that a thread is always running one
          fiber, but can switch between fibers. This replaces the more
          detailed description of what happens when a fiber calls \anyresume.
    \item In [intro.fibers], hoist ``owning thread'' definition to its own
          paragraph 3 and clarify.
    \item Remove assertion that a fiber is an execution agent.
    \item Modify \stdclause{except.throw} paragraphs 2 and 4, and \stdclause{except.handle}
          paragraph 6, to constrain exception propagation to a fiber.
    \item Describe explicit fiber as being ``prepared,'' with a statement that
          it comes into existence on first resumption.
    \item Remove a few stray instances of ``may''.
    \item Move assertion that a received \fiber object could represent either
          an explicit fiber or an implicit fiber to a Note.
    \item Move assertion that no \fiber object represents a running fiber up
          to Overview.
    \item Use \cpp{successor} rather than the more generic \cpp{continuation}
          to reference the \fiber object returned by a terminating fiber.
    \item Remove nesting from \resumewith \except.
    \item Remove Note that the caller of \resumewith can detect whether the
          previous fiber has terminated: not necessarily.
    \item Hoist section on \exfns to have its own table of contents entry.
          Extend with examples of bad behavior when switching out of a catch
          block to a fiber which itself catches some exception before
          switching back to the original fiber.
    \item Remove explicit \cpp{delete} declarations of copy constructor and
          copy assignment: these are implicitly deleted.
    \item Since we want the constructor's \cpp{entry} and \cpp{deleter}
          parameters to support move-only objects,
          remove \emph{Cpp17CopyConstructible} requirements.
    \item For the same reason, state that \cpp{entry\_copy}
          and \cpp{deleter\_copy} are initialized rather than copied.
    \item Therefore ``any exception from initialization of \cpp{entry\_copy}''
          and the same for \cpp{deleter\_copy}.
    \item Remove mention of ``function call stack'' from constructor \except.
    \item \cpp{stack.data()} and \cpp{stack.size()} must meet implementation
          requirements, not the \cpp{span<byte>} itself.
    \item Remove \postcond \cpp{other.empty()} from move constructor and
          move assignment: these are implied by definition.
    \item Move statement about UB from stack overflow to \fiber Overview.
    \item Modify example about early destruction of exceptions to add sequence
          comments, highlight access to a destroyed exception object.
    \item Fix erroneous [fibercontext.mumble] references in class comments.
    \item Add green changebars for entirely new sections.
    \item Remove \cpp{std::} qualification from \cpp{decay\_t} in \effects.
    \item Remove the destructor Note about encouraging a fiber to terminate
          voluntarily.
    \item Clarify that \cpp{current\_exception\_within\_fiber()} is \true if
          \curex reports exceptions ``only'' within the current fiber.
          Remove \cpp{constexpr}: compiler can produce object code that might
          be linked with alternative runtimes.
    \item Remove ``.'' after ``;''.
    \item \resumewith\xspace\mandates\xspace\cpp{is\_invocable\_r<...>} is \true.
          Add periods to \mandates and \precond.
    \item Add \precond to span constructor that \cpp{deleter} must not throw.
          Remove cleanup exceptions from \resumewith \xspace \except. Remove
          ``before this point, no exceptions'' bullet in \effects.\
    \item \resumewith evaluates \cpp{invoke\_r(fn)}. Merge Notes about what
          its \cpp{returned} can be.
    \item Substantially rework \resumewith description. Break out and label
          the four cases: (target not yet entered, target previously
          suspended); (previous exited, previous called \resumewith). Use case
          labels in \effects, \returns and \except. Break out internal-resume
          operation because it's self-referential.
    \item Add span constructor \precond for \cpp{decay\_t<D>}
          meeting \emph{Cpp17MoveConstructible} requirements.
    \item Remove \canresume Note about ``can resume.''
    \item For \resume, use \cpp{std::identity} instead of identity lambda.
\end{itemize}

\uabschnitt{Changes since D0876R15}

\begin{itemize}
    \item Updated to reference N4971.
    \item Inserted a section to clarify relationship between threads and fibers.
    \item Borrowed ``single flow of control'' definition for ``fiber.''
    \item Added Note clarifying ``flow of control'' as state, with reference
    to \stdclause{stacktrace.general}.
    \item Changed stacktrace ``invocation sequence'' to reference ``fiber'' rather
    than ``thread of execution.''
    \item Changed ``thread'' definition to be the execution agent that runs fibers.
    \item Clarified that if a fiber terminates by returning an empty \fiber
          object, \cpp{std::terminate} is called.
    \item Added \cpp{constexpr fiber\_context::current\_exception\_within\_fiber}.
    \item Removed definition of ``function call stack.''
    \item Removed change to definition of expression evaluation conflict.
    \item Removed Note about the second fiber in the program.
    \item Changed ``\fiber instance'' to ``\fiber object.''
    \item Changed ``method'' to ``member function.''
    \item Removed paragraph numbers from internal cross-references.
    \item Clarified editorial directives amongst not-green new text.
    \item Used ``fiber.context'' in stable labels.
    \item Changed the lone remaining preamble section in [fiber.context] from
    ``Empty vs. Non-Empty'' to ``Preamble.''
    \item Moved to ``Preamble'' the 1:1 relationship between non-empty \fiber
    objects and suspended fibers.
    \item Used ``Effects: Equivalent to \cpp{return <expression>}'' for
    \cpp{empty()} and \cpp{operator bool()}.
    \item Referenced \justmain instead of \main.
\end{itemize}

\uabschnitt{Changes since P0876R14}

\begin{itemize}
    \item Invoked ``blocks with forward progress guarantee delegation'' words
          of power for \resumewith, guaranteeing mutual exclusion.
    \item Fixed Mandates and Throws concerning the entry function and deleter
          passed to the implicit-stack or explicit-stack constructor.
    \item Cleaned up wording around initializing, assigning and testing the
          exposition-only \cpp{state} member.
    \item Dampened the optimism of the proposed feature-test macro.
\end{itemize}

\uabschnitt{Changes since P0876R13}

\begin{itemize}
    \item At LEWG's request, retracted changes to \cpp{uncaught\_exceptions()}
          and \cpp{current\_exception()}, instead clarifying that results may
          reflect exceptions on other fibers running on the current thread.
    \item Updated against draft standard N4958.
    \item Deleted ``User-Mode'' from new section title ``Cooperative Threads''
          and removed the explanatory paragraph.
    \item Removed \cpp{explicit} from the explicit-stack constructor.
    \item Added \cpp{system\_error}: \cpp{resource\_unavailable\_try\_again}
          to the \except clause of the implicit-stack constructor.
    \item Changed \cpp{bad\_alloc}
          to \cpp{system\_error}: \cpp{resource\_unavailable\_try\_again}
          in the \except clause of the explicit-stack constructor.
    \item Stated that the move constructor and move assignment operator empty
          the moved-from \fiber.
    \item Removed the \emptyfn precondition from assignment operator; instead
          added the same \cpp{(\! empty())} effect as for the destructor.
    \item Removed \resumewith references to ``execution context.'' Existing
          section 7.6.1.3 Function call \stdclause{expr.call} makes no mention
          of saving or restoring state.
    \item Removed bullets in \resumewith \returns and \except clauses
          regarding \resume, since they can be inferred from \resumewith and
          the trivial-lambda equivalence described for \resume.
    \item Removed the \remarks about concurrent calls from multiple threads
          from \canresume, leaving in place the editorial note about the
          intentional absence of \cpp{const}.
    \item Changed exposition-only \cpp{state} member from unspecified-type
          to \cpp{void*}.
    \item Sanitized stable names.
    \item Moved feature-test macro to appropriate section.
    \item Cleaned up the header-file synopsis.
    \item Grouped class members with forward references.
    \item Added \cpp{std::swap()} specialization.
    \item Added obtrusive paragraph numbers.
    \item Streamlined single-item dash lists.
    \item Changed \textit{Ensures} to \textit{Postconditions}.
    \item Changed template parameters from \cpp{typename} to \cpp{class}.
    \item Tweaked constructor \precond / \mandates.
    \item Clarified that \cpp{entry\_copy}, \cpp{stack\_copy}
          and \cpp{deleter\_copy} are not intended to be data members
          of \fiber.
    \item Streamlined initialization of these exposition objects.
    \item ``Instantiates a \fiber'' => ``Initializes \cpp{state}''
    \item \emptyfn returns \true => \emptyfn is \true, et al.
    \item Removed explicit-stack constructor \precond for stack size and
          alignment, since \except explicitly specifies exceptions for
          violations.
    \item Rephrased \effects of move constructor.
    \item Extracted ``Let'' statements from \effects to preceding paragraphs.
\end{itemize}

\uabschnitt{Changes since P0876R12}

\begin{itemize}
    \item Proposed that \cpp{uncaught\_exceptions()}
          and \cpp{current\_exception()} be specific to the current thread of
          execution.
    \item Specified that constructors \emph{decay-copy} the \entryfn.
    \item Changed \cpp{span<byte, N>} constructor param to
          simply \cpp{span<byte>}; also accepted deleter function, which it
          must \emph{decay-copy}.
    \item Specified constructor exceptions.
    \item Specified that destroying a non-empty \fiber
          calls \cpp{terminate()}.
    \item Clarified that when \resumewith is called, \emptyfn becomes
          true immediately.
    \item Introduced exposition-only \cpp{fiber\_context::state} member to
          streamline wording.
    \item Removed \cpp{concurrency\_v2} namespace.
    \item Changed ``Equivalent to'' to ``As-if''.
    \item Clarified Preconditions vs. Mandates.
\end{itemize}

\uabschnitt{Changes since P0876R11}

\begin{itemize}
    \item Removed \getsource, \gettoken, \reqstop and
          exposition-only \cpp{ssource} members.
    \item Added a \fiber constructor accepting a caller-provided
          uninitialized memory area for the new fiber's function call stack.
\end{itemize}

Bundling a \cpp{stop\_source} into \fiber presented implementability concerns.
Although each fiber (specifically, its function call stack) is itself a
persistent entity, the \fiber representing that fiber is not: a
new \fiber object is synthesized on every suspension. This
presents a problem: how does the code that suspends a fiber find its
associated \cpp{stop\_source} shared state?

A consumer wishing to pass a \cpp{std::stop\_token} to a new fiber can itself
construct a \cpp{std::stop\_source}, obtain from it a \cpp{stop\_token} and
bind that \cpp{stop\_token} in a lambda passed to the \fiber
constructor. Accordingly, the \fiber API need not explicitly support that.

\uabschnitt{Changes since P0876R10}

\begin{itemize}
    \item Removed \cpp{cancel()} method and the \cancelfn constructor
          argument. Replaced with the \cpp{std::jthread} stop token handling
          API: \getsource, \gettoken and \reqstop. This simplifies examples by
          eliminating \cpp{launch()} and \cpp{assert\_on\_cancel}.
    \item Added a section exploring the relationship of \fiber to the larger
          C++ ecosystem.
    \item Reordered some sections to make the paper more accessible for new readers.
\end{itemize}

\uabschnitt{Changes since P0876R9}

\begin{itemize}
    \item Removed \xtresume, \xtresumewith, \xtcancel and \canxtresume, along
          with stated support for resuming a suspended fiber on some thread
          other than the one on which it was launched.
\end{itemize}

In Belfast, EWG came down strongly against cross-thread fiber resumption. The
most emphatic objection was that for a function referencing TLS, multiple
compilers cache TLS pointers on the function's stack frame. Resuming a fiber
containing that stack frame on some other thread would cause problems. In the
best case, the resumed function would merely reference TLS belonging to the
wrong thread -- but at some point the original thread will terminate, its TLS
will be destroyed, and the cached pointers will be left dangling.

With \fiber, any opaque function call might possibly suspend -- but
invalidating cached TLS pointers across every opaque function call is deemed
unacceptable overhead.

\uabschnitt{Changes since P0876R8}

\begin{itemize}
    \item Reinstated cancellation function constructor argument.
    \item Added \cpp{cancel()} and \cpp{cancel\_from\_any\_thread()} member
          functions.
    \item Re-removed \unwindfib.
\end{itemize}

SG1 directed P0876R9 to conform to the Cologne 2019 recommendations, with any
other changes proposed in a separate paper.

\uabschnitt{Changes since D0876R7}

\begin{itemize}
    \item Cancellation function removed from \fiber constructor.
    \item \unwindfib re-added, with implementation-defined behaviour.
    \item Added elaboration of \cpp{filament} example to bind cancellation
          function.
\end{itemize}

P0876R8 diverged from the recommendations of the second SG1 round in Cologne
2019. It did not introduce \cpp{cancel()} or \cpp{cancel\_from\_any\_thread()}
member functions. In fact it removed the \cancelfn constructor argument.

\fiber is intended as the lowest-level stackful context-switching API. Binding
a \cancelfn on the fiber stack is a flourish rather than a necessity. It adds
overhead in both space (on the fiber stack) and time (to traverse the stack to
retrieve the \cancelfn). For this API, it should suffice to pass the desired
\cancelfn to \anyresumewith. If it is important to associate a \cancelfn with
a particular fiber earlier in the lifespan of the fiber, a struct serves.

A more compelling reason to avoid constructing an explicit fiber with
a \cancelfn is that no implicit fiber has any such \cancelfn\xspace -- and the
consuming application cannot tell, a priori, whether a given \fiber object
represents an explicit or an implicit fiber. If \this represents an
implicit fiber, what should the proposed \cpp{cancel()} member function do?

Passing a specific \cancelfn to \anyresumewith avoids that problem.

P0876R8 follows SG1 recommendation in making it Undefined Behaviour to destroy
(or assign to) a non-empty \fiber object.

\unwindfib was reintroduced with implementation-defined behaviour to allow fiber
cleanup leveraging implementation internals. Its use was entirely optional (and
auditable). 

\uabschnitt{Changes since P0876R6}

\begin{itemize}
    \item Implicit stack unwinding (by non-C++ exception) removed.
    \item \unwindfib removed.
    \item Cancellation function added to \fiber constructor.
\end{itemize}

In Cologne 2019, SG1 took the position that:

\begin{itemize}
    \item The \fiber facility is not the only C++ feature that
          requires ``special'' unwinding (special function exit path).
    \item Such functionality should be decoupled from \fiber. It requires its
          own proposal that follows its own course through WG21 process.
    \item Depending on this (yet to be written) proposal would unduly delay
          the \fiber facility.
    \item For now, the \fiber facility should adopt a ``less is
          more'' approach, removing promises about implicit unwinding, placing
          the burden on the consumer of the facility instead.
    \item This leaves the way open for \fiber to integrate with
          a new, improved unwind facility when such becomes available.
\end{itemize}

The idea of making \fiber's constructor accept a cancellation function was
suggested to permit consumer opt-in to P0876R5 functionality where
permissible, or convey to the fiber in question by any suitable means the need
to clean up and terminate.

Requiring the cancellation function is partly because it remains unclear what
the default should be. This could be one of the questions to be answered by a
TS. Moreover, the absence of a default permits specifying later that the
default engages the new, improved unwind facility.

\uabschnitt{Changes since P0876R5}

\begin{itemize}
    \item \cpp{std::unwind\_exception} removed.
    \item \cpp{fiber\_context::can\_resume\_from\_any\_thread()} renamed to
      \canxtresume.
    \item \cpp{fiber\_context::valid()} renamed to \emptyfn with inverted
      sense.
    \item Material has been added concerning the top-level wrapper
      logic governing each fiber.
\end{itemize}

\unwindex was removed in response to deep
discussions in Kona 2019 of the surprisingly numerous problems surfaced by
using an ordinary C++ exception for that purpose.

Problems resolved by discarding \unwindex:
\begin{itemize}
    \item When unwinding a fiber stack, it is essential to know the subsequent
          fiber to resume. \unwindex therefore bound a \fiber. \fiber is
          move-only. But C++ exceptions must be copyable.
    \item It was possible to catch and discard \unwindex, with problematic
          consequences for its bound \fiber.
    \item Similarly, it was possible to catch \unwindex but not rethrow it.
    \item If we attempted to address the problem above by introducing a
          \unwindex operation to extract the bound \fiber, it became possible
          to rethrow the exception with an empty (moved-from) \fiber object.
    \item Throwing a C++ exception during C++ exception unwinding terminates
          the program. It was possible for an exception implementation based
          on \cpp{thread\_local} to become confused by exceptions on different
          fibers on the same thread.
    \item It was possible to capture \unwindex with \cpp{std::exception\_ptr}
          and migrate it to a different fiber -- or a different thread.
\end{itemize}

\newpage
\input{counter_p3620}
\abschnitt{\fiber and the larger C++ ecosystem}

\uabschnitt{higher-level libraries}

\nameref{low_level} enumerates a number of higher-level abstraction libraries
built upon the \bcontext\xspace implementation of the API proposed in this paper.
This is not an exhaustive list, but it suffices to illustrate that there is
widespread interest in this functionality.

The most significant point about this proposal is that, given \fiber, all
those libraries can be written in standard C++. They need not themselves be
integrated into the Standard.

Because it creates and switches between different function call stacks,
though, the \fiber facility cannot be written in portable C++. There is real
value to integrating this library into the Standard.

\emph{Boost.Context} is maintained by one individual to support the specific
set of processors and operating systems to which he has access. The \fiber
facility will ensure support in every implementation of the C++ runtime,
extending into the future.

Given the lively ecosystem of open-source libraries, it's possible that
standardizing \fiber could suffice. It is not essential that
WG21 must standardize additional higher-level libraries before the facility
would become useful. The uptake of \emph{Boost.Context} illustrates that the
community can make good use of \fiber.

However, the evolution of this proposal and the WG21 discussions thereof have
surfaced a number of interesting adjacencies.

\uabschnitt{cancellation}

Given C++ support for concurrency, in various forms, within a program,
cancellation of an asynchronous task remains a topic of widespread interest.
It has been much discussed, e.g. in P1677R2\cite{P1677R2},
P1820R0\cite{P1820R0} and P2175R0\cite{P2175R0}.

Previous revisions of this paper have proposed canceling a suspended fiber by
injecting an exception, e.g. using \fiber\cpp{::}\resumewith. A comparable
approach was rejected for \cpp{std::jthread}, although it's worth noting that
cooperative fibers differ in a very significant respect: every fiber suspends
at a well-defined point, namely a call to \resumewith.\footnote{Although
exception-based cancellation is not implicitly supported, a consumer of \fiber
may still explicitly pass to \resumewith an invocable that raises an exception
in the suspended fiber.}

Evolution of the exception mechanism itself\cite{P0709R4} may affect the
viability of using exceptions for cancellation.

This paper simply notes that an invoker can use lambda binding to pass (e.g.)
a \cpp{std::stop\_token} from the Standard\cite{Standard}, section 33.3, to a
fiber at launch time.

\uabschnitt{modules and optimizations}

Before modules, the only information the compiler could know about a function
in an external translation unit was what a human coder stated in the relevant
header file. But since the information in a module is prepared by the compiler
itself, a subsequent compile of a translation unit that imports that module
can know as much about each module function as it would if the function's
source code was found within the current translation unit.

This permits the compiler to infer and propagate attributes. If a function
neither contains a throw statement nor calls other functions, the compiler can
conclude that it doesn't throw. It can encode this information in the module
produced for that translation unit, so that subsequent compiles can make use
of the knowledge. If another function contains no throw statement and calls
only functions known not to throw, it too can be implicitly marked nothrow.

Similarly, when compiling a function that can never return, the compiler can
so indicate in the output module. Any caller whose code path leads
unconditionally to any such function can also be known never to return.

In much the same way, the module describing the
library's \fiber\cpp{::}\resumewith method can mark it as \emph{can-suspend}.
Then any caller of \resumewith will also be marked \emph{can-suspend}, and so
forth. The compiler can use this to improve its optimization tactics around
any call to a \emph{can-suspend} function.

(The \emph{can-suspend} characteristic of a \cpp{co\_await} coroutine function
is just as pervasive, but in that case the coder must manually propagate it.)

\uabschnitt{synchronization primitives}

The Standard\cite{Standard} provides an assortment of primitives for
synchronizing work between threads, e.g. sections 33.6, 33.7, 33.8, 33.9,
33.10. An essential behaviour of many such synchronization primitives is to
pause, or suspend, execution of the current thread until some external
condition is satisfied.

Such suspension is very different from fiber suspension as proposed in this
paper. This proposal neither requires nor implies a scheduler. A fiber
suspends by explicitly designating the next fiber to resume, either by passing
its \fiber to \resumewith or by returning that \fiber from its \entryfn.

C++ threads, in contrast, assume a thread scheduler, usually provided by the
operating system. Suspending a thread means passing control to the scheduler,
which reallocates CPU resources to other pending threads. At some future time,
the scheduler is responsible for directing some CPU core to resume the suspended
thread.

Fiber suspension as implemented by \fiber is independent of thread suspension.
Suspending the running fiber simply means directing the thread to run a
different fiber; the thread continues running. Conversely, suspending the host
thread (e.g. by invoking a synchronization primitive) means that \emph{no}
fiber is running on that thread.

A higher-level fiber-based library that emulates the \cpp{std::thread} API,
such as \bfiber\cite{bfiber}, necessarily implements a fiber scheduler,
permitting implicit fiber suspension. Standardizing such a library would raise
the interesting question of how to present fiber-aware synchronization
primitives.

A straightforward approach is to present a suite of fiber-aware
synchronization primitives distinct from, but analogous to, the thread-based
synchronization primitives.\footnote{This is the approach taken
by \emph{Boost.Fiber}.} A program running multiple fibers within a thread
would use fiber-aware synchronization primitives rather than thread-based
synchronization primitives. Evaluating a thread-based synchronization
primitive would suspend the entire thread, as usual, halting all fibers within
that thread.

It is tempting to contemplate modifying the semantics of the present suite of
synchronization primitives to make them fiber-aware. Naturally this is a
matter of some concern.

For purposes of this \fiber proposal, though, it is entirely moot.

\uabschnitt{Execution Agent Local Storage}

A similar question arises concerning variable storage duration. Should the
Standard introduce a fiber-specific storage duration, e.g. \cpp{fiber\_local},
analogous to \cpp{thread\_local}\cite{Standard}? (section 6.7.5.3 \bfs{Thread
storage duration})

The Standard defines the general term \emph{execution agent} (section
33.2.5.1) to allow for multiple kinds of parallelism. It seems reasonable to
assume that over time, new types of execution agents will be defined. Will we
want the Standard to present a new \cpp{xyz\_local} storage duration for each
new ``xyz'' execution agent type?

P0772R1\cite{P0772R1} notes that library code should not have to care what
kind of execution agent is running it. Already it's important to ensure that
library code avoids \cpp{static} variables because any such variable prohibits
calling that library from more than one thread. P0772R1 suggests a generalized
variable storage duration dynamically local to the innermost current execution
agent.

(The same consideration about library code impacts the above question about
presenting fiber-aware synchronization primitives.)

It's true that if:

\begin{itemize}
    \item on fiber X, function F relies on a \cpp{thread\_local} variable V
    \item function F calls function G that resumes fiber Y
    \item fiber Y calls function F, or another function that modifies variable V
    \item fiber Y resumes fiber X
    \item on fiber X, function G returns to function F
\end{itemize}

then function F on fiber X will observe fiber Y's value for variable V.

This is analogous to use of a \cpp{static} variable by multiple threads in the
same program -- though not as bad, since it doesn't produce race-related
Undefined Behaviour on top of correctness problems.

\cpp{std::thread} was introduced despite this problem because it's \emph{useful.}

Multiple C++ implementations cache a pointer to thread-local storage in the
stack frame of a function referencing TLS. If a suspended fiber were resumed
by a thread other than the one on which it previously ran, such cached TLS
pointers would point to TLS for the wrong thread. This is why such
cross-thread resumption is forbidden.

(This is the only optimization that has yet been surfaced by implementers as a
potentially problematic interaction with fibers.)

P3346R0\cite{P3346R0} proposed to modify \tlocal to mean fiber-specific. This
was rejected by SG1 in Wrocław in 2024\cite{wroclawp3346}.

That said, in an environment in which \tlocal referenced fiber-specific
storage, TLS pointers cached in function stack frames would remain valid even
if the original fiber were later resumed on some other thread, thus removing
the restriction against cross-thread resumption.

\uabschnitt{tooling} One particularly valuable consequence of adding \fiber to
the Standard will be to add fiber awareness to debuggers, performance
analyzers and other tools that inspect a running C++ program.

Such tools need only be aware of \fiber. They would \emph{not} need to be
further adapted to support higher-level libraries built on
the \fiber facility.

\newpage

\input{control_transfer}
\input{first_class}
\input{stack_management}
\abschnitt{invalidation at resumption}\label{invalidation}

The framework must prevent the resumption of an already running or terminated
(computation has finished) fiber.

Resuming an already running fiber will cause overwriting and corrupting the stack
frames (note, the stack is not copyable).  Resuming a terminated fiber will
cause undefined behaviour because the stack might already be unwound (objects
allocated on the stack were destroyed or the memory used as stack was already
deallocated).

As a consequence each call of \resume will empty the \fiber object.

Whether or not a \fiber is empty can be tested with member function \opbool.

To make this more explicit, functions \allresume are rvalue-reference qualified.
%If a fiber calls \cpp{f.resume()} then the  fiber is suspended and \cpp{f} is
%invalidated. When the fiber is resumed later, it returns from \cpp{f.resume()}
%and object \cpp{f} references the calling fiber (the fiber that has resumed
%the current fiber).

The essential points:
\begin{itemize}
    \item regardless of the number of \fiber declarations, exactly one \fiber
          object represents each suspended fiber
    \item no \fiber object represents the currently-running fiber
\end{itemize}

Section \nameref{solution_gpub} describes how an object of type\\
\fiber is synthesized from the active fiber that suspends.

\zs{
A fiber API must:
\begin{itemize}
    \item prevent accidentally resuming a running fiber
    \item prevent accidentally resuming a terminated fiber
    \item \allresume are rvalue-reference qualified
\end{itemize}
}

\abschnitt{problem: avoiding non-const global variables and
undefined behaviour}\label{problem_gpub}

According to \emph{C++ core guidelines}\cite{coreguidlines}, non-const global
variables should be avoided: they hide dependencies and make the
dependencies subject to unpredictable changes.

Global variables can be changed by assigning them indirectly using a pointer or
by a function call. As a consequence, the compiler can't cache the value of a
global variable in a register, degrading performance (unnecessary
loads and stores to global memory especially in performance critical loops).

Accessing a register is one to three orders of magnitude faster than accessing
memory (depending on whether the cache line is in cache and not invalidated by
another core; and depending on whether the page is in the TLB).

The order of initialisation (and thus destruction) of static global
variables is not defined, introducing additional problems with static
global variables.

\zs{A library designed to be used as building block by other higher-level
frameworks should avoid introducing global variables. If this API were
specified in terms of internal global variables, no higher level layer could undo
that: it would be stuck with the global variables.}

\uabschnitt{switch back to \emph{main()} by returning}
Switching back to \main by returning from the fiber function has two drawbacks:
it requires an internal global variable pointing to the suspended \main and
restricts the valid use cases.
\cppf{return_to_main}
For instance the generator pattern is impossible because the only way
for a fiber to transfer execution control back to \main is to terminate. But
this means that no way exists to transfer data (sequence of values) back and
forth between a fiber and \main.

\zs{Switching to \main only by returning is impractical because it limits the
applicability of fibers and requires an internal global variable pointing to
\main.}

\uabschnitt{static member function returns active \fiber}
P0099R0\cite{P0099R0} introduced a static member function\\
(\cpp{execution_context::current()}) that returned an object representing the active
fiber. This allows passing the active fiber \cpp{m} (for instance representing
\main) into the fiber \cpp{f} via lambda capture. This mechanism enables
switching back and forth between the fiber and \main, enabling a rich set of
applications (for instance generators).
\cppf{static_current}

But this solution requires an internal global variable pointing to the active
fiber and some kind of reference counting. Reference counting is needed because
\cpp{fiber\_context::current()} necessarily requires multiple objects of \fiber for the
active fiber. Only when the last reference goes out of scope can the fiber be
destroyed and its stack deallocated.
\cppf{multi_current}

Additionally a static member function returning an object representing the active fiber
would violate the protection requirements of sections \nameref{stackmgmt} and
\nameref{invalidation}. For instance you could accidentally attempt to resume
the active fiber by invoking \resume.
\cppf{resume_current}

\zs{A static member function returning the active fiber requires a reference
counted global variable and does not prevent accidentally attempting to resume
the active fiber.}

\abschnitt{solution: avoiding non-const global variables and
undefined behaviour}\label{solution_gpub}

\zs{The \emph{avoid non-const global variables} guideline has an important
impact on the design of the \fiber API!}

\uabschnitt{synthesizing the suspended fiber}\label{synthesizing}
The problem of global variables or the need for a static member function
returning the active fiber can be avoided by \bfs{synthesizing} the
\bfs{suspended fiber} and passing it into the resumed fiber (as parameter when the
fiber is first started, or returned from \resume).
\cppfl{synthesized_foo}

In the pseudo-code above the fiber \cpp{f} is started by invoking its member
function \resume at line 7. This operation suspends \cpp{foo}, empties
object \cpp{f} and synthesizes a new \fiber\xspace \cpp{m} that is passed as parameter
to the lambda of \cpp{f} (line 2).

Invoking \cpp{m.resume()} (line 3) suspends the lambda, empties \cpp{m} and
synthesizes a \fiber that is returned by \cpp{f.resume()} at line 7. The
synthesized \fiber is assigned to \cpp{f}. Object \cpp{f} now represents the
suspended fiber running the lambda (suspended at line 3). Control is
transferred from line 3 (lambda) to line 7 (\cpp{foo()}).

Call \cpp{f.resume()} at line 8 empties \cpp{f} and suspends \cpp{foo()}
again. A \fiber representing the suspended \cpp{foo()} is synthesized, returned
from \cpp{m.resume()} and assigned to \cpp{m} at line 3. Control
is transferred back to the lambda and object \cpp{m} represents the suspended
\cpp{foo()}.

Function \cpp{foo()} is resumed at line 4 by executing \cpp{m.resume()} so that
control returns at line 8 and so on ...

Class \cpp{symmetric\_coroutine<>::yield\_type} from N3985\cite{N3985} is
\bfs{not} equivalent to the synthesized \fiber.

\cpp{symmetric\_coroutine<>::yield\_type} does not represent the suspended context,
instead it is a special representation of the same coroutine. Thus \main or
the current thread's \entryfn can \bfs{not} be represented by \cpp{yield\_type}
(see next section \nameref{representation}).

Because \cpp{symmetric\_coroutine<>::yield\_type()} yields back to the starting
point, i.e. invocation of\\
\cpp{symmetric\_coroutine<>::call\_type::operator()()},
both objects (\cpp{call\_type} as well as \cpp{yield\_type}) must be preserved.
Additionally the caller must be kept alive until the called coroutine terminates
or UB happens at resumption.

\zs{This API is specified in terms of passing the suspended \fiber. A higher
level layer can hide that by using private variables.}

\uabschnitt{representing \emph{main()} and thread's \entryfn as fiber}\label{representation}
As shown in the previous section a synthesized object of type \fiber is passed
into the resumed fiber.

\cppf{synthesized_main}

The mechanism presented in this proposal describes switching between stacks: each
fiber has its own stack. The stacks of \main and explicitly-launched threads
are not excluded; these can be used as targets too.

\bfs{Thus every program can be considered to consist of fibers -- some
created by the OS (\main stack; each thread's initial stack) and some created
explicitly by the code.}

This is a nice feature because it allows (the stacks of) \main and each
thread's \entryfn to be represented as fibers. A \fiber
representing \main or a thread's \entryfn can be handled like an
explicitly created \fiber: it can passed to and returned from functions or
stored in a container.

In the code snippet above the suspended \main is represented by object
\cpp{m} and could be stored in containers or managed just like \cpp{f}
by a scheduling algorithm.

\zs{The proposed fiber API allows representing and handling \main and the
current thread's \entryfn by an object of type \fiber in the same way as
explicitly created fibers.}

\uabschnitt{fiber returns (terminates)} When a fiber returns (terminates), what
should happen next? Which fiber should be resumed next? The only way to avoid
internal global variables that point to \main is to explicitly return a non-empty
\fiber object that will be resumed after the active fiber terminates.
\cppfl{terminating_fiber}

In line 5 the fiber is started by invoking \resume on object \cpp{f}. \main
is suspended and an object of type \fiber is synthesized and passed as
parameter \cpp{m} to the lambda at line 2. The fiber terminates by returning
\cpp{m}. Control is transferred to \main (returning from \cpp{f.resume()} at
line 5) while fiber \cpp{f} is destroyed.

In a more advanced example another \fiber is used as return value instead of the
passed in synthesized fiber.
\cppfl{terminating_fiber_complex}

At line 13 fiber \cpp{f2} is resumed and the lambda is entered at line 8. The
synthesized \fiber\xspace \cpp{f} (representing suspended \main) is passed as a
parameter \cpp{f} and stored in \cpp{m} (captured by the lambda) at line 10.
This is necessary in order to prevent destructing \cpp{f} when the lambda
returns. Fiber \cpp{f2} uses \cpp{f1}, that was also captured by the lambda, as
return value. Fiber \cpp{f2} terminates while fiber \cpp{f1} is resumed (entered
the first time). The synthesized \fiber\xspace \cpp{f} passed into the lambda at line 3
represents the terminated fiber \cpp{f2} (e.g. the calling fiber). Thus object
\cpp{f} is empty as the assert statement verifies at line 5. Fiber \cpp{f1} uses
the captured \fiber\xspace \cpp{m} as return value (line 6). Control is returned to
\main, returning from \cpp{f2.resume()} at line 13.

\zs{The \entryfn passed to \fiber's constructor must have
signature `\cpp{fiber\_context(fiber\_context&&)}`. Using \fiber as the return
value from such a function avoids global variables.}

\uabschnitt{returning synthesized \fiber object from \cpp{resume()}}\label{fiberreturn}
An object of type \fiber remains empty after return from \anyresume: the
synthesized fiber is returned, instead of implicitly updating the \fiber
object on which \resume was called.

If the \fiber object were implicitly updated, the fiber would 
change its identity because each fiber is associated with a stack. Each stack
contains a chain of function calls (call stack). If this association were
implicitly modified, unexpected behaviour happens.

The example below demonstrates the problem:
\cppfl{return_from_resume_inplace}

In this pseudo-code the \fiber object is implicitly updated.

The example creates a circle of fibers: each fiber prints its name and resumes
the next fiber (f1 -> f2 -> f3 -> f1 -> ...).

Fiber \cpp{f1} is started at line 27. The synthesized \fiber\xspace \cpp{main} passed 
to the resumed fiber is stored but not used: control flow cycles through the three
fibers.
The for-loop prints the name \emph{f1} and resumes fiber \cpp{f2}. Inside 
\cpp{f2}'s for-loop the name is printed and \cpp{f3} is resumed. Fiber \cpp{f3}
resumes fiber \cpp{f1} at line 7. Inside \cpp{f1} control returns from
\cpp{f2.resume()}. \cpp{f1} loops, prints out the name and invokes \cpp{f2.resume()}. But
this time fiber \cpp{f3} instead of \cpp{f2} is resumed. This is caused by the
fact that the object \cpp{f2} gets the synthesized \fiber of \cpp{f3} implicitly
assigned. Remember that at line 7 fiber \cpp{f3} gets suspended while \cpp{f1}
is resumed through \cpp{f1.resume()}.

This problem can be solved by returning the synthesized \fiber from \anyresume.
\cppf{return_from_resume_invalid}

In the example above the synthesized \fiber returned by each \resume call is
specifically move-assigned to a \fiber object other than the one on which \resume
was called, to properly track the three fibers. (Of course this particular example
depends on static knowledge of the overall control flow. But the API does not, in
general, require that.)

\zs{The synthesized \fiber must be returned from \allresume
in order to prevent changing the identity of the fiber.}
\xspace\newline

If the overall control flow isn't known, member function \anyresumewith
(see section \nameref{resumewith}) can be used to assign the
synthesized \fiber to the correct \fiber object (held by the caller).
\cppf{assign_resumewith}

Picture a higher-level framework in which every fiber can find its associated
\cpp{filament} object, as well as others. Every context switch must be mediated by
passing \emph{the target} \cpp{filament} object to \emph{the running fiber's}
\cpp{resume\_next()}.

Running fiber A has an associated \cpp{filament} object \cpp{filamentA},
whose \fiber\xspace\cpp{filament::f\_} is empty -- because fiber A is running.

Desiring to switch to suspended fiber B (with associated
\cpp{filament} \cpp{filamentB}), running fiber A calls\\
\cpp{filamentA.resume\_next(filamentB)}.

\cpp{resume\_next()} calls \cpp{filamentB.f\_.resume\_with(<lambda>)}.
This empties \cpp{filamentB.f\_} -- because fiber B is now running.

The lambda binds \cpp{&filamentA} as \cpp{this}. Running on fiber B, it
receives a \fiber object representing the newly-suspended fiber A as its
parameter \cpp{f}. It moves that \fiber object to \cpp{filamentA.f\_}.

The lambda then returns a default-constructed (therefore empty) \fiber
object. That empty object is returned by the previously-suspended
\resumewith call in \cpp{filamentB.resume\_next()} -- which is fine because
\cpp{resume\_next()} drops it on the floor anyway.

Thus, the running fiber's associated \cpp{filament::f\_} is always empty,
whereas the \cpp{filament} associated with each suspended fiber is continually
updated with the \fiber object representing that
fiber.\footnote{\bfiber\cite{bfiber} uses this pattern for resuming user-land
threads.}

\zs{It is not necessary to know the overall control flow. It is sufficient to
pass a reference/pointer of the \emph{caller} (fiber that gets suspended) to the
resumed fiber that move-assigns the synthesized \fiber to \emph{caller} (updating
the object).}

\abschnitt{inject function into suspended fiber}\label{resumewith}
Sometimes it is useful to inject a new function (for instance, to throw an
exception or assign the synthesized fiber to the caller as described in
\nameref{fiberreturn}) into a suspended fiber. For this purpose
\anyresumewith may be called, passing the function \cpp{fn()} to execute.

\cppfl{suspender}

The \resumewith call at line 11 injects function \cpp{fn()} into
fiber \cpp{f} as if the \resume call at line 3 had directly
called \cpp{fn()}.

Like an \entryfn passed to \fiber, \cpp{fn()} must accept
\cpp{std::fiber\_context&&} and return\\
\fiber. The \fiber object returned by \cpp{fn()} will, in turn, be returned
to \cpp{f}'s lambda by the \resume at line 3.

In the example below, suppose that code running on the program's main fiber
calls \resume (line 12), thereby entering the first lambda. This is the point
at which \cpp{m} is synthesized and passed into the lambda at line 2.

Suppose further that after doing some work (line 4), the lambda calls
\cpp{m.resume()}, thereby switching back to the main fiber. The lambda remains
suspended in the call to \cpp{m.resume()} at line 5.

At line 18 the main fiber calls \cpp{f.resume\_with()} where the passed lambda
accepts \cpp{fiber\_context &&}. That new lambda is called on the fiber of the suspended
lambda. It is as if the \cpp{m.resume()} call at line 8 directly called the second
lambda.

The function passed to \resumewith has almost the same range of possibilities as
any function called on \thefiber{\cpp{f}}. Its special invocation
matters when control leaves it in either of two ways:

\begin{enumerate}
  \item If it throws an exception, that exception unwinds all previous stack
        entries in that fiber (such as the first lambda's) as well, back to
        a matching \cpp{catch} clause.\footnote{As stated
        in \nameref{exceptions}, if there is no matching \cpp{catch}
        clause in that fiber, \cpp{std::terminate()} is called.}
  \item If the function returns, the returned \fiber object is returned by
        the suspended \anyresume call.
\end{enumerate}

\cppfl{ontop}

The \cpp{f.resume\_with(<lambda>)} call at line 18 passes control to the second
lambda on the fiber of the first lambda.

As usual, \resumewith synthesizes a \fiber object representing the calling
fiber, passed into the lambda as \cpp{m}. This particular lambda returns \cpp{m}
unchanged at line 21; thus that object \cpp{m} is returned by the \resume call
at line 8.

Finally, the first lambda returns at line 10 the \cpp{m} variable updated at
line 8, switching back to the main fiber.

One case worth pointing out is when you call \anyresumewith on a
\fiber that has not yet been resumed for the first time:
\cppfl{initial_resume_with}

In this situation, \cpp{injected()} is called with a \fiber object
representing the caller of \resumewith. When \cpp{injected()} eventually
returns that (or some other) \fiber object, the returned\\
\fiber object is passed into \cpp{topfunc()} as its \cpp{prev} parameter.

\zs{Member function \anyresumewith allows you to inject a
function into a suspended fiber.}

\abschnitt{passing data between fibers}

Data can be transferred between two fibers via global pointer, a calling
wrapper (like \cpp{std::bind}) or lambda capture.
\cppfl{passing_lambda}

The \resume call at line 8 enters the lambda and passes 1 into the
new fiber. The value is incremented by one, as shown at line 4. The expression
\cpp{caller.resume()} at line 5 resumes the original context (represented
within the lambda by \cpp{caller}).

The call to \cpp{lambda.resume()} at line 10 resumes the lambda, returning from
the \cpp{caller.resume()} call at line 5. The \fiber object \cpp{caller}
emptied by the \resume call at line 5 is replaced with the new object
returned by that same \resume call.

Finally the lambda returns (the updated) \cpp{caller} at line 6, terminating its
context.

Since the updated \cpp{caller} represents the fiber suspended by the call at
line 10, control returns to \main.

However, since fiber \cpp{lambda} has now terminated, the updated \cpp{lambda}
is empty. Its \opbool returns \false.

\zs{Using lambda capture is the preferred way to transfer data between two
fibers; global pointers or a calling wrapper (such as \cpp{std::bind}) are
alternatives.}

\abschnitt{termination}\label{termination}

%% There are a few different ways to terminate a given fiber without
%% terminating the whole process, or engaging undefined behaviour.
%% 
%% When a \fiber object is constructed with an \entryfn, its new stack is
%% initialized with the frame of an implicit top-level function that marks the
%% end of the stack. \unwindfib unwinds the stack back to
%% that top-level function, which resumes the \fiber passed to \unwindfib.
%% 
%% Therefore, any of the following will gracefully terminate a fiber:
%% 
%% \begin{itemize}
%%     \item Cause its \entryfn to return a non-empty \fiber.
%%     \item From within the fiber you wish to terminate, call \unwindfib with a
%%           non-empty \fiber. That fiber will be resumed
%%           when the active fiber terminates.
%%     \item Call \cpp{fiber\_context::resume\_with(unwind\_fiber)}. This is what \dtor
%%           does. Since\\\unwindfib accepts a \fiber, and since \resumewith
%%           synthesizes a\\\fiber representing its caller and passes it to the
%%           subject function, this terminates the fiber referenced by the
%%           original \fiber object and then resumes the caller.
%%     \item Engage \dtor: switch to some other fiber, which will
%%           receive a \fiber object representing the current fiber. Make that
%%           other fiber destroy the received \fiber object.
%% \end{itemize}
%% 
%% The above are all equivalent: stack variables are properly destroyed, since
%% the stack is unwound. (See \nameref{unwinding}.)
%% 
%% In an environment that forbids exceptions, 
Every \fiber you launch must
terminate gracefully by returning from its \entryfn.
%% You may not
%% call \unwindfib. You may not call \dtor, explicitly or implicitly, on a
%% non-empty \fiber object. With these restrictions, it is possible to use
%% the \fiber facility without exception support.

When an explicitly-launched fiber's \entryfn returns a non-empty \fiber
object, the running fiber is terminated. Control switches to the fiber
represented by the returned \fiber object. The \entryfn may return (switch to)
any reachable non-empty \fiber object -- it need not be the object originally
passed in, or an object returned from the \resume family of methods.

\emph{Calling} \resume means: ``Please switch to the specified fiber; I
am suspending; please resume me later.''

\emph{Returning} a particular \fiber means: ``Please switch to the specified
fiber; and by the way, I am done.''

Cancellation of another fiber is not explicitly supported
by \fiber. If it is important for consuming code to communicate
to a suspended fiber the desire that it should terminate, lambda binding may
be used to pass some relevant object, e.g. a \cpp{stop\_token}.

It is up to the code running on the fiber in question to observe and respond
to any such termination request. The fiber must be resumed \emph{after} the
request before it could possibly observe the change. Even then, the \entryfn
might not immediately return.

One tactic would be to request termination, then loop over \anyresume calls until
the returned \fiber is \emptyfn. However, that information is ambiguous.

Suppose we have a \fiber object \cpp{f1} representing suspended fiber F,
with an application-specific termination request mechanism. The running fiber
M requests F to terminate, then calls \cpp{f1.resume()}, which in due course
returns another \fiber object \cpp{f2}.

\cpp{f2} has various possible values.

\begin{itemize}
    \item \cpp{f2} might be empty. This might mean that fiber F did in fact
          terminate.
    \item Alternatively, it might mean that fiber F, instead of terminating,
          resumed fiber G, which terminated by resuming fiber M.
    \item Or fiber F might have terminated by resuming fiber G, which might
          have terminated by resuming fiber M.
    \item In other words, if \cpp{f2} is empty, fiber M cannot know the
          present state of fiber F.
    \item \cpp{f2} might not be empty. That might mean that fiber F did not
          terminate before resuming fiber M. \cpp{f2} would represent fiber F.
    \item Or it might mean that fiber F terminated by resuming fiber G, which
          might have resumed fiber M. \cpp{f2} would represent fiber G.
    \item Or it might mean that fiber F, instead of terminating, resumed fiber
          G, which resumed fiber M. \cpp{f2} would (again) represent fiber G.
    \item In other words, if \cpp{f2} is not empty, fiber M cannot know the
          present state of fiber F.
\end{itemize}

The \cpp{autocancel} class introduced in \nameref{autocancel} illustrates a
possible cancellation implementation, subject to the limitations described
above.

%% The \emph{last} fiber on a particular thread has no non-empty \fiber to
%% return. For this reason, returning an empty \fiber object (\opbool
%% returns \false) terminates the calling thread. This is true whether or
%% not the thread's default fiber (see \nameref{fiber-context.general}) has
%% terminated.

%% \uabschnitt{stack unwinding}\label{unwinding}
%% 
%% Stack unwinding caused by an exception, thread termination or fiber
%% destruction exits functions on that stack without executing a \cpp{return} statement. Local variables
%% that go out of scope may have destructors that must be called.
%% The implementation must walk the stack and call the destructor for each object
%% in every such stack frame.
%% 
%% The C++ standard does not define how exception handling is implemented. Stack unwinding differs
%% among different systems. The process of stack unwinding is described in the
%% system ABI, for instance:
%% \begin{itemize}
%%     \item \emph{.eh\_frame}/\emph{personality routine} on SYS V AMD64 ABI\cite{SYSVAMD64} (de facto standard among Unix-like operating systems)
%%     \item \emph{RUNTIME\_FUNCTION}/\emph{UNWIND\_INFO} on x64 Windows\cite{WinX64}
%%     \item \emph{.pdata}/\emph{.xdata} on ARM64 Windows\cite{WinARM64}
%% \end{itemize}
%% 
%% \paragraph{SYS V AMD64 unwind library}
%% is based on DWARF CFI (call frame information) that are stored in the \emph{.eh\_frame} section.
%% Unwinding happens under following circumstances:
%% \begin{itemize}
%%     \item A C++ exception has been thrown
%%     \item unwinding is forced by an external agent (longjmp for instance)
%% \end{itemize}
%% \uwforced takes a \foreignex (non-C++ exception; for instance Java or GO) and walks the stack frame by frame
%% inspecting the \emph{unwind tables} for cleanup functions (for instance destructors of
%% local variables) and catch blocks.
%% 
%% \uwforced calls a \emph{personality routine} (\cpp{__gxx_personality_v0()} for GCC).\footnote{The
%% personality routine passed by a specific runtime serves as interface between system unwinding library
%% and language specific exception handling (not only C++; GO and Java are also supported). It is always invoked via pointer (saved
%% as a function pointer in \ehframe\xspace for each stack frame).}
%% \uwforced takes a stop function that controls the termination of the unwinding
%% (reaching end of stack for fibers).
%% The stop function intercepts calls to the personality routine, letting the external
%% agent override the defaults of the stack frame's personality routine.\footnote{As
%% a consequence the C++ personality routine deals only with C++ exceptions;
%% it does not need to know anything specific about unwinding done by an external
%% agent such as fiber or pthreads cancellation.}
%% When the destination frame (last frame on fiber
%% stack) is reached, control jumps back to the caller without further popping
%% the stack.
%% 
%% The code snippet below is a proof of concept available at \href{https://github.com/boostorg/context/tree/p0876r6}{Boost.Context branch p0876r6}.
%% \cppf{unwind}
%% \cpp{fiber_unwind()} is called by \unwindfib or \dtor and starts the stack unwinding.
%% The foreign exception \cpp{foreign_unwind_ex}\footnote{setting member variable makes \cpp{foreign_unwind_ex} a foreign exception}
%% is allocated and passed as parameter to the unwinding library. Function \cpp{fiber_unwind_stop()} transfers execution control
%% to the calling fiber once the last stack frame has been unwound.
%% 
%% \subparagraph{non-catchable \foreignex}
%% \unwindfib uses a non-C++ \foreignex to force stack unwinding.
%% As stated in the \emph{SYS V AMD64 ABI}\cite{SYSVAMD64} standard:
%% "A runtime is not allowed to catch an exception if the \cpp{_UA_FORCE_UNWIND} flag was passed to the personality routine."
%% and "... since it is not possible to determine if a given catch clause will re-throw or not without executing it ...", the
%% \foreignex must not be catchable by C++ \cpp{try-catch} blocks.
%% 
%% As a consequence, \curex can not return a \cpp{std::exception\_ptr} pointing
%% to a \foreignex.
%% 
%% In order to detect if stack unwinding is currently in progress\\
%% \uncexs counts the \foreignex.
%% 
%% The rationale for moving to an uncatchable exception is further explained in
%% the \nameref{history}.
%% 
%% The specific characteristics of a \foreignex:
%% 
%% \begin{itemize}
%%     \item Throwing the \foreignex can only be effected by the \fiber
%%     facility. The proposed \unwindfib function is the only way to cause that
%%     explicitly.
%%     \item The ultimate "catch" -- the point at which stack unwinding stops --
%%     is likewise determined by the \fiber facility. There is no explicit syntax
%%     for this.
%%     \item Along the way, as with a normal C++ exception, every object in every
%%     stack frame is destroyed.
%%  \item \catchall clauses along the way are executed, but:
%%  \begin{itemize}
%%      \item \cpp{throw;} resumes stack unwinding, as usual
%%      \item a \catchall clause that does not execute a \cpp{throw;}
%%      statement behaves as if it ends with a \cpp{throw;} statement
%%      \item a \catchall clause that attempts to throw a normal C++
%%      exception engages Undefined Behaviour
%%  \end{itemize}
%%  \item \cpp{catch (}\emph{anything else}\cpp{)}
%%     \item \cpp{catch} clauses along the way are ignored.
%% \end{itemize}
%% 
%% \zs{The system's exception handling, i.e. its unwinding framework, is used to clean up the stack
%% of a fiber by using a foreign exception that is not catchable by C++ \cpp{try-catch} blocks.}
%% 
%% Since unwinding a fiber's stack requires destroying objects declared in stack
%% frames,
%% %% and may involve executing \catchall clauses,
%% it is worth pointing out that destroying a non-empty \fiber on a thread other
%% than the thread on which it was last resumed will run those object destructors
%% %% and \catchall clauses
%% on the thread destroying the \fiber object.
%% 
%% As a consequence, destroying a \fiber object representing a thread's default
%% fiber (see \nameref{fiber-context.general})
%% from any other thread engages Undefined Behaviour.\footnote{One unobvious case
%% would be if a fiber running on non-\main thread \cpp{T} stores a \fiber
%% representing \cpp{T}'s default fiber in a static variable, whether
%% module-scope or function-scope. That variable will be destroyed at program
%% termination, probably on a thread other than \cpp{T}.}
%% 
%% \subparagraph{marker frame}
%% 
%% The \fiber facility behaves as if there is an implicit top-level function above
%% each explicit fiber's \entryfn. (See \nameref{fiber-context.general}) This
%% top-level function serves to delimit stack unwinding. Once the stack has been
%% unwound to that point, it is as if control returns to the implicit top-level function. The
%% implicit top-level function is conceptually responsible for freeing the explicit fiber's
%% stack memory and for resuming the \fiber designated as the next fiber.
%% 
%% \subparagraph{destroying a \fiber representing a thread's default fiber}
%% 
%% Similarly, the C++ runtime behaves as if there is a stack marker at or above \main (and
%% each explicitly-launched thread's \entryfn) that serves to delimit stack
%% unwinding due to the \foreignex. Unlike an explicit fiber's top-level
%% function, though, the conceptual top-level function on a thread's default fiber
%% does \emph{not} deallocate that fiber's stack: the OS, which provided the
%% stack in the first place, will do that.
%% 
%% Unwinding the stack belonging to a thread's default fiber leaves the stack
%% allocated but unreachable. That thread may continue to execute explict fibers
%% as long as desired.
%% 
%% Ultimately, however, it must be possible to exit a fiber in such as way as to
%% terminate the calling thread. Returning an empty \fiber object from
%% a fiber's \entryfn terminates the running thread. Consequently, passing an
%% empty \fiber object to \unwindfib also terminates the calling thread.
%% 
%% The \fiber facility does not defend against the case in which a thread's
%% default fiber suspends (rather than terminating), but the explicit fiber it
%% resumes ultimately causes thread termination in either of the ways described
%% above. A higher-level library built on \fiber can provide a scheduler.
%% The \fiber facility intentionally does not.
%% 
%% The conceptual top-level function above \main, given an empty \fiber object to resume,
%% terminates the whole process instead of that one thread.

\input{exceptions}
%%\abschnitt{stack destruction}\label{destruction}

On construction of a \fiber a stack is allocated. When the \entryfn returns,
the stack will be destroyed. If the function has not yet returned,
the \fiber object representing that fiber must not be destroyed.
%% and the
%% destructor of the \fiber object representing that context is called,
%% the stack will be unwound and destroyed.
%% the calling program is responsible for unwinding the stack.

%%The fiber's \cancelfn is used to trigger cleanup.
%%
%%Consider a running fiber \cpp{f2} that destroys the \fiber object
%%representing \cpp{f1}.
%%
%%\cpp{f1}'s destructor, running on \cpp{f2}, implicitly calls member-function
%%\resumewith, passing the fiber's \cancelfn as
%%argument. Fiber \cpp{f1} will be temporarily resumed and the \cancelfn is
%%invoked.
%%
%%The \cancelfn must communicate to fiber \cpp{f1} the need to terminate. It
%%might throw an exception. It might set a distinguished value in some object
%%tested by code on \cpp{f1}. In any case, fiber \cpp{f2} remains suspended
%%in \cpp{f1}'s destructor until \cpp{f1}'s \entryfn returns (the \fiber
%%object synthesized for) \cpp{f2} -- or until \cpp{f2} is explicitly resumed
%%in some other way.

%% Function \unwindfib caches an object of type \fiber that
%% represents \cpp{f2}, then unwinds \cpp{f1}'s stack
%% (walking the stack and destroying automatic variables in reverse order of
%% construction).
%% The first frame on \cpp{f1}'s stack, the one created by \fiber's constructor,
%% stops the unwinding. It terminates \cpp{f1} by returning
%% \cpp{f2}. Control is returned to \cpp{f2} and \cpp{f1}'s
%% stack gets deallocated.

%% The stack on which \main is executed, as well as the stack implicitly
%% created by \thread's constructor, is allocated by the operating
%% system. Such stacks are recognized by \fiber, and are not deallocated by its
%% destructor.

\abschnitt{\fiber as building block for higher-level frameworks}\label{low_level}

A low-level API enables a rich set of higher-level frameworks that provide
specific syntaxes/semantics suitable for specific domains. As an example, the
following frameworks are based on the low-level fiber switching API of
\bcontext\cite{bcontext} (which implements the API proposed here).

\uabschnitt{\bcoroutine}\cite{bcoroutine2} implements \bfs{asymmetric coroutines}
\cpp{coroutine<>::push_type} and\\
\cpp{coroutine<>::pull_type}, providing a
unidirectional transfer of data. These stackful coroutines are only used in
pairs. When an object of type \cpp{coroutine<>::push_type} is explicitly
constructed, \cpp{coroutine<>::pull_type} is synthesized and passed as
parameter into the coroutine function. In the
example below, \cpp{coroutine<>::push_type} (variable \cpp{writer}) provides the
resume operation, while \cpp{coroutine<>::pull_type} (variable \cpp{in})
represents the suspend operation. Inside the lambda,\cpp{in.get()}
pulls strings provided by \cpp{coroutine<>::push_type}'s output iterator support.
\cppf{bcoroutine_ex}

\uabschnitt{\synca}\cite{synca} (by Grigory Demchenko) is a small, efficient
library to perform asynchronous operations using source code that resembles synchronous
operations. The main
features are a \bfs{GO-like} syntax, support for transferring execution context
explicitly between different thread pools or schedulers (portals/teleports) and
asynchronous network support.
\cppf{synca_ex}

The code itself looks like synchronous invocations while internally it uses
asynchronous scheduling.

\uabschnitt{\bfiber}\cite{bfiber} implements \bfs{user-land threads} and combines
fibers with schedulers (the scheduler algorithm is a customization point). The API
is modelled after the \thread API and contains objects such as
\cpp{future}, \cpp{mutex},\\
\cpp{condition_variable} ...
\cppf{bfiber_ex}

\uabschnitt{Facebook's \fbfibers}\cite{fbfiber} is an asynchronous C++ framework
using \bfs{user-land threads} for parallelism. In contrast to \bfiber,
\fbfibers\xspace exposes the scheduler and permits integration with various
event dispatching libraries.
\cppf{fbfiber_ex}

\fbfibers\xspace is used in many critical applications at Facebook for instance
in \fbmcrouter\cite{fbmcrouter} and some other Facebook services/libraries like
ServiceRouter (routing framework for \fbthrift\cite{fbthrift}), Node API (graph
ORM API for graph databases) ...

\uabschnitt{Bloomberg's \bbquantum}\cite{bbquantum} is a full-featured and
powerful C++ framework that allows users to dispatch units of work (a.k.a.
tasks) as coroutines and execute them concurrently using the 'reactor' pattern.
Its main features are support for streaming futures which allows faster processing
of large data sets, task prioritization, fast pre-allocated memory pools and
parallel \cpp{forEach} and \cpp{mapReduce} functions.
\cppf{bbquantum}

\bbquantum\xspace is used in large projects at Bloomberg.

\uabschnitt{Habanero Extreme Scale Software Research Project\cite{habanero}}
provides a task-based parallel programming model via its \hclib\cite{hclib}.
The runtime provides work-stealing, async-finish,\footnote{async-finish is a
variant of the fork-join model. While a task might fork a group of
child tasks, the child tasks might fork even more tasks. All tasks can
potentially run in parallel with each other. The model allows a parent task to
selectively join a subset of child tasks.}
parallel-for and future-promise parallel programming patterns. The library is not an exascale
programming system itself, but it manages intra-node resources and schedules
components within an exascale programming system.

\uabschnitt{Intel's \tbb}\cite{tbb} internally uses fibers for long running
jobs\footnote{because of the requirement to support a broad range of
architectures \href{https://github.com/intel/tbb/blob/tbb_2020/src/tbb/co_context.h\#L190}
{\swapcontext} was used} as reported by Intel.

\uabschnitt{\userver}\cite{userver} is a modern open source asynchronous
framework with a rich set of abstractions, database connectors/drivers,
protocols and synchronization primitives for fast and comfortable creation
of IO-bound C++ microservices, services and utilities.

\uabschnitt{Alibaba's \photon}\cite{photon} supports a large number of services
and clients, especially the image service of Alibaba’s container platform,
which supports various Internet services for billions of users.
Also used in some ByteDance services.

\uabschnitt{Alibaba's \libeasy}\cite{libeasy} supports a large number of
servers, including storage, database, etc. Not officially open-sourced, but
has been published as part of some open source projects, such as Oceanbase,
tair, etc.

\uabschnitt{Baidu's \bthread}\cite{bthread} has 1 million+ deployed instances
(not counting clients) and thousands of kinds of services.

\uabschnitt{Tencent's \libco}\cite{libco} is a c/c++ coroutine library that
is widely used in backend service of WeChat, which is the largest IM service
in China, with billions of users. 

\uabschnitt{\libgo}\cite{libgo} is developed by Meizu, one of the top mobile
phone vendors in China. Libgo is used in Kiev, Meizu's distributed service
framework for its applications.

\uabschnitt{\statethreads}\cite{state-threads} was first developed by
Netscape, then maintained by SGI and Yahoo!. It is now used in a realtime
media streaming server called \href{https://github.com/ossrs/srs}{SRS}, and
maintained by SRS's developers. \emph{state-threads} was used in the
\href{https://dl.acm.org/doi/10.1145/3302424.3303967}
{distributed block store for Meituan}, another top Internet company in China.

\zs{As shown in this section a low-level API can act as building block for a
rich set of high-level frameworks designed for specific application domains
that require different aspects of design, semantics and syntax.}

\input{stl}
\abschnitt{possible implementation strategies}\label{implementations}

\zs{This proposal does \so{NOT} seek to standardize any particular implementation or
impose any specific calling convention!}

Modern \bfs{micro-processors} are \bfs{register machines}; the content of
processor registers represents the execution context of the program at a given
point in time.

\bfs{Operating systems} maintain for each process all relevant data (execution
context, other hardware registers etc.) in the process table. The operating system's
\bfs{CPU scheduler} periodically suspends and resumes processes in order to
share CPU time between multiple processes. When a process is suspended, its
execution context (processor registers, instruction pointer, stack pointer, ...)
is stored in the associated process table entry. On resumption, the CPU
scheduler loads the execution context into the CPU and the process continues
execution.

The CPU scheduler does a \bfs{full context switch}. Besides preserving
the execution context (complete CPU state), the cache must be invalidated and
the memory map modified.

A kernel-level context switch is several orders of magnitude slower than a
context switch at user-level\cite{Tanenbaum2009}.

\uabschnitt{hypothetical fiber preserving complete CPU state} This strategy tries to
preserve the complete CPU state, e.g. all CPU registers. This requires that the
implementation identifies the concrete micro-processor type and supported processor
features. For instance the x86-architecture has several flavours of extensions
such as MMX, SSE1-4, AVX1-2, AVX-512.

Depending on the detected processor features, implementations of certain
functionality must be switched on or off. The CPU scheduler in the operating system
uses such information for context switching between processes.

A fiber implementation using this strategy requires such a detection mechanism
too (equivalent to swapper/\cpp{system_32()} in the Linux kernel).

Aside from the complexity of such detection mechanisms, preserving the complete
CPU state for each fiber switch is expensive.

\zs{A context switch facility that preserves the complete CPU state like an
operating system is possible but impractical for user-land.}

\uabschnitt{fiber switch using the calling convention}\label{callingconvention}
For \fiber, not all registers need be preserved because the context
switch is effected by a visible function call. It need not be completely transparent like
an operating-system context switch; it only needs to be as transparent as a call
to any other function. The calling convention -- the part of the ABI that
specifies how a function's arguments and return values are passed -- determines
which subset of micro-processor registers must be preserved by the called
subroutine.

The \bfs{calling convention}\cite{SYSVABI} of \bfs{SYSV ABI} for \bfs{x86\_64}
architecture determines that general purpose registers R12, R13, R14, R15, RBX
and RBP must be preserved by the sub-routine - the first arguments are passed
to functions via RDI, RSI, RDX, RCX, R8 and R9 and return values are stored in
RAX, RDX.

So on that platform, the \resume implementation preserves the \bfs{general
purpose registers} (R12-R15, RBX and RBP) specified by the calling convention.
In addition, the \bfs{stack pointer} and \bfs{instruction pointer} are
preserved and exchanged too -- thus, from the point of view of calling
code, \resume behaves like an ordinary function call.

In other words, \resume acts on the level of a simple function invocation --
with the same performance characteristics (in terms of CPU cycles).

This technique is used in \bcontext\cite{bcontext} which acts as building block
for (e.g.) \fbfibers\xspace and \bbquantum; see section \nameref{low_level}.

\uabschnitt{in-place substitution at compile time} During code generation,
a compiler-based implementation could inject the assembler code responsible
for the fiber switch directly into each function that calls \resume. That would save
an extra indirection (JMP + PUSH/MOV of certain registers used to
invoke \resume).

\uabschnitt{CPU state on the stack} Because each fiber must preserve CPU
registers at suspension and load those registers at resumption, some storage
is required.

Instead of allocating extra memory for each fiber, an implementation can use
the stack by simply advancing the stack pointer at suspension and pushing the
CPU registers (CPU state) onto the stack owned by the suspending fiber. When
the fiber is resumed, the values are popped from the stack and loaded into the
appropriate registers.

This strategy works because only a running fiber creates new stack frames
(moving the stack pointer). While a fiber is suspended, it is safe to keep the
CPU state on its stack.

Using the stack as storage for the CPU state has the additional advantage
that \fiber need not itself contain the stored CPU state: it need only contain
a pointer to the stack location.

Section \nameref{synthesizing} describes how global variables are avoided
by synthesizing a \fiber from the active fiber (execution context) and passing
this synthesized \fiber (representing the now-suspended fiber) into the resumed
fiber. Using the stack as storage makes this mechanism very easy to
implement.\footnote{The implementation of \bcontext\cite{bcontext} utilizes this
technique.}
Inside \resume the code pushes the relevant CPU registers onto the stack, and
from the resulting stack address constructs a new \fiber. This object is then
passed (or returned) into the resumed fiber (see \nameref{synthesizing}).

\zs{Using the active fiber's stack as storage for the CPU state is efficient because no
additional allocations or deallocations are required.}

\abschnitt{\exfns}\label{exc_scope}

Both \exfns must report exceptions solely on the current fiber.
Reporting exceptions thrown on any other fiber would make them
unreliable in practice.

A straightforward implementation could make \allresume save and restore the
data underlying \exfns as part of saving and restoring the rest of the fiber
state. Since \exfns data is necessarily thread-local, the likely cost would be
a TLS access on every \anyresume call.

Alternatively, \fiber's constructor could update an internal associative
container whose key is the high end of the new fiber stack area. \exfns could
call \cpp{upper\_bound()}, passing the current stack pointer, to discover
which stack is current. This would shift the cost from every context switch
to \exfns calls.

The examples in \nameref{exlife}, \nameref{throw} and
\nameref{exfns} have been floated to illustrate problems that can
arise when \exfns are not specific to the current fiber.

In those small examples, the problematic code is obvious. But the power of
fibers is that a function need not know whether some function it calls (or
some indirect callee thereof) will resume another fiber. It's not practical
simply to forbid coders from switching fibers within a catch block.

In St. Louis in June 2024, EWG requested\cite{stlouisnotes} implementation
experience with fiber-specific exception state.

In Wrocław in November 2024,\cite{wroclawnotes} we presented
\href{https://github.com/secondlife/3p-boost/blob/nat/exstate/patches/libs/context/0001-switch-exception-state.patch}{a small patch}
to the Boost.Context reference implementation. With that patch, all three
exception state test programs behave correctly when built with libstdc++ on
Windows and Linux. Microsoft questioned whether fiber-specific exception state
is implementable in MSVC, and EWG agreed to take up this matter in Hagenberg.

On February 14, 2025, Gor Nishanov stated\cite{onwindows} that a Windows
Fibers implementation of \fiber would be possible, while expressing concern
about potential performance.



\input{register_window}
\input{performance}
\input{accelerators}
\abschnitt{multi-threading environment}\label{xthread}

Any thread in a program may be shared by multiple fibers.

A newly-constructed fiber is not yet associated with any thread. However,
once a fiber has been resumed the first time by some thread, it must
thereafter be resumed only by that same thread.

There could potentially be Undefined Behaviour if:
\begin{itemize}
    \item a function running on a fiber references \cpp{thread\_local} variables
    \item the compiler/runtime implementation caches a pointer
          to \cpp{thread\_local} storage in that function's stack frame
    \item that fiber is suspended, and
    \item the suspended fiber is resumed on a different thread.
\end{itemize}

The cached TLS pointer is now pointing to storage belonging to some other
thread. If the original thread terminates before the new thread, the cached
TLS pointer is now dangling.

For this reason, it is forbidden to resume a fiber on any thread other than
the one on which it was first resumed.

\input{acknowledgment}
\newpage
\abschnitt{Wording}\label{api}

This wording is relative to N4981.\cite{Standard}

\zs{Append to \stdsection{3.6}{defns.block} as indicated:}

\add{\tsnoten{1 to entry}{Unless stated otherwise, blocking blocks the current
thread.}}

\zs{Modify \stdsection{4.1.2}{intro.abstract} paragraph 7.3 as indicated:}

\begin{description}
    \item[---] The input and output dynamics of interactive devices shall take
               place in such a fashion that prompting output is actually
               delivered before \replace{a program}{an input operation} waits
               for input. What constitutes an interactive device is
               implementation-defined.
\end{description}

\zs{Modify \stdsection{6.9.2.1}{intro.multithread.general} paragraph 1 as indicated:}

A \emph{thread of execution} (also known as a \emph{thread}) is
\replace{a single flow of control}{the primary execution agent}\\
\add{\xref{thread.req.lockable.general}}
within a program\delete{, including the initial invocation of a specific top-level
function, and}\\
\replace{recursively including every function invocation subsequently
executed by the thread}{. When the host environment first}\\
\add{enters a program, it provides a default thread to perform the program's
execution steps}.

\add{When a thread is created, it runs a default fiber ([intro.fibers]).}

\zs{Insert before \stdsection{6.9.3}{basic.start} and renumber existing 6.9.3 to 6.9.4:}

\setcounter{section}{6}
\setcounter{subsection}{9}
\setcounter{subsubsection}{2}
\setcounter{secnumdepth}{4}
\cbstart

\rSec3[intro.fibers]{Fibers and Threads}

1 A \emph{fiber} is a single flow of control within a program, including the
initial invocation of a specific top-level function, and recursively including
every function invocation subsequently executed by the fiber. The execution
steps of a fiber are performed by a thread.

\tsnote{``Flow of control'' here refers to state necessary to program
execution, for example the contents of a processor's registers including its
instruction pointer, and the invocation sequence \xref{stacktrace.general} of
functions that have been entered but have not yet returned.}

2 A thread is always running exactly one fiber. Member functions of \fiber
([fiber.context.class]) can direct the calling thread to \emph{suspend} the
running fiber and \emph{resume} a designated other fiber. This transition from
one fiber to another is a \emph{context switch}.

3 An \emph{implicit fiber} is the default fiber on any thread. All other
fibers are \emph{explicit fibers.}

4 An explicit fiber is created using \fiber. Constructing a \fiber object \emph{prepares} a
fiber, which can consume resources. A fiber can thus be in one of three
states: prepared, running or suspended.

5 When a thread first enters a prepared fiber, that thread becomes the
fiber's \emph{owning thread.} The owning thread never changes.
\tsnote{A thread is the owning thread of its default fiber.}
\tsnote{If a thread resumes a fiber owned by another thread, the behaviour is
undefined.}
\cbend

\zs{Modify \stdsection{14.2}{except.throw} paragraph 2 as indicated:}

When an exception is thrown, control is transferred to the nearest handler
with a matching type \xref{except.handle}; ``nearest'' means the handler for
which the \nt{stmt.block}{compound-statement} or
\nt{class.base.init}{ctor-initializer} following the \cpp{try} keyword was
most recently entered by the \replace{thread of control}{running fiber} and
not yet exited.

\zs{Modify \stdsection{14.2}{except.throw} paragraph 4 Note 3 as indicated:}

\tsnoten{3}{A thrown exception does not propagate to other
\replace{threads}{fibers} unless caught, stored, and rethrown using
appropriate library functions; see \stdclause{propagation} and \stdclause{futures}.}

\zs{Modify \stdsection{14.4}{except.handle} paragraph 6 as indicated:}

If no match is found among the handlers for a try block, the search for a
matching handler continues in a dynamically surrounding try block of the same
\replace{thread}{fiber}.

\zs{Modify \stdsection{14.4}{except.handle} paragraph 8 as indicated:}

8 The exception with the most recently activated handler
\add{in the running fiber ([intro.fibers])} that is still active is called the
\emph{currently handled exception}.

\zs{Modify \stdsection{14.6.3}{except.uncaught} paragraph 1 as indicated:}

... The function \stdterm{\cpp{std::uncaught\_exceptions}}{uncaught.exceptions}
returns the number of uncaught exceptions in the
\replace{current thread}{running fiber ([intro.fibers])}.

\zs{Insert new final subclause in clause 33 \stdclause{thread} as indicated:}

\setcounter{section}{33}
\setcounter{subsection}{11}
\setcounter{secnumdepth}{4}

\cbstart

\rSec2[fiber.context]{fiber\_context}

\rSec3[fiber.context.overview]{Overview}

1 A \fiber object is either \emph{empty} or \emph{non-empty}. A
default-constructed or moved-from \fiber is empty. Otherwise, a \fiber is
non-empty, and represents either a prepared or a suspended fiber.

2 An explicit fiber is prepared by passing an \emph{\entryfn} to \fiber's
constructor. At the first call to one of the \anyresume member functions,
that \entryfn is entered, and the fiber is running.

3 Every call to one of the \anyresume member functions on an accessible
non-empty \fiber object performs a context switch.
\begin{itemize}
    \item suspends the running fiber, making it the \emph{previous fiber}
    \item resumes \thisfiber, which was either prepared or suspended, making
          it the running fiber.
\end{itemize}
In addition, returning a non-empty \fiber from a fiber's \entryfn:
\begin{itemize}
    \item terminates the running fiber
    \item resumes \thefiber{the returned \fiber}.
\end{itemize}

4 When a prepared fiber is first entered, a synthesized non-empty \fiber
object representing the previous fiber is passed as a parameter to
its \entryfn. When a suspended fiber is resumed, a synthesized \fiber object
representing the previous fiber is returned from the relevant \anyresume
member function.
\tsnote{The synthesized \fiber object received in either of those ways might
represent either an explicit fiber or an implicit fiber.}

%% \rSec3[fiber.context.toplevel]{Implicit Top-Level Function}

%% On every explicit fiber, the behaviour is equivalent to calling the \entryfn
%% passed to \fiber's constructor from an implicit top-level function.
%% If the fiber is later
%% unwound, this conceptual top-level stack frame serves as delimiter: this point
%% is where unwinding stops.

5 When a running fiber returns a \fiber from its \entryfn, thus resuming the
designated fiber, the synthesized \fiber passed into the resumed fiber is
empty.

6 If a fiber's \entryfn returns an empty \fiber object, \cpp{std::terminate} is called.
If a fiber's \entryfn exits via an exception, \cpp{std::terminate} is called.

7 Regardless of the number of \fiber objects in the program, exactly one of them
represents each prepared or suspended fiber. No \fiber object represents a running fiber.

8 A \fiber object can optionally be constructed by passing an explicit
\cpp{span<byte>} in which to track the fiber's
\stdterm{invocation sequence}{stacktrace.general}. If at any time during the
life of a fiber the data storage required to track its invocation sequence
exceeds the \cpp{size()} of that \cpp{span<byte>}, the behaviour is undefined.

%% Returning a \fiber object from the explicit fiber's \entryfn is equivalent
%% to returning control to the implicit top-level function.
%% Similarly,
%% when \unwindfib unwinds a fiber stack, it conceptually returns the \fiber
%% object it was passed to the implicit top-level function. Either way, the
%% The
%% conceptual implicit top-level function is responsible for deallocating the
%% explicit fiber's stack memory on return from the \entryfn.
%% 
%% Similarly, on every implicit fiber, the behaviour is equivalent to passing control through an
%% implicit top-level function above \justmain and above the \entryfn for
%% each \thread.
%% The conceptual stack frame for this implicit top-level function delimits
%% stack unwinding for each of these stacks. If the fiber stack is unwound,
%% control is conceptually returned to this implicit top-level function.
%% The conceptual top-level
%% function for an implicit fiber does not deallocate the fiber's stack memory,
%% since the host environment will do that.

%% \begin{itemize}
%%     \item
%%     \item If an empty \fiber object is returned to the conceptual top-level
%%     function for an explicit fiber, the calling thread is terminated.
%%     \item If an empty \fiber object is returned to the conceptual top-level
%%     function for the default fiber of an explicit thread, that thread is
%%     terminated.
%%     \item If an empty \fiber object is returned to the conceptual top-level
%%     function above \justmain, the process is terminated.
%% \end{itemize}

%--------------------------------- synopsis ----------------------------------
\rSec3[fiber.context.syn]{Header <fiber\_context> synopsis}

\cppf{synopsis}

%--------------------------------- class def ---------------------------------
\rSec3[fiber.context.class]{Class fiber\_context}

\cppf{fiber}

\newcommand{\state}{\cpp{state}}

\rSec4[fiber.context.cons]{Constructors, move and assignment}

%---------------------------- implicit stack ctor ----------------------------
\mbrhdr{template<class F> explicit fiber\_context(F\&\& entry)}

1 \constraints
\begin{description}
    \item[---] \cpp{remove\_cvref\_t<F>} is not the same type as \fiber.
\end{description}

2 \mandates
\begin{description}
    \item[---] \cpp{is\_constructible\_v<decay\_t<F>, F>} is \true.
    \item[---] \cpp{is\_invocable\_r\_v<fiber\_context, decay\_t<F>,
               fiber\_context&&>} is \true.
\end{description}

3 \effects
\begin{description}
    \item[---] Let \cpp{entry\_copy} be an object of
               type \cpp{decay\_t<F>} direct-non-list-initialized
               with \cpp{std::forward<F>(entry)}. 
    \item[---] Initializes \cpp{state} to prepare a fiber that will, when
               first resumed, enter \cpp{entry\_copy}.
               \tsnote{\cpp{entry\_copy} is not a member of \fiber because it
               is destroyed on fiber termination, not when a \fiber object is
               destroyed. Storage for \cpp{entry\_copy} is associated
               with \cpp{state}.}
    \item[---] Any necessary resources are created. \tsnote{This includes
               storage for the new fiber's invocation sequence.}
    \item[---] The prepared fiber has no owning thread.
\end{description}

4 \postcond
\emptyfn is \false.

5 \except
\begin{description}
    \item[---] \cpp{bad\_alloc} if unable to allocate storage while preparing
               the new fiber.
    \item[---] \cpp{system\_error} if unable to prepare the new fiber for any
               other reason.
    \item[---] Any exception from initialization of \cpp{entry\_copy}.
\end{description}

6 \errors
\cpp{resource\_unavailable\_try\_again} -- the system lacked the necessary resources to prepare another fiber.

%---------------------------- explicit stack ctor ----------------------------
\mbrhdr{template<class F, class D> fiber\_context(F\&\& entry, span<byte> stack, D\&\& deleter)}

1 \mandates
\begin{description}
    \item[---] \cpp{is\_constructible\_v<decay\_t<F>, F>} is \true.
    \item[---] \cpp{is\_constructible\_v<decay\_t<D>, D>} is \true.
    \item[---] \cpp{is\_invocable\_r\_v<fiber\_context, decay\_t<F>,
               fiber\_context&&>} is \true.
    \item[---] \cpp{is\_invocable\_v<decay\_t<D>, span<byte>>} is \true.
\end{description}

2 \precond
\begin{description}
    \item[---] \cpp{decay\_t<D>} meets the \emph{Cpp17MoveConstructible} requirements.
    \item[---] \cpp{invoke(deleter, stack)} does not throw an exception.
\end{description}

3 \effects
\begin{description}
    \item[---] Let \cpp{entry\_copy} be an object of
               type \cpp{decay\_t<F>} direct-non-list-initialized
               with \cpp{std::forward<F>(entry)}. 
    \item[---] Let \cpp{stack\_copy} be a copy of \cpp{stack}.
               \tsnote{It might be advantageous to obtain from the host
               environment a memory block with a read-only guard page to trap
               stack overflow.}
    \item[---] Let \cpp{deleter\_copy} be an object of
               type \cpp{decay\_t<D>} direct-non-list-initialized
               with \cpp{std::forward<F>(deleter)}. 
    \item[---] Initializes \cpp{state} to prepare a fiber that will, when
               first resumed, enter \cpp{entry\_copy}.
               \tsnote{\cpp{entry\_copy}, \cpp{stack\_copy} and
               \cpp{deleter\_copy} are not members of \fiber because they 
               are destroyed on fiber termination, not when a \fiber object is
               destroyed. Storage for \cpp{entry\_copy}, \cpp{stack\_copy} and
               \cpp{deleter\_copy} is associated with \cpp{state}.}
    \item[---] Any necessary resources are created.
    \item[---] The prepared fiber has no owning thread.
\end{description}

4 \postcond
\emptyfn is \false.

5 \except
\begin{description}
    \item[---] \cpp{invalid\_argument} if \cpp{stack.data()} fails to meet
               implementation-defined alignment requirements.
    \item[---] \cpp{length\_error} if \cpp{stack.size()} is less than the
               implementation-defined minimum length.
    \item[---] \cpp{system\_error} if unable to prepare the new fiber.
    \item[---] Any exception from initialization of \cpp{entry\_copy}.
    \item[---] Any exception from initialization of \cpp{deleter\_copy}.
\end{description}

6 \errors
\cpp{resource\_unavailable\_try\_again} -- the system lacked the necessary resources to prepare another fiber.

%--------------------------------- move ctor ---------------------------------
\mbrhdr{fiber\_context(fiber\_context\&\& other) noexcept}

1 \effects
Initializes \cpp{state} with \cpp{exchange(other.state, nullptr)}.

%----------------------------------- dtor ------------------------------------
\mbrhdr{\cpp{\~fiber\_context()}}

1 \effects
If \emptyfn is \false, \cpp{terminate} is invoked \xref{except.terminate}.

%------------------------------ move assignment ------------------------------
\mbrhdr{fiber\_context\& operator=(fiber\_context\&\& other) noexcept}

1 \effects
\begin{description}
    \item[---] If \emptyfn is \false, \cpp{terminate} is invoked \xref{except.terminate}.
    \item[---] Equivalent to: \cpp{this->state = exchange(other.state, nullptr)}.
\end{description}

2 \returns
\this

\rSec4[fiber.context.mem]{Members}
%-------------------------------- resume_with --------------------------------
\mbrhdr{template<class Fn> fiber\_context resume\_with(Fn\&\& fn) \&\&}

The operation of \resumewith involves at least two and possibly three fibers.
Within [fiber.context.mem], for exposition only:

\begin{itemize}
    \item Entering \resumewith performs a context switch.
    \item The \emph{calling fiber} is the fiber calling \resumewith.
    \item The \emph{target fiber} is \thefiber{\state}.
    \item \resumewith synthesizes a \fiber object representing the calling
          fiber. Let \cpp{caller} be that synthesized \fiber object.
    \item Because \resumewith suspends the calling fiber, return
          from \resumewith necessarily requires some other fiber to perform a
          subsequent context switch back to the original calling fiber.
          When \resumewith returns, that other fiber is the previous fiber.
          \tsnote{The previous fiber can be other than the target fiber.}
    \item Let \cpp{previous} be the synthesized \fiber object representing the
          suspended previous fiber.
\end{itemize}

At entry to \resumewith, the target fiber can either be in the prepared state
(not yet entered) or in the suspended state (waiting to return from \resumewith).

\begin{description}
    \item[---]
          If the running fiber is suspended, that implies that at some earlier
          time, it called \resumewith[other], where \cpp{other} was some
          non-empty \fiber object. In that case, let
          exposition-only \emph{internal-resume(\cpp{before})},
          where \cpp{before} is a \fiber object, denote the following sequence
          of steps:
        \begin{itemize}
            \item Return \cpp{before} from \resumewith[other].
        \end{itemize}   
    \item[---] Otherwise, let \emph{internal-resume(\cpp{before})}
          denote the following sequence of steps:
        \begin{itemize}
            \item Execute
                  \cpp{invoke\_r<fiber\_context>(entry\_copy, std::move(before))}
                  and let \cpp{successor} be the resulting \fiber, then
            \item destroy \cpp{entry\_copy}, then
            \item if \cpp{stack\_copy} and \cpp{deleter\_copy} exist:
                \begin{itemize}
                    \item execute \cpp{invoke(deleter\_copy, stack\_copy)}, then
                    \item destroy \cpp{deleter\_copy}, then
                \end{itemize}
            \item exit the running fiber, then
            \item reclaim implementation-provided resources, then
            \item direct the current thread to resume \thefiber{\cpp{successor}}, then
            \item execute \emph{internal-resume(\cpp{fiber\_context()})}.
        \end{itemize}
\end{description}

1 \mandates
\cpp{is\_invocable\_r\_v<fiber\_context, decay\_t<Fn>, fiber\_context&&>} is \true.

2 \precond
\canresume is \true.

3 \effects
\begin{description}
    \item[---] Resets \state so that \emptyfn is \true.
    \item[---] Directs the current thread to suspend the calling fiber and resume
               the target fiber.
    \item[---] Associates the calling thread as the target fiber's owning thread.
    \item[---] Evaluates \cpp{invoke\_r(std::forward<Fn>(fn), std::move(caller))}.
               Let \cpp{returned} be the \fiber object returned by \cpp{fn}.
               \tsnote{\cpp{returned} can be other than \cpp{caller}.
               \cpp{returned} can be empty.}
    \item[---] Executes \emph{internal-resume(\cpp{returned})}.
\end{description}

4 \returns

\begin{description}
    \item[---] If the previous fiber resumed the calling fiber by returning
          a \fiber object representing the calling fiber, an empty \fiber.
    \item[---] If the previous fiber resumed the calling fiber by
          calling \cpp{resume\_with(somefn)}, the \fiber object returned
          by \cpp{invoke\_r<fiber\_context>(somefn, std::move(previous))}.
\end{description}

5 \except

If the previous fiber resumed the calling fiber by calling \cpp{resume\_with(somefn)}:
\begin{itemize}
    \item Any exception thrown by \cpp{invoke\_r<fiber\_context>(somefn,
          std::move(previous))}.
\end{itemize}

\tsnote{\resumewith throws nothing before suspending the calling fiber and
ensuring \emptyfn is \true.} 

6 \postcond
\emptyfn is \true.

\tsnote{Because \anyresume empties the object on which it is called, these
member functions are rvalue-reference qualified.}

%---------------------------------- resume -----------------------------------
\mbrhdr{fiber\_context resume() \&\&}

1 \effects
Equivalent to:\\
\cpp{return resume\_with(identity());}

%-------------------------------- can_resume ---------------------------------
\mbrhdr{bool can\_resume() const noexcept}

1 \returns
\begin{description}
    \item[---] \false if \emptyfn is \true
    \item[---] \true if \thisfiber is in the prepared state (has no owning thread)
    \item[---] \true if \currthread is \ownthread
    \item[---] \false otherwise.
\end{description}

%----------------------------------- empty -----------------------------------
\mbrhdr{bool empty() const noexcept}

1 \effects
Equivalent to: \cpp{return (\! state);}

%------------------------------- operator bool -------------------------------
\mbrhdr{explicit operator bool() const noexcept}

1 \effects
Equivalent to: \cpp{return (\! empty());}

%----------------------------------- swap ------------------------------------
\mbrhdr{void swap(fiber\_context\& other) noexcept}

1 \effects
Equivalent to: \cpp{swap(this->state, other.state)}.

\rSec4[fiber.context.special]{Specialized algorithms}
\mbrhdr{friend void swap(fiber\_context\& lhs, fiber\_context\& rhs) noexcept}

1 \effects
Equivalent to: \cpp{lhs.swap(rhs)}.


%% \rSec3[fiber.context.unwinding]{Function unwind\_fiber()}
%% 
%% \mbrhdr{[[ noreturn ]] void unwind\_fiber(fiber\_context\&\& other)}
%% 
%% 1 \effects
%% terminate the running fiber.
%% 
%% 2 \remarks
%% \begin{description}
%%     \item[---] The underlying Unwinding facility (for instance the unwind facility
%%                described in \emph{System V ABI for AMD64}) unwinds the stack
%%                to the implicit top-level stack frame and terminates the
%%                running fiber as described above.
%%     \item[---] Unwinding the fiber's stack causes its stack variables to be
%%                destroyed.
%%     \item[---] During this specific stack unwinding, 
%% %% only \catchall clauses are executed. No other
%%                no \cpp{catch} clauses are executed, not even \catchall.
%%     \item[---] Once the running fiber has been fully unwound, \cpp{other} is
%%                returned to the fiber's conceptual top-level function as
%%                described in \nameref{fiber-context.toplevel}.
%% %%  \item[---] Unwinding the fiber's stack causes relevant \catchall
%% %%             clauses to be executed.
%% %%  \item[---] During this specific stack unwinding, a \catchall
%% %%             clause that does not execute a \cpp{throw;} statement behaves
%% %%             as if it ended with a \cpp{throw;} statement.
%% %%  \item[---] During this specific stack unwinding, if a \catchall
%% %%             clause attempts to throw any C++ exception, the
%% %%             behaviour is undefined.
%% \end{description}
%% 
%% 3 \returns
%% \begin{description}
%%     \item[---] None: \unwindfib does not return
%% \end{description}
%% 
%% 4 \except
%% \begin{description}
%%     \item[---] None catchable by C++
%% \end{description}

\cbend

\zs{Modify \stdsection{19.6.1}{stacktrace.general} as indicated:}

1 Subclause \stdclause{stacktrace} describes components that C++ programs may use to
store the stacktrace of the \delete{current thread of}\\
\replace{execution}{running fiber ([intro.fibers])}
and query information about the stored stacktrace at runtime.

2 The \emph{invocation sequence} of the current evaluation $x_0$
in the \replace{current thread of execution}{running fiber} is a sequence
($x_0$,...,$x_n$) of evaluations such that, for $i \geq 0$,
$x_i$ is within the function invocation $x_{i+1}$ \xref{intro.execution}.

\abschnitt{Header File}

\zs{Add a new header file to Table 24 in \stdsection{16.4.2.3}{headers}:}

\add{\cpp{<fiber\_context>}}

\abschnitt{Feature-test Macro}
\zs{Add a new feature-test macro to \stdsection{17.3.2}{version.syn} as indicated:}

\add{\cpp{#define \__cpp\_lib\_fiber\_context 202XXXL // also in <fiber\_context>}}

\newpage
\abschnitt{Appendix A: potential premature destruction of exception object}\label{exlife}

In \stdclause{except.throw} paragraph 4, the destruction of an exception
object is specified to potentially occur when an active handler for the
exception exits, not when a handler exits while the exception is still the
currently handled exception. With a Boost
implementation which predates the proposed changes to \stdclause{except}
(in an Itanium C++ ABI environment), it is possible to observe cases where an exception is
destroyed at a different point than specified (and, in particular, when a
handler for the exception is still active in a fiber). Consider
\href{https://github.com/secondlife/3p-boost/blob/nat/exstate/tests/early_exc_destroy.cpp}{the following program}.

\cppf{early_exc_destroy}

\newpage
\abschnitt{Appendix B: throw-expression with no operand}\label{throw}

Both \stdclause{expr.throw} paragraph 3 and \cpp{current\_exception()}
(\stdclause{propagation} paragraph 8) reference the ``currently handled
exception'' (\stdclause{except.handle} paragraph 8). Thus, the
construct \cpp{throw;} is by definition equivalent to\\
\cpp{std::rethrow\_exception(std::current\_exception());}
(\stdclause{propagation} paragraph 9).

The existing definition of currently handled exception:

``The exception with the most recently activated handler that is still active
is called the \emph{currently handled exception.}''

does not clearly constrain the scope to the current thread. This constraint
must be inferred from \stdclause{except.throw} paragraph 2:

``When an exception is thrown, control is transferred to the nearest handler
with a matching type \xref{except.handle}; ``nearest'' means the handler for
which the \nt{stmt.block}{compound-statement} or
\nt{class.base.init}{ctor-initializer} following the \cpp{try} keyword was
most recently entered by the thread of control and not yet exited.''

This is the reason for the proposed changes to \stdclause{except}.
If ``currently handled exception'' means the exception with the
most recently activated handler within any fiber on the current thread, we can get
\href{https://github.com/secondlife/3p-boost/blob/nat/exstate/tests/nullary_throw.cpp}{the following result}.

\cppfl{nullary_throw}

Worse still, the exceptions in question aren't necessarily related to each
other, and line 36 is more likely to read \cpp{catch (const Bad& caught)} --
in which case the \cpp{throw;} on line 34 would \emph{not} be caught.

\input{exfns}
\newpage
\abschnitt{Appendix D: support code for examples}\label{autocancel}

Destroying a non-empty \fiber object invokes Undefined Behaviour
(see \nameref{termination}). To simplify code examples in this paper, we
introduce an \cpp{autocancel} wrapper class that launches a fiber and tracks
the sequence of \fiber objects representing that fiber. When
an \cpp{autocancel} object is destroyed, it sets a stop flag and loops until
the fiber voluntarily terminates.

\cppf{autocancel}

\input{references}

%//////////////////////////////////////////////////////////////////////////////

\end{document}
